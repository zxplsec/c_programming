\section{关系运算符与逻辑运算符}

\subsection{关系运算符}

\begin{frame}\ft{\subsecname}
关系运算符用于比较两个值。\\[.1in]
\begin{enumerate}
\item
  运算符 \lstinline|==|  检查两个给定的操作数是否相等。若相等,返回 \lstinline|true|;否则返回 \lstinline|false|。\\[.1in]
\item[] 如 \lstinline|5 == 5| 返回\lstinline|true|。\\[.1in]
\item
  运算符 \lstinline|!=|  检查两个给定的操作数是否相等。若不相等,返回\lstinline|true|;否则返回\lstinline|false|。\\[.1in]
\item[] 如 \lstinline|5 != 5| 返回\lstinline|false|。
\end{enumerate}
\end{frame}

\begin{frame}\ft{\subsecname}
  \begin{enumerate}
\item[3.]
  运算符 \lstinline|>| 检查第一个操作数是否大于第二个操作数。若成立,返回\lstinline|true|;否则返回\lstinline|false|。\\[.1in]
\item[] 如 \lstinline|6 > 5| 返回\lstinline|true|。\\[.1in]
\item[4.]
  运算符 \lstinline|<| 检查第一个操作数是否小于第二个操作数。若成立,返回\lstinline|true|;否则返回\lstinline|false|。\\[.1in]
\item[] 如 \lstinline|6 < 5| 返回\lstinline|false|。
\end{enumerate}
\end{frame}

\begin{frame}\ft{\subsecname}
  \begin{enumerate}
\item[5.]
  运算符 \lstinline|>=| 检查第一个操作数是否大于或等于第二个操作数。若成立,返回\lstinline|true|;否则返回\lstinline|false|。\\[.1in]
  \item[] 如 \lstinline|5 >= 5| 返回\lstinline|true|。\\[.1in]
\item[6.]
  运算符 \lstinline|<=| 检查第一个操作数是否小于或等于第二个操作数。若成立,返回\lstinline|true|;否则返回\lstinline|false|。\\[.1in]
\item[] 如 \lstinline|5 <= 5| 返回\lstinline|true|。
\end{enumerate}
  
\end{frame}

\begin{frame}[fragile,allowframebreaks]\ft{\subsecname}
\lstinputlisting[language=c,backgroundcolor=\color{red!10},numbers=left]{ch05/code/rel_operand.c}
  
\end{frame}
 

\begin{frame}[fragile]\ft{\subsecname}
\begin{lstlisting}[backgroundcolor=\color{red!10}]
a = 10, b = 4
a > b
a >=  b
a >= b
a > b
a != b
a != b
\end{lstlisting}  
\end{frame}


\subsection{逻辑运算符}
\begin{frame}
  逻辑运算符用于连接两个及以上条件,或对原条件取否。\\[.1in]
\begin{enumerate}
\item
  \blue{逻辑与}: 当两个条件同时满足时,运算符 \lstinline|&&| 返回\lstinline|true|;否则返回 \lstinline|false|。\\[.1in]
\item[] 如,当 \lstinline|a| 和 \lstinline|b| 均为 \lstinline|true| (即非零)时,\lstinline|a && b| 返回\lstinline|true|。\\[.1in]
\item
  \blue{逻辑或}:当至少有一个条件满足时,运算符 \lstinline|\|\|| 返回\lstinline|true|;否则返回 \lstinline|false|。\\[.1in]
  如,当 \lstinline|a| 和 \lstinline|b| 至少有一个为 \lstinline|true| (即非零)时,\lstinline|a \|\| b| 返回\lstinline|true|。当然,当 \lstinline|a| 和 \lstinline|b| 均为 \lstinline|true| 时, \lstinline|a \|\| b|返回\lstinline|true|。\\[.1in]
\item
  \blue{逻辑非}:当条件不满足时,运算符 \lstinline|!| 返回 \lstinline|true| ;否则返回 \lstinline|false| 。\\[.1in]
如,若 \lstinline|a| 为 \lstinline|false| 时,\lstinline|a| 返回 \lstinline|true|。
\end{enumerate}
  
\end{frame}

\begin{frame}[fragile,allowframebreaks]\ft{\subsecname}
\lstinputlisting[language=c,backgroundcolor=\color{red!10},numbers=left]{ch05/code/logic_operand.c}
  
\end{frame}


\begin{frame}[fragile]\ft{\subsecname}

\begin{lstlisting}[backgroundcolor=\color{red!10},frame=no]
AND condition not satisfied
a is greater than b OR c is equal to d
a is not zero  
\end{lstlisting}
\end{frame}

\subsection{逻辑运算符中的短路现象}

\begin{frame}[fragile]\ft{\subsecname} 
对于逻辑与,若第一个操作数为 \lstinline|false| ,则第二个操作数将不会被计算。

\lstinputlisting[language=c,backgroundcolor=\color{red!10},numbers=left]{ch05/code/logic_short_circuit1.c}
\vspace{.1in}\pause 

该程序不会打印 \lstinline|Hello|。
\end{frame}

\begin{frame}[fragile]\ft{\subsecname} 
但下面的程序将打印 \lstinline|Hello|。
\lstinputlisting[language=c,backgroundcolor=\color{red!10},numbers=left]{ch05/code/logic_short_circuit2.c}
\end{frame}

\begin{frame}[fragile]\ft{\subsecname} 
对于逻辑或,若第一个操作数为 \lstinline|true| ,则第二个操作数不会被计算。

\lstinputlisting[language=c,backgroundcolor=\color{red!10},numbers=left]{ch05/code/logic_short_circuit3.c}
\vspace{.1in} \pause 

该程序不会打印 \lstinline|Hello|。
\end{frame}

\begin{frame}[fragile]\ft{\subsecname} 

但下面的程序将打印 \lstinline|Hello|。 
\lstinputlisting[language=c,backgroundcolor=\color{red!10},numbers=left]{ch05/code/logic_short_circuit4.c}

\end{frame}
