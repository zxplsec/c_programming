%%%%%%%%%%%%%%%%%%%%%%%%%%%%%%%%%%%%%%%%%%%%%%%%%%%%%%%%%%%%%%%%%%%%
\subsection{典型题型3~~(非齐次线性方程组)}

\begin{frame}\ft{\subsecname}
  
    \begin{li}[$\bigstar\bigstar\bigstar\bigstar\bigstar$]
      设有线性方程组
      $$
      \left\{
      \begin{array}{rrrcr}
        (1+\lambda)x_1&+x_2&+x_3&=&0\\[0.05in]
        x_1&+(1+\lambda)x_2&+x_3&=&3\\[0.05in]
        x_1&+x_2&+(1+\lambda)x_3&=&\lambda
      \end{array}
      \right.
      $$
      问$\lambda$取何值时,此方程组
      \begin{itemize}
      \item[(1)]有唯一解?
      \item[(2)]无解? 
      \item[(3)]有无穷多解? 并在有无穷多解时求其通解。
      \end{itemize}
    \end{li}
    \pause

    \begin{jie}
    $$
    |\MA|=\left|
    \begin{array}{ccc}
      1+\lambda&1&1\\
      1&1+\lambda&1\\
      1&1&1+\lambda
    \end{array}
    \right| = (3+\lambda)\lambda^2.
    $$
    故当$\lambda\ne0$且$\lambda\ne-3$时,有唯一解。
  \end{jie}
\end{frame}



\begin{frame}\ft{\subsecname}
  
    当$\lambda=0$时,原方程组为
    $$
    \left\{
    \begin{array}{l}
      x_1+x_2+x_3=0,\\
      x_1+x_2+x_3=3,\\
      x_1+x_2+x_3=0      
    \end{array}
    \right.
    $$
    它为矛盾方程组,故无解。\pause \vspace{0.1in}

    当$\lambda=-3$时,增广矩阵为
    $$
    \left(
    \begin{array}{rrrr}
      -2&1&1&\red{0}\\
      1&-2&1&\red{3}\\
      1&1&-2&\red{-3}
    \end{array}
    \right) \xlongrightarrow[]{\mbox{初等行变换}}
    \left(
    \begin{array}{rrrr}
      1&0&-1&\red{-1}\\
      0&1&-1&\red{-2}\\
      0&0&0&\red{0}
    \end{array}
    \right)
    $$ \pause 
    得同解方程组为
    $$
    \left\{
    \begin{array}{l}
      x_1=x_3-1\\[0.05in]
      x_2=x_3-2\\[0.05in]
      x_3=x_3
    \end{array}
    \right.
    $$ \pause 
    通解为
    $$
    \left(
    \begin{array}{c}
      x_1\\x_2\\x_3
    \end{array}
    \right) = c\left(
    \begin{array}{c}
     1\\1\\1
    \end{array}
    \right)+\left(
    \begin{array}{r}
      -1\\-2\\0
    \end{array}
    \right) \quad c\in\mathbb R
    $$
  
\end{frame}




\begin{frame}\ft{\subsecname}
  
    \begin{li}[2005-2006第一学期]
      已知$\MA\vx=\vb$,其中$\MA=\left(
      \begin{array}{ccc}
        2-\lambda & 2 & -2\\
        2 & 5-\lambda & -4\\
        -2 & -4 & 5-\lambda
      \end{array}
      \right), \vb=\left(
      \begin{array}{c}
        1\\2\\-1-\lambda
      \end{array}
      \right)$,问$\lambda$为何值时,该方程组有唯一解、无解或无穷多解?并在有无穷多解时求其解。
    \end{li}
\end{frame}


\begin{frame}\ft{\subsecname}
    \begin{li}[2005-2006第二学期;2006-2007第二学期]
      设线性方程组$\left\{
      \begin{array}{rcl}
        \lambda x_1+x_2+x_3&=&0\\
        x_1+\lambda_2+x_3&=&3\\
        x_1+x_2+x_3&=&\lambda-1
      \end{array}
      \right.$,
      问$\lambda$为何值时,此方程组有唯一解、无解或无穷多个解?并在无穷多解时求出其通解。
    \end{li}
  
\end{frame}


\begin{frame}\ft{\subsecname}
  
    \begin{li}[2006-2007第一学期]
      当$a,b$为何值时,方程组
      $
      \left(
      \begin{array}{ccc}
        1&1&2\\
        1&0&1\\
        5&3&a+8
      \end{array}
      \right)\left(
      \begin{array}{c}
        x_1\\x_2\\x_3
      \end{array}
      \right)=\left(
      \begin{array}{c}
        1\\2\\b+7
      \end{array}
      \right)
      $
      有唯一解、无解或无穷多解?在有解时,给出方程组的解。
    \end{li}
\end{frame}


\begin{frame}\ft{\subsecname}    
    \begin{li}[2007-2008第一学期,2010-2011第二学期]
      设线性方程组$\left\{
      \begin{array}{rcl}
        \lambda x_1+x_2+x_3&=&\lambda-3\\
        x_1+\lambda x_2+x_3&=&-2\\
        x_1+x_2+\lambda x_3&=&-2
      \end{array}
      \right.$,
      问$\lambda$为何值时,此方程组有唯一解、无解或无穷多个解?并在无穷多解时求出其通解。
    \end{li}
  
\end{frame}



\begin{frame}\ft{\subsecname}
  
    \begin{li}[2008-2009第一学期]
      设线性方程组$\left(
      \begin{array}{ccc}
        1+\lambda&1&1\\
        1&1+\lambda&1\\
        1&1&\lambda
      \end{array}
      \right)\left(
      \begin{array}{c}
        x_1\\x_2\\x_3
      \end{array}
      \right)=\left(
      \begin{array}{c}
        1\\\lambda\\\lambda^2
      \end{array}
      \right)$,
      问$\lambda$为何值时,此方程组有唯一解、无解或无穷多个解?并在无穷多解时求出其通解。
    \end{li}
\end{frame}


\begin{frame}\ft{\subsecname}
    \begin{li}[2009-2010第一学期]
      设线性方程组$\left(
      \begin{array}{ccc}
        2-\lambda&2&-2\\
        2&5-\lambda&-4\\
        -2&-4&5-\lambda
      \end{array}
      \right)\left(
      \begin{array}{c}
        x_1\\x_2\\x_3
      \end{array}
      \right)=\left(
      \begin{array}{c}
        1\\2 \\-1-\lambda
      \end{array}
      \right)$,
      问$\lambda$为何值时,此方程组有唯一解、无解或无穷多个解?并在无穷多解时求出其通解。
    \end{li}
  
\end{frame}






\begin{frame}\ft{\subsecname}
  
    \begin{li}[2009-2010第一学期]
      设线性方程组$\left(
      \begin{array}{ccc}
        1&a&1\\
        1&2a&1\\
        1&1&b
      \end{array}
      \right)\left(
      \begin{array}{c}
        x_1\\x_2\\x_3
      \end{array}
      \right)=\left(
      \begin{array}{c}
        3\\4 \\4
      \end{array}
      \right)$,
      问$a,b$为何值时,此方程组有唯一解、无解或无穷多个解?并在无穷多解时求出其通解。
    \end{li}

    \begin{li}[2011-2012第一学期]
    已知线性方程组$\MA\vx=\vb$存在两个不同的解,其中$$\MA=\left(
    \begin{array}{ccc}
        2-\lambda&2&-2\\
        2&5-\lambda&-4\\
        -2&-4&5-\lambda
      \end{array}
    \right),\vb=\left(
      \begin{array}{c}
        1\\2\\-1-\lambda
      \end{array}
      \right).$$
      就该方程组无解、有唯一解、有无穷多解诸情形,对$\lambda$进行讨论,并在无穷多解时求其通解。
    \end{li}

  
\end{frame}

\begin{frame}\ft{\subsecname}
  
    \begin{li}[2011-2012第二学期]
    已知线性方程组$\MA\vx=\vb$,其中$\MA=\left(
    \begin{array}{rrrr}
        1&1&1&3\\
        2&1&3&5\\
        3&2&a&7\\
        1&-1&3&-1
      \end{array}
    \right),\vb=\left(
      \begin{array}{c}
        0\\1\\1\\b
      \end{array}
      \right).$
      就该方程组无解、有唯一解、有无穷多解诸情形,对$a,b$进行讨论,并在无穷多解时求其通解。
    \end{li}

    \begin{li}[2012-2013第二学期]
    已知线性方程组$\MA\vx=\vb$,其中$\MA=\left(
    \begin{array}{ccc}
        2&4&\lambda-5\\
        2&5-\lambda&-4\\
        2-\lambda&2&-2
      \end{array}
    \right),\vb=\left(
      \begin{array}{c}
        \lambda+1\\2\\1
      \end{array}
      \right).$
     就该方程组无解、有唯一解、有无穷多解诸情形,对$\lambda$进行讨论,并在无穷多解时求其通解。         \end{li}


\end{frame}


\begin{frame}\ft{\subsecname}
  
    \begin{li}[2013-2014第一学期]
    已知线性方程组$\MA\vx=\vb$,其中$$\MA=\left(
    \begin{array}{ccc}
        1&a&a^2\\
        1&a&ab\\
        b&a^2&a^2b
      \end{array}
    \right),\vb=\left(
      \begin{array}{c}
        1\\a\\a^2b
      \end{array}
      \right).$$
     就该方程组无解、有唯一解、有无穷多解诸情形,对$a,b$进行讨论,并在无穷多解时求其通解。    \end{li}

  
\end{frame} 



%% \begin{frame}\ft{\subsecname}
%%   
%%     \begin{li}[2006-2007第二学期]
%%       已知$\MA=\left(
%%       \begin{array}{rrr}
%%         1&1&2\\
%%         -1&1&0\\
%%         1&0&1
%%       \end{array}
%%       \right), MB=\left(
%%       \begin{array}{rrr}
%%         1&2&0\\
%%         -1&0&-2\\
%%         a&b&c
%%       \end{array}
%%       \right)$
%%       \begin{itemize}
%%       \item[1.] 问$a,b,c$为何值时,$\rank(\MA,MB)=\rank(\MA)$?
%%       \item[2.] 求矩阵方程$\MA\X=MB$的全部解?
%%       \end{itemize}
%%     \end{li}
%%   
%% \end{frame}


%% \begin{frame}\ft{\subsecname}
%%   
%%     \begin{li}[2006-2007第二学期]
%%       设线性方程组$\left\{
%%       \begin{array}{rcl}
%%         x_1+2x_2-2x_3&=&0\\
%%         2x_1-x_2+\lambda x_3&=&0\\
%%         3x_1+x_2-x_3&=&0
%%       \end{array}
%%       \right.$的系数矩阵为$\MA$,若$3$阶非零矩阵$MB$满足$\MAMB=\M0$。
%%       求$|\MA|,|MB|,\lambda$。
%%     \end{li}
%%   
%% \end{frame}


%% \begin{frame}\ft{\subsecname}
%%   
%%     \begin{li}[2006-2007第二学期]
%%       设$\MA$为$m\times n$矩阵,$\vx$为$n$维实向量,证明$\MA^T\MA\vx=\M0$与$\MA\vx=\M0$同解。
%%     \end{li}

%%   
%% \end{frame}




