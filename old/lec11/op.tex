% -*- coding: utf-8 -*-
% !TEX program = xelatex

\documentclass[10pt,notheorems]{beamer}

\usetheme[style=beta]{epyt} % alpha, beta, delta, gamma, zeta

\usepackage[UTF8,noindent]{ctex}
\usepackage{extarrows}
%\usepackage{courier}
\usepackage{animate}
\usepackage{dcolumn}
\usepackage{pgf}
\usepackage{tikz}
\usetikzlibrary{calc}
\usetikzlibrary{arrows,snakes,backgrounds,shapes,patterns}
\usetikzlibrary{matrix,fit,positioning,decorations.pathmorphing}
\usepackage{listings}
\lstset{
        frame=no,
        keywordstyle=\color{acolor1},  
        basicstyle=\ttfamily\small,
        commentstyle=\color{acolor5},
        breakindent=0pt,
        rulesepcolor=\color{red!20!green!20!blue!20},
        rulecolor=\color{black},
        tabsize=4,
        numbersep=5pt,
        numberstyle=\footnotesize,
        breaklines=true,
        %% backgroundcolor=\color{red!10},
        showspaces=false,
        showtabs=false,
        showstringspaces=false,
        extendedchars=false,
        escapeinside=``
}



\newcommand{\mylead}[1]{\textcolor{acolor1}{#1}}
\newcommand{\mybold}[1]{\textcolor{acolor2}{#1}}
\newcommand{\mywarn}[1]{\textcolor{acolor3}{#1}}

%%%% \renewcommand *****
%\renewcommand{\lstlistingname}{}
\newcommand{\tf}{\ttfamily}
%\newcommand{\ttt}{\texttt}
%\newcommand{\blue}{\textcolor{blue}}
%\newcommand{\red}{\textcolor{red}}
%\newcommand{\purple}{\textcolor{purple}}
\newcommand{\ft}{\frametitle}
\newcommand{\fst}{\framesubtitle}
\newcommand{\bs}{\boldsymbol}
\newcommand{\ds}{\displaystyle}
\newcommand{\vd}{\vdots}
\newcommand{\cd}{\cdots}
\newcommand{\dd}{\ddots}
\newcommand{\id}{\iddots}
\newcommand{\XX}{\mathbf{X}}
\newcommand{\PP}{\mathbf{P}}
\newcommand{\QQ}{\mathbf{Q}}
\newcommand{\xx}{\mathbf{x}}
\newcommand{\yy}{\mathbf{y}}
\newcommand{\bb}{\mathbf{b}}
\newcommand{\abd}{\boldsymbol{a}}

\renewcommand{\proofname}{证明}




%\newtheorem{theorem}{定理}
%\newtheorem{definition}[theorem]{定义}
%\newtheorem{example}[theorem]{例子}
%\newtheorem{dingli}[theorem]{定理}
%\newtheorem{li}[theorem]{例}
%
%\newtheorem*{theorem*}{定理}
%\newtheorem*{definition*}{定义}
%\newtheorem*{example*}{例子}
%\newtheorem*{dingli*}{定理}
%\newtheorem*{li*}{例}

\renewcommand{\proofname}{证明}
\newtheorem*{jie}{解}
\newtheorem*{zhu}{注}
\newtheorem*{dingli}{定理} 
\newtheorem*{dingyi}{定义} 
\newtheorem*{xingzhi}{性质} 
\newtheorem*{tuilun}{推论} 
\newtheorem{li}{例} 
\newtheorem*{jielun}{结论} 
\newtheorem*{zhengming}{证明}
\newtheorem*{wenti}{问题}
\newtheorem*{jieshi}{解释}
\newtheorem{biancheng}{编程}

\renewcommand{\proofname}{证明}

\begin{document}

\title{第11次C上机}
\subtitle{字符串}
\author{张晓平}
\institute{武汉大学数学与统计学院}


\begin{frame}[plain]\transboxout
\titlepage
\end{frame}

% \begin{frame}\transboxin
% \begin{center}
% \tableofcontents[]%hideallsubsections]
% \end{center}
% \end{frame}

% \AtBeginSection[]{
% \begin{frame}[allowframebreaks]
% \tableofcontents[currentsection,sectionstyle=show/hide]
% \end{frame}
% }
%\AtBeginSubsection[]{
%\begin{frame}[allowframebreaks]
%\tableofcontents[currentsection,currentsubsection,subsectionstyle=show/shaded/hide]
%\end{frame}
%}


\begin{frame}[fragile]
\begin{li} 
设计并测试一个函数{\tf my\_strchr()},其功能是搜索由函数的第一个参数指定的字符串,在其中查找由函数的第二个参数指定的字符的第一个出现的位置。如果找到,返回指向这个字符的指针;如果没有找到,返回{\tf NULL}。在一个使用循环语句为这个函数提供输入的完整程序中进行测试。
\end{li}
\end{frame}


\begin{frame}[fragile,allowframebreaks]
\lstinputlisting
[language=c,numbers=left,frame=single]
{Code_OP/ex11_01.c}
\end{frame}


\begin{frame}[fragile]
\begin{li} 
编写一个函数{\tf is\_within()},它接受两个参数,一个是字符,另一个是字符串指针。其功能是如果字符在字符串中,就返回一个非零值(真);否则返回0值(假)。在一个使用循环语句为这个函数提供输入的完整程序中进行测试。
\end{li}
\end{frame}


\begin{frame}[fragile,allowframebreaks]
\lstinputlisting
[language=c,numbers=left,frame=single]
{Code_OP/ex11_02.c}
\end{frame}

\begin{frame}[fragile]
\begin{li} 
函数{\tf strncpy(s1, s2, n)}从{\tf s2}复制{\tf n}个字符给{\tf s1},并在必要时截断{\tf s2}或为其填充额外的空字符。如果{\tf s2}的长度大于或等于{\tf n},目标字符串就没有标识结束的空字符。函数返回{\tf s1}。自己编写这个函数{\tf my\_strncpy()},并编制一个完整程序测试它。
\end{li}
\end{frame}


\begin{frame}[fragile,allowframebreaks]
\lstinputlisting
[language=c,numbers=left,frame=single]
{Code_OP/ex11_03.c}
\end{frame}




\begin{frame}[fragile]
\begin{li} 
编写一个函数{\tf string\_in()},它接受两个字符串指针参数。如果第二个字符串被包含在第一个字符串中,函数就返回被包含的字符串开始的地址。如,{\tf string\_in("hats'',"at")}返回{\tf hats}中{\tf a}的地址,否则返回{\tf NULL}。 最后编制一个完整程序测试它。
\end{li}
\end{frame}


\begin{frame}[fragile,allowframebreaks]
\lstinputlisting
[language=c,numbers=left,frame=single]
{Code_OP/ex11_04.c}
\end{frame}



\end{document}
