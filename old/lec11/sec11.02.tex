\section{字符串输入}

\begin{frame}[fragile]\ft{\secname} 
把一个字符串读到程序中,必须首先预留存储字符串的空间,然后通过输入函数来获取该字符串。
\end{frame}

\begin{frame}[fragile]\ft{\secname:创建存储空间} 
在读入字符串之前,必须分配足够大的存储区来存放读入的字符串。\textcolor{acolor1}{不要指望计算机读的时候会计算字符串长度,然后为字符串分配空间。} 例如,如果你这么做
\begin{lstlisting}[language=c,showstringspaces=true]
char * name;
scanf("%s", name);
\end{lstlisting}
虽然编译会通过,但后果可能会非常严重,因{\tf name}可能指向任何地方。
\end{frame}

\begin{frame}[fragile]\ft{\secname:创建存储空间} 
创建存储空间,有两种方式:
\begin{itemize}
\item 
最简单的方法就是在声明中明确指出数组大小:
\begin{lstlisting}[language=c,showstringspaces=true]
char name[81];
\end{lstlisting}
\vspace{0.05in}
\item 
另一种方式就是使用C库中分配存储空间的函数,如malloc函数。
\end{itemize}
\end{frame}

\begin{frame}[fragile]\ft{\secname:创建存储空间} 
为字符串预留空间后,就可以读取字符串了。C库提供了三种读取字符串的函数:
\begin{itemize}
\item {\tf scanf()}
\item {\tf gets()}
\item {\tf fgets()}
\end{itemize}
\end{frame}

\begin{frame}[fragile]\ft{\secname:{\tf gets()}} 
函数gets(代表get string)从键盘获取一个字符串。\vspace{0.2in}

\textcolor{acolor1}{它读取换行符之前(不包括换行符)的所有字符,在这些字符后添加一个空字符,然后把这个字符串交给调用它的程序。}

\vspace{0.2in}
它将读取换行符并将其丢弃,这样下一次读取就会在新的一行开始。
\end{frame}

\begin{frame}[fragile]\ft{\secname:{\tf gets()}} 
\lstinputlisting
[language=c,numbers=left,frame=single]
{Code/name1.c}
\end{frame}

\begin{frame}[fragile]\ft{\secname:{\tf gets()}} 
\begin{lstlisting}[]
Hi, what's your name?
warning: this program uses gets(), which is unsafe.
Xiaoping Zhang
Nice name, Xiaoping Zhang.
\end{lstlisting}

\rule{\textwidth}{0.3mm} \pause \vspace{.1in}
由于{\tf gets()}不检查目标数组是否能够容纳输入,故编译器会给出警告,提醒用户使用{\tf gets()}会不安全。
\end{frame}

\begin{frame}[fragile]\ft{\secname:{\tf gets()}} 
\lstinputlisting
[language=c,showstringspaces=true,numbers=left,frame=single]
{Code/name2.c}
\end{frame}

\begin{frame}[fragile]\ft{\secname:{\tf gets()}} 
\begin{lstlisting}[]
Hi, what's your name?
warning: this program uses gets(), which is unsafe.
Xiaoping Zhang
Xiaoping Zhang? Ah! Xiaoping Zhang!
\end{lstlisting}
\end{frame}


\begin{frame}[fragile]\ft{\secname:{\tf fgets()}} 
由于{\tf gets()}不检查目标数组是否能够容纳输入,多出来的字符简单溢出到相邻的内存区。{\tf fgets()}改进了该不足,它让用户指定最大读入字符数。
\end{frame}

\begin{frame}[fragile]\ft{\secname:{\tf fgets()}} 
{\tf gets()}与{\tf fgets()}的三个不同:
\begin{enumerate}
\item {\tf fgets()}用第二个参数来说明最大读入字符数。
\item[] 若该参数值为{\tf n},则{\tf fgets()}会读取最多{\tf n-1}个字符或读完第一个换行符为止。\\[0.1in]
\item 若{\tf fgets()}读取到换行符,就会把它存到字符串中,而{\tf gets()}会丢弃它。\\[0.1in]
\item {\tf fgets()}用第三个参数来说明读哪一个文件。从键盘输入时,可以使用{\tf stdin()}作为该参数。
\end{enumerate}
\end{frame}

\begin{frame}[fragile]\ft{\secname:{\tf fgets()}} 
\lstinputlisting
[language=c,showstringspaces=true,numbers=left,frame=single]
{Code/name3.c}
\end{frame}

\begin{frame}[fragile]\ft{\secname:{\tf gets()}} 
\begin{lstlisting}[]
Hi, what's your name?
Xiaoping Zhang
Xiaoping Zhang
? Ah! Xiaoping Zhang
!
\end{lstlisting}
\rule{\textwidth}{0.3mm} \pause \vspace{.1in}

该显示表明了fgets的不足,原因在于fgets把换行符存储到字符串中,导致每次显示字符串时就会显示换行符。 
\end{frame}

\begin{frame}[fragile]\ft{\secname:{\tf scanf()}} 

可用带{\tf \%s}的{\tf scanf()}来读取一个字符串。 \vspace{.1in}

{\tf scanf()}与{\tf gets()}的主要差别在于字符串何时结束。\vspace{.1in}

{\tf scanf()}更基于获取单词而不是获取字符串,而{\tf gets()}会读取所有字符,直到遇到第一个换行符为止。
\end{frame}

\begin{frame}[fragile]\ft{\secname:{\tf scanf()}} 

{\tf scanf()}使用两种方法决定输入结束,但无论哪种方法,字符串都以遇到的第一个非空白字符开始。

\begin{enumerate}
\item 若使用{\tf \%s}格式,字符串读到(但不包括)下一个空白字符。\\[0.1in]
\item 若指定字段宽度,如{\tf \%10s},则{\tf scanf()}会读入{\tf 10}个字符或直到遇到第一个空白字符。
\end{enumerate}
\end{frame}

\begin{frame}[fragile]\ft{\secname:{\tf scanf()}}
  \begin{footnotesize}
\begin{table}
\centering
\caption{以\_代表空格}
\begin{tabular}{c|l|l|l}\hline
输入语句&原始输入队列&name中的内容&剩余队列\\\hline
{\tf scanf("\%s", name);}  & \tf Fleebert\_Hup & \tf Fleebert & \tf \_Hup \\\hline
{\tf scanf("\%5s", name);} & \tf Fleebert\_Hup & \tf Fleeb    & \tf ert\_Hup\\\hline
{\tf scanf("\%5s", name);} & \tf Ann\_Ular & \tf Ann & \tf \_Ular\\\hline
\end{tabular}
\end{table}
  \end{footnotesize}
\end{frame}

\begin{frame}[fragile]\ft{\secname:{\tf scanf()}} 
\lstinputlisting
[language=c,showstringspaces=true,numbers=left,frame=single]
{Code/scan_str.c}
\end{frame}

\begin{frame}[fragile]\ft{\secname:{\tf scanf()}} 
\begin{lstlisting}[backgroundcolor=\color{blue!20}]
Please enter 2 names.
Jesse Jukes
I read the 2 names Jesse and Jukes.
\end{lstlisting}

\begin{lstlisting}[backgroundcolor=\color{blue!20}]
Please enter 2 names.
Liza Applebottham
I read the 2 names Liza and Applebotth.
\end{lstlisting}

\begin{lstlisting}[backgroundcolor=\color{blue!20}]
Please enter 2 names.
Portensia Callowit
I read the 2 names Porte and nsia.
\end{lstlisting}

\end{frame}