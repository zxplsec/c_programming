\section{字符串输出}

\begin{frame}[fragile]\ft{\secname:{\tf puts()}} 
C有三个用于输出字符串的标准库函数:
\begin{itemize}
\item {\tf puts()}
\item {\tf fputs()}
\item {\tf printf()}
\end{itemize}
\end{frame}



\begin{frame}[fragile,allowframebreaks]\ft{\secname:{\tf puts()}} 
\lstinputlisting
[language=c,showstringspaces=true,numbers=left,frame=single]
{Code/put_out.c}
\end{frame}


\begin{frame}[fragile]\ft{\secname:{\tf puts()}} 
\begin{lstlisting}[basicstyle=\ttfamily]
I'm an argument to puts().
I am a #defined string.
An array was initialized to me.
A pointer was initialized to me.
ray was initialized to me.
inter was initialized to me.
\end{lstlisting}
\end{frame}


\begin{frame}[fragile]\ft{\secname:{\tf puts()}} 
\begin{itemize}
\item
使用{\tf puts()},只需给出字符串参数的地址。 \\[0.1in]
\item 
与{\tf printf()}不同,{\tf puts()}显示字符串时会自动在其后添加一个换行符。\\[0.1in]
\item
{\tf puts()}遇到空字符时便会停止输出,故应确保有空字符存在。
\end{itemize}
\end{frame}

\begin{frame}[fragile]\ft{\secname:{\tf puts()}} 
\begin{lstlisting}[language=c]
char str1[80] = "An array was initialized to me.";
...
puts(&str1[5]);
\end{lstlisting}
输出结果为
\begin{lstlisting}[basicstyle=\ttfamily]
ray was initialized to me.
\end{lstlisting}
\pause \rule{\textwidth}{0.3mm} \vspace{0.1mm}

因表达式{\tf \&str1[5]}是数组{\tf str1}第{\tf 6}个元素的地址,该元素为字符{\tf 'r'},也正是{\tf puts()}输出字符串的起点。
\end{frame}

\begin{frame}[fragile]\ft{\secname:{\tf puts()}} 
\begin{lstlisting}[basicstyle=\ttfamily]
const char * str2 = "A pointer was initialized to me.";
...
puts(str2+4);
\end{lstlisting}
输出结果为
\begin{lstlisting}[basicstyle=\ttfamily]
inter was initialized to me.
\end{lstlisting} 
\pause \rule{\textwidth}{0.3mm} \vspace{0.1mm}

{\tf str2+4}指向包含第一个{\tf 'i'}字符的内存单元,{\tf puts()}输出字符串从这里开始。
\end{frame}

\begin{frame}[fragile]\ft{\secname:{\tf puts()}} 
\lstinputlisting
[language=c,numbers=left,frame=single]
{Code/nono.c}
\pause 
不要效仿该程序!
\end{frame}

\begin{frame}[fragile]\ft{\secname:{\tf puts()}}
输出结果为
\begin{lstlisting}[basicstyle=\ttfamily]
WOW!Side A
\end{lstlisting} 
\pause \rule{\textwidth}{0.3mm} \vspace{0.1mm}

{\tf dont}缺少空字符,不是字符串,于是{\tf puts()}就不知道该在哪里停止。它会一直输出内存中{\tf dont}后面的字符,直到发现一个空字符。\vspace{0.1mm}

从运行结果来看,在内存中{\tf side\_a}数组存储在{\tf dont}数组之后。
\end{frame}

\begin{frame}[fragile]\ft{\secname:{\tf fputs()}} 
{\tf fputs()}是{\tf puts()}的面向文件版本。两者的主要区别是:\vspace{0.1mm}

\begin{itemize}
\item
{\tf fputs()}需要第二个参数来说明要写的文件。可用{\tf stdout}作为参数来进行输出显示,{\tf stdout}在{\tf stdio.h}中定义。 \\[0.1in]
\item 
与{\tf puts()}不同,{\tf fputs()}并不为输出自动添加换行符。
\end{itemize}
\end{frame}

\begin{frame}[fragile]\ft{\secname:{\tf fputs()}}
gets丢掉输入的换行符,而puts为输出添加换行符。\vspace{0.1in}

假定写一个循环,读取一行并把它回显在下一行,可这么写:
\begin{lstlisting}[language=c]
char line[81];
while (gets(line))
  puts(line);
\end{lstlisting} 
\end{frame}

\begin{frame}[fragile]\ft{\secname:{\tf fputs()}}
{\tf fgets()}存储输入的换行符,而{\tf fputs()}不会为输出添加换行符。\vspace{0.1in}

故以上程序也可写成
\begin{lstlisting}[language=c]
char line[81];
while (fgets(line, 81, stdin))
  fputs(line, stdout);
\end{lstlisting} \vspace{0.1in}

以上两段代码中,{\tf line}数组中的字符都被显示在单独的一行。
\end{frame}

\begin{frame}[fragile]\ft{\secname:{\tf fputs()}}
注意:{\tf puts()}是为和{\tf gets()}一起使用而设计的,而{\tf fputs()}是为和{\tf fgets()}一起使用而设计的。
\end{frame}

\begin{frame}[fragile]\ft{\secname:{\tf printf()}} 
输出字符串时,{\tf printf()}不如{\tf puts()}使用方便,但其可以格式化多种数据类型,因而更通用。

\begin{lstlisting}[basicstyle=\ttfamily]
printf("%s\n", string);
\end{lstlisting}
等效于
\begin{lstlisting}[basicstyle=\ttfamily]
puts(string);
\end{lstlisting}
但前者更简洁。
\end{frame}

\begin{frame}[fragile]\ft{\secname:{\tf printf()}} 
不过,想在一行上输出多个字符串时,{\tf printf()}更为简单。如
\begin{lstlisting}[language=c]
printf("Well, %s, %s\n", name, MSG);
\end{lstlisting}
\end{frame}

