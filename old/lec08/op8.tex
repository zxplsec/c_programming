\subsection{上机操作}

\begin{frame}[fragile]\ft{\subsecname}
\begin{block}{例1}
编写一个程序,把输入作为字符流读取,直至遇到EOF。令其报告输入中的大写字母个数和小写字母个数。
\end{block}
\end{frame}

\begin{frame}[fragile]\ft{\subsecname}
\begin{block}{例2}
改写猜数程序:猜1-100中的某个数字z,按二分法进行。
\begin{itemize}
\item 
假设你最初猜50,让其询问z是大于、小于还是等于猜测值。
\item
若小于50,则令下一次猜测值为50和100的平均值75。
\item
若大于75,则令下一次猜测值为50和75的平均值62。
\end{itemize}
\end{block}
\end{frame}

\begin{frame}[fragile]\ft{\subsecname}
\begin{block}{例3}
编写一个程序,显示一个菜单,提供加法、减法、乘法或除法的选项。获取选择后,该程序请求两个数,然后执行选择的操作。
\begin{itemize}
\item 
该程序应该只接受所提供的菜单选项,应使用float类型的数,并且如果用户未能输入数字应允许其重新输入。
\item
在除法的情况下,如果用户输入0作为第二个数,该程序应该提示用户输入一个新的值。
\end{itemize}
\end{block}
\end{frame}

\begin{frame}[fragile]\ft{\subsecname}
\begin{lstlisting}
Enter the operation of your choice:
a. add         b. substract
c. multiply    d. divide
q. quit
a
Enter first number: 22.4
Enter second number: one
one is not a number.
Please enter a number,
such as 2.5, -1.78E8, or 3: 1
22.40 + 1.00 = 23.40.
\end{lstlisting}
\end{frame}

\begin{frame}[fragile]\ft{\subsecname}
\begin{lstlisting}[backgroundcolor=\color{blue!20}]
Enter the operation of your choice:
a. add         b. substract
c. multiply    d. divide
q. quit
d
Enter first number: 1
Enter second number: 0
Enter a number other than 0: 0.2
1.00 / 0.20 = 5.00.
Enter the operation of your choice:
a. add         b. substract
c. multiply    d. divide
q. quit
q
Bye.
\end{lstlisting}
\end{frame}

