\documentclass[10pt,a4paper%,twoside,openright,titlepage,fleqn,%
%headinclude,footinclude,BCOR5mm,%
%numbers=noenddot,cleardoublepage=empty,%
tablecaptionabove]{article}

\usepackage{geometry}
\geometry{left=2.5cm,right=2.5cm,top=2.5cm,bottom=2.5cm}

\usepackage{amsmath,amssymb,amsthm}

%% -----------------设置数学公式字体-------------------------
%% Font style 1
%% \newcommand\ibinom[2]{\genfrac\lbrace\rbrace{0pt}{}{#1}{#2}}
%% \usepackage{bm}

%% Font style 2
%% \newcommand\ibinom[2]{\genfrac\lbrace\rbrace{0pt}{}{#1}{#2}} 
%% \usepackage[boldsans]{ccfonts} 
%% \usepackage{bm} 

%% Font style 3
\newcommand\ibinom[2]{\genfrac\lbrace\rbrace{0pt}{}{#1}{#2}}
\usepackage[euler-digits]{eulervm}
\usepackage{bm}

%% Font style 4
%% \usepackage{fourier}
%% \newcommand\ibinom[2]{\genfrac\lbrace\rbrace{0pt}{}{#1}{#2}}
%% \usepackage{bm}

%% Font style 5
%% \newcommand\ibinom[2]{\genfrac\lbrace\rbrace{0pt}{}{#1}{#2}}
%% \usepackage{mathptmx}
%% \usepackage{bm} 


%% %% Font style 6
%% \newcommand\ibinom[2]{\genfrac\lbrace\rbrace{0pt}{}{#1}{#2}}
%% \usepackage{txfonts}
%% \usepackage{bm}
%% -----------------------------------------------------------

\usepackage{titlesec} %设置标题
\usepackage{titletoc}

\usepackage{extarrows}
\usepackage{verbatim,color,xcolor}
\usepackage{pgf}
\usepackage{tikz}
\usetikzlibrary{calc}
\usetikzlibrary{arrows,snakes,backgrounds,shapes,patterns}
\usetikzlibrary{matrix,fit,positioning,decorations.pathmorphing}
%% \usepackage{classicthesis}
\usepackage{CJK}
\usepackage{mathdots}

\usepackage{listings}
\lstset{
  keywordstyle=\color{blue!70},
  frame=single,
  basicstyle=\ttfamily\small,
  commentstyle=\small\color{red},
  breakindent=0pt,
  rulesepcolor=\color{red!20!green!20!blue!20},
  rulecolor=\color{black},
  tabsize=4,
  numbersep=5pt,
  breaklines=true,
  %% backgroundcolor=\color{red!10},
  showspaces=false,
  showtabs=false,
  extendedchars=false,
  escapeinside=``,
  frame=no,
}


\newcommand{\blue}{\textcolor{blue}}
\newcommand{\red}{\textcolor{red}}
\newcommand{\purple}{\textcolor{electricpurple}}
\newcommand{\ds}{\displaystyle}
\newcommand{\cd}{\cdots}
\newcommand{\dd}{\ddots}
\newcommand{\vd}{\vdots}
\newcommand{\id}{\iddots}

\newcommand{\R}{\mathbb R}
\def\nn{{\boldsymbol{n}}}
\def\xx{{\boldsymbol{x}}}
\def\F{{\boldsymbol{F}}}
\def\div{{\mathrm{div}}}
\def\tf{\ttfamily}


\begin{document}

\begin{CJK}{UTF8}{gkai}
 

\newtheorem{li}{例}
\newtheorem{jielun}{结论}
\newtheorem{dingli}{定理}
\newtheorem{mingti}{{命题}} 
\newtheorem{yinli}{{引理}} 
\newtheorem{tuilun}{{推论}}
\newtheorem{dingyi}{{定义}} 
\newtheorem{example}{{例}}
\newtheorem*{example*}{{例}}
\newtheorem*{jie}{{解}}
\newtheorem*{zhengming}{{证明}}
\newtheorem{zhu}{{注}}
\newtheorem*{zhu*}{{注}}
\newtheorem{xingzhi}{{性质}}
\newtheorem{wenti}{{问题}}
\newtheorem{rem}{{Remark}}
\newtheorem{lem}{{Lemma}}
\pagenumbering{roman}
\pagestyle{plain}

\pagenumbering{arabic}

\title{关系运算符与逻辑运算符}
%\author{张晓平}
%\date{}                                           % Activate to display a given date or no date
\maketitle

\section{关系运算符}
关系运算符用于比较两个值。
\begin{enumerate}
\item
  运算符 {\tf ==}  检查两个给定的操作数是否相等。若相等,返回{\tf true};否则返回{\tf false}。如 {\tf 5 == 5} 返回{\tf true}。
\item
  运算符 {\tf !=}  检查两个给定的操作数是否相等。若不相等,返回{\tf true};否则返回{\tf false}。如 {\tf 5 != 5} 返回{\tf false}。
\item
  运算符 {\tf >} 检查第一个操作数是否大于第二个操作数。若成立,返回{\tf true};否则返回{\tf false}。如 {\tf 6 > 5} 返回{\tf true}。
\item
  运算符 {\tf <} 检查第一个操作数是否小于第二个操作数。若成立,返回{\tf true};否则返回{\tf false}。如 {\tf 6 < 5} 返回{\tf false}。
\item
  运算符 {\tf >=} 检查第一个操作数是否大于或等于第二个操作数。若成立,返回{\tf true};否则返回{\tf false}。如 {\tf 5 >= 5} 返回{\tf true}。
\item
  运算符 {\tf <=} 检查第一个操作数是否小于或等于第二个操作数。若成立,返回{\tf true};否则返回{\tf false}。如 {\tf 5 <= 5} 返回{\tf true}。
\end{enumerate}

\begin{lstlisting}[language=c,backgroundcolor=\color{red!10}]
// C program to demonstrate working of relational operators
#include <stdio.h>
 
int main()
{
    int a=10, b=4;
 
    // relational operators
    // greater than example
    if (a > b)
        printf("a is greater than b\n");
    else printf("a is less than or equal to b\n");
 
    // greater than equal to
    if (a >= b)
        printf("a is greater than or equal to b\n");
    else printf("a is lesser than b\n");
 
    // less than example
    if (a < b)
        printf("a is less than b\n");
    else printf("a is greater than or equal to b\n");
 
    // lesser than equal to
    if (a <= b)
        printf("a is lesser than or equal to b\n");
    else printf("a is greater than b\n");
 
    // equal to
    if (a == b)
        printf("a is equal to b\n");
    else printf("a and b are not equal\n");
 
    // not equal to
    if (a != b)
        printf("a is not equal to b\n");
    else printf("a is equal b\n");
 
    return 0;
}  
\end{lstlisting}

\begin{lstlisting}[backgroundcolor=\color{red!10}]
Output:
a is greater than b
a is greater than or equal to b
a is greater than or equal to b
a is greater than b
a and b are not equal
a is not equal to b  
\end{lstlisting}


\section{逻辑运算符}
逻辑运算符用于连接两个及以上条件,或对原条件取补。%%% They are used to combine two or more conditions/constraints or to complement the evaluation of the original condition in consideration.

\begin{enumerate}
\item
  逻辑与: 当两个条件同时满足时,运算符 {\tf \&\&} 返回{\tf true};否则返回 {\tf false}。如,当 {\tf a} 和 {\tf b} 均为 {\tf true} (即非零)时,{\tf a \&\& b} 返回{\tf true}。
\item
  逻辑或:当至少有一个条件满足时,运算符 {\tf ||} 返回{\tf true};否则返回 {\tf false}。如,当 {\tf a} 和 {\tf b} 至少有一个为 {\tf true} (即非零)时,{\tf a || b} 返回{\tf true}。当然,当 {\tf a} 和 {\tf b} 均为 {\tf true} 时, {\tf a || b}返回{\tf true}。
\item
  逻辑非:当条件不满足时,运算符 {\tf !} 返回 {\tf true} ;否则返回 {\tf false} 。如,若 {\tf a} 为 {\tf false} 时,{\tf a} 返回 {\tf true}。
\end{enumerate}
\begin{lstlisting}[language=c,backgroundcolor=\color{red!10}]
// C program to demonstrate working of logical operators
#include <stdio.h>
 
int main()
{
    int a=10, b=4, c = 10, d = 20;
 
    // logical operators
 
    // logical AND example
    if (a>b && c==d)
        printf("a is greater than b AND c is equal to d\n");
    else printf("AND condition not satisfied\n");
 
    // logical AND example
    if (a>b || c==d)
        printf("a is greater than b OR c is equal to d\n");
    else printf("Neither a is greater than b nor c is equal "
                " to d\n");
 
    // logical NOT example
    if (!a)
        printf("a is zero\n");
    else printf("a is not zero");
 
    return 0;
}  
\end{lstlisting}

\begin{lstlisting}[backgroundcolor=\color{red!10}]
AND condition not satisfied
a is greater than b OR c is equal to d
a is not zero  
\end{lstlisting}

\section{逻辑运算符中的短路现象}

对于逻辑与,若第一个操作数为 {\tf false} ,则第二个操作数将不会被计算。如以下程序将不会打印 {\tf Hello World}。
\begin{lstlisting}[language=c,backgroundcolor=\color{red!10}]
#include <stdio.h>
#include <stdbool.h>
int main()
{
    int a=10, b=4;
    bool res = ((a == b) && printf("Hello World"));
    return 0;
}  
\end{lstlisting}
但下面的程序将打印 {\tf Hello World}。
\begin{lstlisting}[language=c,backgroundcolor=\color{red!10}]
#include <stdio.h>
#include <stdbool.h>
int main()
{
    int a=10, b=4;
    bool res = ((a != b) && printf("GeeksQuiz"));
    return 0;
}
\end{lstlisting}
对于逻辑或,若第一个操作数为 {\tf true} ,则第二个操作数不会被计算。如以下程序不会打印 {\tf Hello World}。
\begin{lstlisting}[language=c,backgroundcolor=\color{red!10}]
#include <stdio.h>
#include <stdbool.h>
int main()
{
    int a=10, b=4;
    bool res = ((a != b) || printf("GeeksQuiz"));
    return 0;
}
\end{lstlisting}
但下面的程序将打印 {\tf Hello World}。
\begin{lstlisting}[language=c,backgroundcolor=\color{red!10}]
#include <stdio.h>
#include <stdbool.h>
int main()
{
    int a=10, b=4;
    bool res = ((a == b) || printf("GeeksQuiz"));
    return 0;
}
\end{lstlisting}
\end{CJK}
\end{document}
