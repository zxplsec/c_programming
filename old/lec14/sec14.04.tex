\section{文件的定位}
\begin{frame}[fragile]\ft{\secname}
文件中有一个位置指针,指向当前读写的位置。每当进行一次读写后,该指针自动指向下一个字符的位置。可以用{\tf ftell()}获得当前的位置指针,也可以用{\tf rewind()/fseek()}改变位置指针,使其指向需要读写的位置。
\end{frame}

\begin{frame}[fragile]\ft{\secname:{\tf rewind()}}
函数原型声明:
\begin{lstlisting}[language=c,backgroundcolor=\color{red!20}]
void rewind(FILE * stream);
\end{lstlisting}
使文件{\tf fp}的位置指针指向文件开始。
\end{frame}

\begin{frame}[fragile]\ft{\secname:{\tf rewind()}}
\begin{li}
把一个文件的内容显示在屏幕上,并同时复制到另一个文件。
\end{li}
\end{frame}

\begin{frame}[fragile,allowframebreaks]\ft{\secname:{\tf rewind()}}
  \lstinputlisting
  [language=c,numbers=left,frame=single]
  {code/rewind.c}
\end{frame}

\begin{frame}[fragile]\ft{\secname:{\tf rewind()}}
  假设有文件{\tf file.txt},其中内容为
  \begin{lstlisting}[backgroundcolor=\color{red!20}]
C primer plus
C programming
  \end{lstlisting}
  则执行结果为
  \begin{lstlisting}[backgroundcolor=\color{red!20}]
$ gcc rewind.c -o rewind
$ ./rewind file.txt file1.txt
C primer plus
C programming
  \end{lstlisting}
  
\end{frame}

\begin{frame}[fragile]\ft{\secname:{\tf fseek()}}
对流式文件可以进行顺序读写,也可以进行随机读写,关键在于控制文件的位置指针。
用{\tf fseek()}可以实现改变文件的位置指针。
\begin{itemize}
\item 
函数原型声明
\begin{lstlisting}[language=c,backgroundcolor=\color{red!20}]
int fseek(FILE * stream, long offset, int fromwhere);
\end{lstlisting}
\item
功能:把文件的位置指针从起始点开始,移动指定位移量的字节数。
\item 
若一切正常,{\tf fseek()}的返回值为{\tf 0};若有错,如试图移动超出文件范围,则返回值为{\tf -1}。
\end{itemize}
\end{frame}

\begin{frame}[fragile]\ft{\secname:{\tf fseek()}}
\begin{table}
\centering
\begin{tabular}{|p{3cm}|p{3cm}|p{2cm}|}\hline\hline
起始点&符号常量&值\\\hline
文件开始位置&{\tf SEEK\_SET}&{\tf 0}\\\hline
当前位置&{\tf SEEK\_CUR}&{\tf 1}\\\hline
文件尾&{\tf SEEK\_END}&{\tf 2}\\\hline\hline
\end{tabular}
\end{table}
\end{frame}

\begin{frame}[fragile]\ft{\secname:{\tf fseek()}}
  \begin{lstlisting}[language=c,backgroundcolor=\color{red!20}]
fseek(fp, 0L, SEEK_SET);  //`找到文件的开始处`
fseek(fp, 10L, SEEK_SET); //`找到文件的第10个字节`
fseek(fp, 2L, SEEK_CUR);  //`从文件的当前位置向前移动`
                          //`2个字节`
fseek(fp, 0L, SEEK_END);  //`到达文件结尾处`
fseek(fp, -10L, SEEK_END);//`从文件结尾处退回10个字节`    
  \end{lstlisting}
\end{frame}

\begin{frame}[fragile]\ft{\secname:{\tf ftell()}}
{\tf ftell()}通过距文件开始处的字节数来确定文件的当前位置。在{\tf ANSI C}下,该定义适用于以二进制模式打开的文件,但对以文本模式打开的文件来讲,不一定是这样。
例如:
\begin{lstlisting}[language=c,backgroundcolor=\color{red!20}]
i = ftell(fp);
if(i == -1L) printf("error\n");
\end{lstlisting}
变量{\tf i}存放当前位置,如调用函数时出错(如不存在{\tf fp}文件),则输出{\tf "error"}。
\end{frame}

\begin{frame}[fragile]\ft{\secname:{\tf fseek()}和{\tf ftell()}的应用举例}
  \begin{li}
    读取文件名,反序显示一个文件。
  \end{li}
\end{frame}

\begin{frame}[fragile,allowframebreaks]\ft{\secname:{\tf fseek()}和{\tf ftell()}的应用举例}
  \lstinputlisting
  [language=c,numbers=left,frame=single]
  {code/reverse.c}
\end{frame}


\begin{frame}[fragile]\ft{\secname:{\tf fseek()}和{\tf ftell()}的应用举例}
  \begin{lstlisting}[backgroundcolor=\color{red!20}]
// file4    
Hello World!
I love WHU!    
  \end{lstlisting}
  \begin{lstlisting}[backgroundcolor=\color{red!20}]
Enter the name of the file to be processed: 
file4
!UHW evol I
!dlroW olleH    
  \end{lstlisting}
\end{frame}


\begin{frame}[fragile]\ft{\secname:{\tf fseek()}和{\tf ftell()}的应用举例}
  \begin{li}
    创建一个{\tf double}型数值的文件,然后允许你访问这些内容。
  \end{li}
\end{frame}

\begin{frame}[fragile,allowframebreaks]\ft{\secname:{\tf fseek()}和{\tf ftell()}的应用举例}
  \lstinputlisting
  [language=c,numbers=left,frame=single]
  {code/randbin.c}
\end{frame}


\begin{frame}[fragile,allowframebreaks]\ft{\secname:{\tf fseek()}和{\tf ftell()}的应用举例}
  \begin{lstlisting}[backgroundcolor=\color{red!20}]
Enter an index in the range 0-999
1
The value there is 100.500000.
Next index (out of range to quit):
4
The value there is 400.200000.
Next index (out of range to quit):
5
The value there is 500.166667.
Next index (out of range to quit):
100
The value there is 10000.009901.
Next index (out of range to quit):
-1
Bye!    
  \end{lstlisting}
\end{frame}
