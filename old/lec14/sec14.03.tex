\section{文件的读写}
\begin{frame}[fragile]\ft{\secname}
  当文件打开后,就可以对它进行读写了。常用的读写函数有:\vspace{.05in}

  \begin{itemize}
  \item {\tf fputc()}和{\tf fgetc()}    \\[0.1in]
  \item {\tf fprintf()}和{\tf fscanf()} \\[0.1in]
  \item {\tf fread()}和{\tf fwrite()}   
  \end{itemize}
\end{frame}


\begin{frame}[fragile]\ft{\secname:{\tf fputc()}}
  对于{\tf fputc()},
  \begin{itemize}
    \item 原型声明
    \begin{lstlisting}[language=c,backgroundcolor=\color{red!20}]
int fputc(int c, FILE * stream);
    \end{lstlisting}
  \item 函数说明:{\tf fputc()}将参数{\tf c}转为{\tf unsigned char}后写入参数{\tf stream}指定的文件中。
  \item 返回值:成功时返回字符{\tf c}的{\tf ASCII}码,失败时返回{\tf EOF}。
  \end{itemize}
\end{frame}

\begin{frame}[fragile]\ft{\secname:{\tf fgetc()}}
  对于{\tf fgetc()},
  \begin{itemize}
    \item 原型声明
    \begin{lstlisting}[language=c,backgroundcolor=\color{red!20}]
int fgetc(FILE * stream);
    \end{lstlisting}
    \item 函数说明:{\tf fgetc()}从参数{\tf stream}所指的文件中读取一个字符。 若读到文件尾而无数据时便返回{\tf EOF}.
  \end{itemize}
  \begin{itemize}
  \item[$\bullet$]  对于文本文件,遇文件尾时返回文件结束标志{\tf EOF}。\\[0.1in]
  \item[$\bullet$]  对于二进制文件,用{\tf feof(fp)} 判别是否遇文件尾,{\tf feof(fp)==1}说明遇文件尾。
  \end{itemize}
\end{frame}

\begin{frame}[fragile]\ft{\secname:{\tf fgetc()}}
  \begin{li}
    从文本文件中顺序读入文件内容,并在屏幕上显示出来。
  \end{li}
\end{frame}

\begin{frame}[fragile,allowframebreaks]\ft{\secname:{\tf fgetc()}}
  \lstinputlisting
  [language=c,numbers=left,frame=single]
  {code/file2screen.c}

\end{frame}

\begin{frame}[fragile]\ft{\secname:{\tf fgetc()}}
  从二进制文件中顺序读入文件内容,可以用:
  \begin{lstlisting}[language=c,backgroundcolor=\color{red!20}]
    while(!feof(fp))
    {
      ch = fgetc(fp);
      ...
    }  
  \end{lstlisting}
  \textcolor{acolor1}{这种方法也适用于文本文件。}
\end{frame}

\begin{frame}[fragile]\ft{\secname:{\tf fputc()}和{\tf fgetc()}的应用举例}
  \begin{li}
    从键盘输入一些字符,逐个把它们送入磁盘文件,直到从键盘输入\#为止。    
  \end{li}
\end{frame}

\begin{frame}[fragile,allowframebreaks]\ft{\secname:{\tf fputc()}和{\tf fgetc()}的应用举例}
  \lstinputlisting
  [language=c,numbers=left,frame=single]
  {code/screen2file.c}
\end{frame}


\begin{frame}[fragile]\ft{\secname:{\tf fputc()}和{\tf fgetc()}的应用举例}
  \begin{li}
    将一个磁盘文件的内容复制到另一个磁盘文件。
  \end{li}
\end{frame}

\begin{frame}[fragile,allowframebreaks]\ft{\secname:{\tf fputc()}和{\tf fgetc()}的应用举例}
  \lstinputlisting
  [language=c,frame=single,numbers=left]
  {code/copy.c}
\end{frame}




\begin{frame}[fragile]\ft{\secname:{\tf fscanf()}和{\tf fprintf()}}
  {\tf fprintf()}、{\tf fscanf()}与{\tf printf()、scanf()}的作用相仿,都是格式化读写函数。{\tf fprintf()}和{\tf fscanf()}的读写对象是磁盘文件,而{\tf printf()}和{\tf scanf()}的读写对象是终端。
\end{frame}

\begin{frame}[fragile]\ft{\secname:{\tf fscanf()}和{\tf fprintf()}}
  它们的函数原型为:
  \begin{lstlisting}[language=c,backgroundcolor=\color{red!20}]
int fprintf(FILE * stream, const char *format, ...);
int fscanf (FILE * stream, const char *format, ...);
  \end{lstlisting}
  除增加“文件指针”外,其他与{\tf printf()/scanf()}用法相同。
\end{frame}

\begin{frame}[fragile]\ft{\secname:{\tf fscanf()}和{\tf fprintf()}}
  例如:
  \begin{lstlisting}[language=c,backgroundcolor=\color{red!20}]
fprintf(fp, "%d, %6.2f", i, t);
  \end{lstlisting}
  它的作用是将整型变量{\tf i}和浮点型变量{\tf t}的值按{\tf \%d}和{\tf \%6.2f}的格式输出到{\tf fp}所指向的文件中。 \vspace{0.1in}


  如果{\tf i=3, t=4.5},则输出到磁盘文件上的是以下字符串:
  \begin{lstlisting}[language=c,backgroundcolor=\color{red!20}]
3, 4.50
  \end{lstlisting}

\end{frame}

\begin{frame}[fragile]\ft{\secname:{\tf fscanf()}和{\tf fprintf()}}
  同样,用{\tf fscanf()}可以从磁盘文件上读入{\tf ASCII}字符:
  \begin{lstlisting}[language=c,backgroundcolor=\color{red!20}]
fscanf(fp, "%d, %f", &i, &t);
  \end{lstlisting}
  磁盘文件上如果有以下字符:
  \begin{lstlisting}[language=c,backgroundcolor=\color{red!20}]
3, 4.5
  \end{lstlisting}
  则将磁盘文件的数据{\tf 3}送给变量{\tf i},{\tf 4.5}送给变量{\tf t}。
\end{frame}

\begin{frame}[fragile]\ft{\secname:{\tf fscanf()}和{\tf fprintf()}的应用举例}
  \begin{li}
    编制程序,向磁盘文件添加单词,并显示文件内容在屏幕上。
  \end{li}
\end{frame}


\begin{frame}[fragile,allowframebreaks]\ft{\secname:{\tf fscanf()}和{\tf fprintf()}的应用举例}
  \lstinputlisting
  [language=c,numbers=left,frame=single]
  {code/ex13_05.c}
\end{frame}


\begin{frame}[fragile]\ft{\secname:{\tf fscanf()}和{\tf fprintf()}}
  用{\tf fprintf()}和{\tf fscanf()}对磁盘文件操作,由于在输入时要将{\tf ASCII}码转换为二进制形式,在输出时又要将二进制转换为字符,花费时间比较多,因此,在内存与磁盘频繁交换数据的情况下,最好不用{\tf fprintf()}和{\tf fscanf()},而用{\tf fread()}和{\tf fwrite()}。
\end{frame}

\begin{frame}[fragile]\ft{\secname:{\tf putw()}和{\tf getw()}}
  {\tf putw()}和{\tf getw()}用来对磁盘文件读写一个字(整数)。例如:
  \begin{lstlisting}[language=c,backgroundcolor=\color{red!20}]
putw(10, fp);    /* `整数10写入文件fp` */
i = getw(fp);     /* `从文件fp读一个整数给变量i` */
  \end{lstlisting}
\end{frame}

\begin{frame}[fragile]\ft{\secname:{\tf fgets()}和{\tf fputs()}}
  {\tf fgets()}的作用是从指定文件读入一个字符串。例如:
  \begin{lstlisting}[language=c,backgroundcolor=\color{red!20}]
fgets(str, n, fp); // `从文件fp读n-1个字节到str`
                   // `str最后一个字节加`'\0'
  \end{lstlisting}
  \vspace{0.1in}

  {\tf fputs()}的作用是向指定的文件输出一个字符串。例如:
  \begin{lstlisting}[language=c,backgroundcolor=\color{red!20}]
fputs(str, fp);    // `把字符串str写入fp` 
  \end{lstlisting}

\end{frame}

\begin{frame}[fragile]\ft{\secname:{\tf fread()}和{\tf fwrite()}}
  以下代码将{\tf num}作为一个含{\tf 8}个字符的字符串{\tf 0.333333}存储
  \begin{lstlisting}[language=c]
double num = 1./3.;
fprintf(fp,"%f", num);    
\end{lstlisting}
  \begin{itemize}
  \item 使用{\tf \%.2}可以把它存储为含{\tf 4}个字符的字符串{\tf 0.33};\\
  \item 使用{\tf \%.12f}可把它存储为含{\tf 14}个字符的字符串{\tf 0.333333333333}。
  \end{itemize}
  改变占位符可以改变存储这一值所需的空间大小,也会导致存储不同的值。
\end{frame}

\begin{frame}[fragile]\ft{\secname:{\tf fread()}和{\tf fwrite()}}

  在{\tf num}的值存为{\tf 0.33}以后,读取文件时就没法恢复其精度。总之,\textcolor{acolor1}{{\tf fprintf()}以一种可能改变数字值的方式将其转换为字符串。}
\end{frame}

\begin{frame}[fragile]\ft{\secname:{\tf fread()}和{\tf fwrite()}}
  \begin{itemize}
  \item 最精确的存储数字的方法就是使用与程序所使用的相同的位格式。故一个double值应该存储在一个double大小的单元中。\\[0.1in]
  \item 如果把数据存储在一个使用与程序具有相同表示方法的文件中,就称数据以二进制形式存储。这中间没有从数字形式到字符串形式的转换。\\[0.1in]
  \item \textcolor{acolor3}{{\tf fread()}和{\tf fwrite()}提供了这种二进制服务,用来读写一个数据块。  }

  \end{itemize}

\end{frame}

\begin{frame}[fragile]\ft{\secname:{\tf fread()}和{\tf fwrite()}}
  {\tf fwrite()}将二进制数据写入文件,其原型为
  \begin{lstlisting}[language=c,backgroundcolor=\color{red!20}]
size_t fwrite(const void * ptr, size_t size, size_t nmemb, FILE * fp);
  \end{lstlisting}
  其中:
  \begin{itemize}
  \item 指针{\tf ptr}是要写入的数据块的地址。
  \item {\tf size}表示要写入的数据块的大小(以字节为单位)。
  \item {\tf nmemb}表示数据块的数目。
  \item {\tf fp}指定要写入的文件
  \item 返回成功写入的项目数。正常情况下,它与{\tf nmemb}相等,若有错,返回值会小于{\tf nmemb}。
  \end{itemize}
\end{frame}

\begin{frame}[fragile]\ft{\secname:{\tf fread()}和{\tf fwrite()}}
  要保存一个{\tf 256}字节大小的数据对象(如一个数组),可以这么做
  \begin{lstlisting}[language=c,backgroundcolor=\color{red!20}]
char buffer[256];
fwrite(buffer, 256, 1, fp)
  \end{lstlisting}
  这一调用将一块{\tf 256}字节大小的数据块从缓冲区写入文件。
\end{frame}

\begin{frame}[fragile]\ft{\secname:{\tf fread()}和{\tf fwrite()}}
  要保存一个含{\tf 10}个{\tf double}值的数组,可以这么做
  \begin{lstlisting}[language=c,backgroundcolor=\color{red!20}]
double arr[10];
fwrite(arr, sizeof(double), 10, fp)
  \end{lstlisting}
  这一调用将{\tf arr}数组中的数据写入文件,数据分为{\tf 10}块,每块都是{\tf double}大小。
\end{frame}


\begin{frame}[fragile]\ft{\secname:{\tf fread()}和{\tf fwrite()}}
  {\tf fread()}从文件中读取二进制数据,其原型为
  \begin{lstlisting}[language=c,backgroundcolor=\color{red!20}]
size_t fread(const void * ptr, size_t size, size_t nmemb, FILE * fp);
  \end{lstlisting}
  其中:
  \begin{itemize}
  \item 指针{\tf ptr}是读入文件数据的内存地址。
  \item {\tf size}表示要读入的数据块的大小(以字节为单位)。
  \item {\tf nmemb}表示数据块的数目。
  \item {\tf fp}指定要读入的文件
  \item 返回成功读入的项目数。正常情况下,它与{\tf nmemb}相等,若有错,返回值会小于{\tf nmemb}。
  \end{itemize}
\end{frame}

\begin{frame}[fragile]\ft{\secname:{\tf fread()}和{\tf fwrite()}}
  要恢复前一例子中保存的含{\tf 10}个{\tf double}值的数组,可以这么做
  \begin{lstlisting}[language=c,backgroundcolor=\color{red!20}]
double arr[10];
fread(arr, sizeof(double), 10, fp)
  \end{lstlisting}
  这一调用将{\tf 10}个{\tf double}值复制到{\tf arr}数组中。
\end{frame}


\begin{frame}[fragile]\ft{\secname:{\tf fread()}和{\tf fwrite()}}
  如果有如下的结构体类型:
  \begin{lstlisting}[language=c,backgroundcolor=\color{red!20}]
struct student_type
{
  char name[10]; 
  int num; 
  int age; 
  char addr[30];
} stu[40];
  \end{lstlisting}
  结构体数组{\tf stu}有{\tf 40}个元素,每一个元素用来存放一个学生的数据。
\end{frame}

\begin{frame}[fragile]\ft{\secname:{\tf fread()}和{\tf fwrite()}}
  假设学生的数据已经存放在磁盘文件中,可以用下面的{\tf for}语句和{\tf fread()}读入{\tf 40}个学生的数据:
  \begin{lstlisting}[language=c,backgroundcolor=\color{red!20}]
for(i=0; i<40; i++)
  fread(&stu[i], sizeof(struct student_type), 1, fp);
  \end{lstlisting}
  或:
  \begin{lstlisting}[language=c,backgroundcolor=\color{red!20}]
fread(stu, sizeof(struct student_type), 40, fp);
  \end{lstlisting}
\end{frame}

\begin{frame}[fragile]\ft{\secname:{\tf fread()}和{\tf fwrite()}}
  以下程序可以将内存中的学生数据输出到磁盘文件中去:
  \begin{lstlisting}[language=c,backgroundcolor=\color{red!20}]
for(i=0; i<40; i++) 
 fwrite(&stu[i], sizeof(struct student_type), 1, fp);
  \end{lstlisting}
  或者只写一次
  \begin{lstlisting}[language=c,backgroundcolor=\color{red!20}]
fwrite(stu, sizeof(struct student_type), 40, fp); 
  \end{lstlisting}
\end{frame}

\begin{frame}[fragile]\ft{\secname:{\tf fread()}和{\tf fwrite()}的应用举例}
  \begin{li}
    从键盘上输入一批学生数据,然后存储到磁盘上。
  \end{li}
\end{frame}


\begin{frame}[fragile,allowframebreaks]\ft{\secname:{\tf fread()}和{\tf fwrite()}的应用举例}
  \lstinputlisting
  [language=c,numbers=left,frame=single]
  {code/ex13_03.c}
\end{frame}


\begin{frame}[fragile]\ft{\secname:{\tf fread()}和{\tf fwrite()}的应用举例}
  \begin{li}
    将一个浮点型数组读入磁盘文件,并从磁盘文件中读取数据显示在屏幕中。
  \end{li}
\end{frame}


\begin{frame}[fragile,allowframebreaks]\ft{\secname:{\tf fread()}和{\tf fwrite()}的应用举例}
  \lstinputlisting
  [language=c,numbers=left,frame=single]
  {code/ex13_04.c}
\end{frame}

