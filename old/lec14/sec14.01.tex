\section{文件类型}

\begin{frame}[fragile]\ft{\secname}
所谓“文件”是指一组相关数据的有序集合。 \vspace{.05in}

\begin{itemize}
\item 数据以文件的形式存放在外部介质(一般是磁盘、磁带、光盘等)上,在操作系统中是以文件为单位对数据进行管理的。\\[0.1in]
\item 以文件名作为访问文件的标识。
\end{itemize}
\end{frame}

\begin{frame}[fragile]\ft{\secname}
C语言把文件看作一个字节序列,即由一连串的字节组成。根据文件中的数据组织形式,数据文件可分为ASCII码文件和二进制文件。\vspace{.05in}

  \begin{itemize}
  \item ASCII码文件,也称“文本文件”,其每一个字节存放一个ASCII码。\\[0.1in]
  \item 二进制文件,把内存中的数据按其在内存中的存储形式存放在磁盘上。
  \end{itemize}
\end{frame}

\begin{frame}[fragile]\ft{\secname}
例:十进制整数{\tf 10000},在内存中占两个字节,其存放形式是:{\tf 00100111 00010000}。\vspace{.05in}

  \begin{itemize}
  \item 在二进制文件中也按这种方式存放。\\[0.1in]
  \item 在{\tf ASCII}文件中,十进制整数{\tf 10000}存放为{\tf 31H、30H、30H、30H、30H},占五个字节,它们分别是{\tf 1、0、0、0、0、0}字母的{\tf ASCII}码。
  \end{itemize}
\end{frame}

\begin{frame}[fragile]\ft{\secname}
按照操作系统对磁盘文件的读写方式,文件可以分为“缓冲文件系统”和“非缓冲文件系统”。
\end{frame}

\begin{frame}[fragile]\ft{\secname}
  \begin{dingyi}[缓冲文件系统]
    操作系统在内存中为每一个正在使用的文件开辟一个读写缓冲区。
  \end{dingyi}


  \begin{itemize}
  \item 从内存向磁盘输出数据必须先送到内存中的缓冲区,装满缓冲区后才一起送到磁盘上。\\[0.1in]
  \item 如果从磁盘向内存读入数据,则一次从磁盘文件将一批数据输入到内存缓冲区,然后再从缓冲区逐个地将数据送到程序数据区。
   \end{itemize} 
  \begin{dingyi}[非缓冲文件系统]
    操作系统不自动开辟确定大小的读写缓冲区,而由程序为每个文件设定缓冲区。
  \end{dingyi}
\end{frame}

\begin{frame}[fragile]\ft{\secname}

  \begin{itemize}
  \item
    在{\tf UNIX}系统下,用缓冲文件系统来处理文本文件,用非缓冲文件系统处理二进制文件。\\[0.1in]
  \item
    {\tf ANSI C}标准只采用缓冲文件系统。
   \end{itemize}  
\end{frame}

\begin{frame}[fragile]\ft{\secname}
缓冲文件系统中,每一个使用的文件都在内存中开辟一个“文件信息区”,用来存放文件的相关信息(文件的名字、文件当前的读写位置、文件操作方式等)。这些信息保存在一个结构体变量中,该结构体是由系统定义的,取名为{\tf FILE}。
\end{frame}

\begin{frame}[fragile]\ft{\secname}
在{\tf stdio.h}中,有以下文件类型的声明:
\begin{lstlisting}[language=c,backgroundcolor=\color{red!20}]
typedef struct {
  int level;            // 缓冲区“满”或“空”的程度
  unsigned flags;       // 文件状态标志
  char fd;              // 文件描述符
  unsigned char hold;   // 如无缓冲区不读取字符
  int bsize;            // 缓冲区的大小
  unsigned char * buffer;// 数据缓冲区的位置
  unsigned char * curp;  // 指针,当前的指向
  unsigned istemp;      // 临时文件,指示器
  short token;          // 用于有效性检查
} FILE;  
\end{lstlisting}
\end{frame}

\begin{frame}[fragile]\ft{\secname}
定义文件指针变量的一般形式为:
\begin{lstlisting}[language=c,backgroundcolor=\color{red!20}]
FILE *`文件结构指针变量名`  
\end{lstlisting}


例如:
\begin{lstlisting}[language=c,backgroundcolor=\color{red!20}]
FILE * fp;
\end{lstlisting}


注意:只有通过文件指针,才能调用相应的文件。
\end{frame}






