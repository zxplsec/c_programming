% -*- coding: utf-8 -*-
% !TEX program = xelatex

\documentclass[10pt,notheorems]{beamer}

\usetheme[style=beta]{epyt} % alpha, beta, delta, gamma, zeta

\usepackage[UTF8,noindent]{ctex}
\usepackage{extarrows}
%\usepackage{courier}
\usepackage{animate}
\usepackage{dcolumn}
\usepackage{pgf}
\usepackage{tikz}
\usetikzlibrary{calc}
\usetikzlibrary{arrows,snakes,backgrounds,shapes,patterns}
\usetikzlibrary{matrix,fit,positioning,decorations.pathmorphing}
\usepackage{listings}
\lstset{
        frame=no,
        keywordstyle=\color{acolor1},  
        basicstyle=\ttfamily\small,
        commentstyle=\color{acolor5},
        breakindent=0pt,
        rulesepcolor=\color{red!20!green!20!blue!20},
        rulecolor=\color{black},
        tabsize=4,
        numbersep=5pt,
        numberstyle=\footnotesize,
        breaklines=true,
        %% backgroundcolor=\color{red!10},
        showspaces=false,
        showtabs=false,
        showstringspaces=false,
        extendedchars=false,
        escapeinside=``
}



\newcommand{\mylead}[1]{\textcolor{acolor1}{#1}}
\newcommand{\mybold}[1]{\textcolor{acolor2}{#1}}
\newcommand{\mywarn}[1]{\textcolor{acolor3}{#1}}

%%%% \renewcommand *****
%\renewcommand{\lstlistingname}{}
\newcommand{\tf}{\ttfamily}
%\newcommand{\ttt}{\texttt}
%\newcommand{\blue}{\textcolor{blue}}
%\newcommand{\red}{\textcolor{red}}
%\newcommand{\purple}{\textcolor{purple}}
\newcommand{\ft}{\frametitle}
\newcommand{\fst}{\framesubtitle}
\newcommand{\bs}{\boldsymbol}
\newcommand{\ds}{\displaystyle}
\newcommand{\vd}{\vdots}
\newcommand{\cd}{\cdots}
\newcommand{\dd}{\ddots}
\newcommand{\id}{\iddots}
\newcommand{\XX}{\mathbf{X}}
\newcommand{\PP}{\mathbf{P}}
\newcommand{\QQ}{\mathbf{Q}}
\newcommand{\xx}{\mathbf{x}}
\newcommand{\yy}{\mathbf{y}}
\newcommand{\bb}{\mathbf{b}}
\newcommand{\abd}{\boldsymbol{a}}

\renewcommand{\proofname}{证明}




%\newtheorem{theorem}{定理}
%\newtheorem{definition}[theorem]{定义}
%\newtheorem{example}[theorem]{例子}
%\newtheorem{dingli}[theorem]{定理}
%\newtheorem{li}[theorem]{例}
%
%\newtheorem*{theorem*}{定理}
%\newtheorem*{definition*}{定义}
%\newtheorem*{example*}{例子}
%\newtheorem*{dingli*}{定理}
%\newtheorem*{li*}{例}

\renewcommand{\proofname}{证明}
\newtheorem*{jie}{解}
\newtheorem*{zhu}{注}
\newtheorem*{dingli}{定理} 
\newtheorem*{dingyi}{定义} 
\newtheorem*{xingzhi}{性质} 
\newtheorem*{tuilun}{推论} 
\newtheorem{li}{例} 
\newtheorem*{jielun}{结论} 
\newtheorem*{zhengming}{证明}
\newtheorem*{wenti}{问题}
\newtheorem*{jieshi}{解释}
\newtheorem{biancheng}{编程}

\renewcommand{\proofname}{证明}

\begin{document}

\title{第12次C上机}
\subtitle{结构体}
\author{张晓平}
\institute{武汉大学数学与统计学院}


\begin{frame}[plain]\transboxout
\titlepage
\end{frame}

% \begin{frame}\transboxin
% \begin{center}
% \tableofcontents[]%hideallsubsections]
% \end{center}
% \end{frame}

% \AtBeginSection[]{
% \begin{frame}[allowframebreaks]
% \tableofcontents[currentsection,sectionstyle=show/hide]
% \end{frame}
% }
%\AtBeginSubsection[]{
%\begin{frame}[allowframebreaks]
%\tableofcontents[currentsection,currentsubsection,subsectionstyle=show/shaded/hide]
%\end{frame}
%}


\begin{frame}[fragile]
  \begin{li}
    编写一个程序,由用户键入日、月和年。月份可以是月份号、月份名或月份简写。然后程序返回一年中到给定日子(包括这一天)的总天数。注,当以下情况之一满足时,这一年是闰年: 
    \begin{enumerate}
    	\item 年份是4的倍数而不是100的倍数; 
    	\item 年份是400的倍数。
    \end{enumerate}
\end{li}
\end{frame}


\begin{frame}[fragile,allowframebreaks]
\lstinputlisting
[language=c,numbers=left,frame=single]
{Code_OP/ex12_01.c}
\end{frame}

\begin{frame}[fragile]
  \begin{li}
    编写一个程序,按照以下要求,创建一个含有两个成员的结构模板:
    \begin{enumerate}
    \item 第一个成员是学号;
    \item 第二个成员是一个包含两个成员的结构,其中
      \begin{itemize}
        \item 第一个成员是姓;
        \item 第二个成员是名。
      \end{itemize}
    \end{enumerate}
    创建并初始化一个包含5个此类型的数组。编写一个函数实现以下形式的输出,将结构数组传递给这个函数:
      \begin{lstlisting}
Li Ming - 20161101
      \end{lstlisting}
\end{li}
\end{frame}


\begin{frame}[fragile,allowframebreaks]
\lstinputlisting
[language=c,numbers=left,frame=single]
{Code_OP/ex12_02.c}
\end{frame}

\begin{frame}[fragile]
\begin{li}
	编写一个程序,满足以下条件:
	\begin{enumerate}
		\item 定义一个name结构模板,它含两个成员:一个字符串存放名字,另一个字符串存放姓氏;
		\item 定义一个student结构模板,它含3个成员:一个name结构,一个存放3个浮点数分数的grade数组,以及一个存放这3个分数的平均分的变量;
		\item 使用main()声明一个具有CSIZE(=4)个student结构的数组,并随意初始化这些结构的名字部分。使用函数4、5、6以及7部分所描述的任务。
		\item 
		请求用户输入学生姓名和分数,以交互地获取每个学生的成绩。将分数放到相应结构的grade数组成员中。
		\item 为每个结构计算平均分,并将这个值赋给合适的成员。
		\item 输出每个结构中的信息。
		\item 输出结构的每个数值成员的班级平均分。
	\end{enumerate}
\end{li}
\end{frame}


\begin{frame}[fragile,allowframebreaks]
\lstinputlisting
[language=c,numbers=left,frame=single]
{Code_OP/ex12_02.c}
\end{frame}

\end{document}
