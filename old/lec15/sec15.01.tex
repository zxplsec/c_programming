\section{位操作介绍}
\begin{frame}
C语言的设计具备了汇编语言的运算能力,它支持全部的位操作符。位操作符是对字节或字中的位进行测试、置位或移位处理,在对微处理器的编程中,特别适合对寄存器、I/O端口进行操作。
\end{frame}

\begin{frame}
  C语言提供了六种位操作运算符:
  \begin{itemize}
  \item 位与
  \item 位或
  \item 位异或
  \item 取反
  \item 左移
  \item 右移
  \end{itemize}
  这些运算符只能用于整型操作数,即只能用于带符号或无符号的{\tf char, short, int}与{\tf long}类型。
\end{frame}

\begin{frame}
  \begin{table}
    \centering
    \begin{tabular}{p{0.5cm}|p{1.5cm}|p{8cm}}\hline
      {\tf \&} & 位与 & 如果两个相应的二进制位都为1,则该位的结果值为1,否则为0\\\hline
      {\tf |}  & 位或 & 两个相应的二进制位中只要有一个为1,该位的结果值为1\\\hline
      {\tf $\hat{}$}  & 位异或 & 若参加运算的两个二进制位值相同则为0,否则为1\\\hline
      {\tf $\tilde{}$} & 取反 & 一元运算符,用来对一个二进制数按位取反,即将0变1,将1变0\\\hline
      {\tf <<} & 左移 & 用来将一个数的各二进制位全部左移N位,右补0\\\hline
      {\tf >>} & 右移 & 将一个数的各二进制位右移N位,移到右端的低位被舍弃,对于无符号数,高位补0\\\hline
    \end{tabular}
  \end{table}
\end{frame}


\begin{frame}
  \begin{dingyi}[位与运算]
    参加运算的两个数据,按二进制位进行“与”运算。   如果两个相应的二进制位都为1,则该位的结果值为1;否则为0。
  \end{dingyi}
\end{frame}


\begin{frame}[fragile]
  \begin{li}
    计算{\tf 3 \& 5}
  \end{li}
  \begin{jie}
    因{\tf 3 = 00000011(2)},{\tf 5 = 00000101(2)},而位与运算为
    \begin{lstlisting}
         00000011
      &  00000101
      -------------
         00000001
    \end{lstlisting}
    故{\tf 3 \& 5 = 1}。
  \end{jie}
\end{frame}


\begin{frame}
  \lstinputlisting
  [language=c,frame=single,numbers=left]
  {code/and1.c}
\end{frame}


\begin{frame}
  位与的用途:

  \begin{itemize}
  \item \blue{清零}
  \item[]  若想对一个存储单元清零,即使其全部二进制位为{\tf 0},只要找一个二进制数,其中各位符合以下条件: 原来的数中为{\tf 1}的位,新数中相应位为{\tf 0}。然后使二者进行{\tf \&}运算,即可达到清零目的。
  \end{itemize}
\end{frame}


\begin{frame}[fragile]
  \begin{li}
    设原数为{\tf 43 = 00101011(2)},另找一个数{\tf 148 = {10010100(2)}},将两者做位与运算:
    \begin{lstlisting}
         00101011
      &  10010100
      -------------
         00000000
    \end{lstlisting}
  \end{li}
\end{frame}

\begin{frame}
  \lstinputlisting
  [language=c,frame=single,numbers=left]
  {code/and2.c}
\end{frame}


\begin{frame}

  \begin{itemize}
  \item \blue{取一个数中某些指定位}
  \item[]  若有一个整数{\tf a (2 byte)},想要取其中的低字节,只需要将{\tf a}与{\tf 8}个{\tf 1}做位与运算即可。
  \end{itemize}
\end{frame}


\begin{frame}[fragile]
  \begin{li}
    \begin{lstlisting}
         00101100 10101100
      &  00000000 11111111
      ---------------------
         00000000 10101100
    \end{lstlisting}
  \end{li}
\end{frame}


\begin{frame}[fragile]

  \begin{itemize}
  \item \blue{保留指定位}
  \end{itemize}
  
  \begin{li}
    有一个数{\tf 84 = 01010100(2)}, 想把其中从左边算起的第{\tf 3, 4, 5, 7, 8}位保留下来,可做以下运算:
    \begin{lstlisting}
         01010100
      &  00111011
      ---------------------
         00010000
    \end{lstlisting}
  \end{li}
\end{frame}


\begin{frame}[fragile]
  一元运算符 {\tf \~{}} 将每个1变为0,将每个0变为1。
  \begin{li}
    \begin{lstlisting}
      ~  00111011
      ---------------------
         11000100
    \end{lstlisting}
  \end{li}
\end{frame}
