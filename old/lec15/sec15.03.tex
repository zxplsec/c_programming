\section{C的位运算符}
\begin{frame}

  C提供位的逻辑运算符和移位运算符。
\end{frame}
\subsection{位逻辑运算符}
\begin{frame}
  C语言提供了六种位逻辑运算符:
  \begin{itemize}
  \item 位与
  \item 位或
  \item 位异或
  \item 取反
  \end{itemize}
  这些运算符只能用于整型数据,即只能用于带符号或无符号的{\tf char, short, int}与{\tf long}类型。将这些运算符称为位{\tf (bitwise)}的原因是它们对每个位进行操作,而不影响左右两侧的位。
\end{frame}

\begin{frame}
  \begin{zhu}
    请不要将这些运算符与常规的逻辑运算符相混淆{\tf (\&\&, ||, !)},常规的逻辑运算符对整个值进行操作。
  \end{zhu}
\end{frame}

\begin{frame}[fragile]
  \blue{ 一、二进制反码或按位取反:{\tf \~{}}}
  
  一元运算符 {\tf \~{}} 将每个1变为0,将每个0变为1。
  \begin{li}
    \begin{lstlisting}[backgroundcolor=\color{red!20}]
      ~  00111011
      ---------------------
         11000100
    \end{lstlisting}
  \end{li}
\end{frame}


\begin{frame}[fragile]
  设{\tf val}是一个{\tf unsigned char},已赋值为{\tf 2}。因{\tf 2 = 00000010(2)},故{\tf \~{}val}的值为{\tf 11111101},即{\tf 253}。

  \pause \vspace{.1in}
  
  \begin{zhu}
    该运算符不改变{\tf val}的值。

    若想将{\tf val}的值变为{\tf \~{}val},可以这样做:
    \begin{lstlisting}[backgroundcolor=\color{red!20}]
      val = ~val;
    \end{lstlisting}
  \end{zhu}
\end{frame}


\begin{frame}[fragile]
  \blue{二、位与:{\tf \&}}

  {\tf \&}运算符通过对两个操作数逐位进行比较产生一个新值。对于每个位,只有两个操作数的对应位都为{\tf 1},结果才为{\tf 1}。
  \begin{lstlisting}[backgroundcolor=\color{red!20}]
       00000011
    &  00000101
    -------------
       00000001
  \end{lstlisting}
\end{frame}

\begin{frame}[fragile]

  \begin{zhu}
    C也有一个位与-赋值运算符:{\tf \&=}。
  \end{zhu}
  以下两条语句等效:
  \begin{lstlisting}[backgroundcolor=\color{red!20}]
    val &= 0377;
    val = val & 0377;
  \end{lstlisting}
\end{frame}



\begin{frame}[fragile]
  \blue{三、位或:{\tf |}}

  {\tf |}运算符通过对两个操作数逐位进行比较产生一个新值。对于每个位,如果任意操作数中对应的位为{\tf 1},那么结果位就为{\tf 1}。
  \begin{lstlisting}[backgroundcolor=\color{red!20}]
       10010011
    |  00111101
    -------------
       10111111
  \end{lstlisting}
\end{frame}


\begin{frame}[fragile]

  \begin{zhu}
    C也有一个位或-赋值运算符:{\tf |=}。
  \end{zhu}
  以下两条语句等效:
  \begin{lstlisting}[backgroundcolor=\color{red!20}]
    val |= 0377;
    val = val | 0377;
  \end{lstlisting}
\end{frame}

\begin{frame}[fragile]
  \blue{四、位异或:{\tf \^{}}}

  {\tf \^{}}运算符通过对两个操作数逐位进行比较产生一个新值。对于每个位,如果任意操作数中对应的位有一个为{\tf 1},但不都为{\tf 1},那么结果位就为{\tf 1}。
  \begin{lstlisting}[backgroundcolor=\color{red!20}]
       10010011
    ^  00111101
    -------------
       10101110
  \end{lstlisting}
\end{frame}


\begin{frame}[fragile]

  \begin{zhu}
    C也有一个位异或-赋值运算符:{\tf \^{}=}。
  \end{zhu}
  以下两条语句等效:
  \begin{lstlisting}[backgroundcolor=\color{red!20}]
    val ^= 0377;
    val = val ^ 0377;
  \end{lstlisting}
\end{frame}


\subsection{用法:掩码}
\begin{frame}
  位与运算符通常跟掩码一起使用。掩码是某些位设为开{\tf (1)}而某些位设为关{\tf (0)}的位组合。
\end{frame}

\begin{frame}[fragile]
  \begin{li}
    设已经定义符号常量{\tf MASK}为{\tf 2},即二进制的{00000010},只有位{\tf 1}是非零。则以下语句
    \begin{lstlisting}[backgroundcolor=\color{red!20}]
      flags = falgs & MASK;
    \end{lstlisting}
    将导致{\tf flags}除位{\tf 1}外的所有位都设置为{\tf 0}。因掩码中的{\tf 0}覆盖了{\tf flags}中相应的位,故该过程称为\blue{“使用掩码”}。
  \end{li}
\end{frame}

\begin{frame}[fragile]
  以此类推,可将掩码中的{\tf 0}看做不透明,将{\tf 1}看做透明。表达式{\tf flags \& MASK}就好像使用掩码覆盖{\tf flags}:{\tf flags}中的位只有在{\tf MASK}中的对应位是{\tf 1}时才可见。
\end{frame}

\begin{frame}[fragile]
  可使用\blue{“位与-赋值”运算符}来简化代码,如
  \begin{lstlisting}[backgroundcolor=\color{red!20}]
    flag &= MASK;
  \end{lstlisting}
  \pause \vspace{.1in}

  一种常见的C用法如下
  \begin{lstlisting}[backgroundcolor=\color{red!20}]
    ch &= 0xff;
  \end{lstlisting}
  因{\tf 0xff}的二进制形式为{\tf 11111111},该掩码留下{\tf ch}的低{\tf 8}位,将其余位设为{\tf 0}。
\end{frame}


\subsection{用法:打开位}
\begin{frame}
  有时,可能需要打开一个值中特定的位,同事保持其他位不变。例如,一台PC通过将知发送到端口来控制硬件。如要打开扬声器,可能需要打开某一位,同时保持其他位不变。\blue{可以使用“位或”运算符来实现。}
\end{frame}

\begin{frame}[fragile]
  \begin{li}
    设{\tf MASK}为{\tf 00000010},以下语句
    \begin{lstlisting}[backgroundcolor=\color{red!20}]
      flags = flags | MASK;
    \end{lstlisting}
    将{\tf flags}中的位{\tf 1}设为{\tf 1},并保留其他所有位不变。

    \pause \vspace{.1in}

    作为缩写,可使用\blue{位或-赋值}运算符:
    \begin{lstlisting}[backgroundcolor=\color{red!20}]
      flags |= MASK;
    \end{lstlisting}
  \end{li}
\end{frame}

\subsection{用法:关闭位}
\begin{frame}[fragile]
  关闭特定位与打开特定位是同样有用的。假设想关闭{\tf flags}中的位{\tf 1}。设{\tf MASK}为{\tf 00000010},则以下语句
  \begin{lstlisting}[backgroundcolor=\color{red!20}]
    flags = flags & ~MASK;
  \end{lstlisting}
  将关闭位{\tf 1},而其他所有位不变。 其缩写形式为
  \begin{lstlisting}[backgroundcolor=\color{red!20}]
    flags &= ~MASK;
  \end{lstlisting}
\end{frame}

\subsection{用法:转置位}
\begin{frame}
  \blue{转置一个位}表示如果该位打开,则关闭该位;如果该位关闭,则打开该位。
  \blue{可使用“位异或”运算符来转置一个位。}
\end{frame}

\begin{frame}
  其思想为:  设{\tf b}是一个位({\tf 1}或{\tf 0}),则
  \begin{itemize}
  \item 若{\tf b}为{\tf 1},则{\tf 1\^{}b}为{\tf 0};
  \item 若{\tf b}为{\tf 0},则{\tf 1\^{}b}为{\tf 1};
  \item 无论{\tf b}是{\tf 0}还是{\tf 1},{\tf 0\^{}b}总是为{\tf b}。
  \end{itemize}
  因此,若使用{\tf \^{}}将一个值与掩码组合,则该值中对应掩码位为{\tf 1}的位被转置,对应掩码位为{\tf 0}的位不改变。
\end{frame}

\begin{frame}[fragile]
  \begin{li}
    要转置{\tf flag}中的位{\tf 1},可使用以下任一语句:
    \begin{lstlisting}[backgroundcolor=\color{red!20}]
      flag = flag ^ MASK;
      flag ^= MASK;
    \end{lstlisting}
  \end{li}
\end{frame}

\subsection{用法:查看某一位的值}
\begin{frame}
  \begin{wenti}
    我们已经知道了改变一位的值的方法,那么如何来查看某一位的值呢?
  \end{wenti}
\end{frame}

\begin{frame}[fragile]
  \begin{li}
    查看{\tf flag}的位{\tf 1}的值。
  \end{li}
  \pause \vspace{.1in}

  不能简单的比较{\tf flag}与{\tf MASK (00000010)}:
  \begin{lstlisting}[backgroundcolor=\color{red!20}]
    if (flag == MASK)
      puts("Wow!");
  \end{lstlisting}
  \pause \vspace{0.1in}
  
  必须屏蔽{\tf flag}中的其他位,以便只把{\tf flag}中的位{\tf 1}和{\tf MASK}做比较:
  \begin{lstlisting}[backgroundcolor=\color{red!20}]
    if ( (flag & MASK) == MASK)
      puts("Wow!");
  \end{lstlisting}
  因位运算符的优先级低于{\tf ==},故需在{\tf flag \& MASK}的两侧加圆括号。
\end{frame}

\subsection{移位操作符}
\begin{frame}
  移位运算符将位向左或向右移。
\end{frame}

\begin{frame}
  \blue{一、左移:{\tf <<}} \vspace{.1in}

  左移运算符{\tf <<}将其左侧操作数的值的每位向左移动,移动的位数由其右侧操作数指定。
  空出的位用{\tf 0}填充,并且丢弃移出左侧操作数末端的位。
\end{frame}

\begin{frame}[fragile]
  \begin{lstlisting}[backgroundcolor=\color{red!20}]
    10001010 << 2
    -------------
    00101000
  \end{lstlisting}
  该操作产生一个新值,但不改变其操作数。如,设{\tf a}为{\tf 1},
  则{\tf a << 2}为{\tf 4},但{\tf a}仍为{\tf 1}。

  \vspace{.1in}
  \blue{可使用“左移-赋值”运算符{\tf (<<=)}来实际改变一个变量的值。}

  \begin{lstlisting}[backgroundcolor=\color{red!20}]
    int a = 1;
    int b;
    b = a << 2; // assign 4 to b
    a <<= 2;    // change a to 4
  \end{lstlisting}
\end{frame}

\begin{frame}
  \blue{二、右移:{\tf >>}} \vspace{.1in}

  左移运算符{\tf >>}将其左侧操作数的值的每位向右移动,移动的位数由其右侧操作数指定。
  丢弃移出左侧操作数右端的位。对于{\tf unsigned}类型,使用{\tf 0}填充左端空出的位;对于有符号类型,结果依赖于机器:空出的位可能用{\tf 0}填充,或使用符号(最左端的)位的副本来填充。
\end{frame}

\begin{frame}[fragile]
  对有符号值,
  \begin{lstlisting}[backgroundcolor=\color{red!20}]
    10001010 >> 2  // signed
    -------------
    00100010       // some os

    10001010 >> 2  // signed
    -------------
    11100010       // the other os
  \end{lstlisting}
  对无符号值,
  \begin{lstlisting}[backgroundcolor=\color{red!20}]
    10001010 >> 2  // signed
    -------------
    00100010       // any os
  \end{lstlisting}

\end{frame}

\begin{frame}[fragile]
  \blue{可使用“右移-赋值”运算符{\tf (>>=)}来实际改变一个变量的值。}

  \begin{lstlisting}[backgroundcolor=\color{red!20}]
    int a = 16;
    int b;
    b = a >> 2; // b = 2, a = 16
    a >>= 2;    // a = 2
  \end{lstlisting}
\end{frame}

\begin{frame}
  \blue{三、用法:移位运算符} \vspace{.1in}

  移位运算符能够提供快捷、高效的(依赖于硬件)对{\tf 2}的幂的乘法和除法。

  \begin{table}
    \centering
    \begin{tabular}{p{3cm}|p{6cm}}\hline
      {\tf number << n} & {\tf number}乘以{\tf 2}的{\tf n}次幂\\\hline
      {\tf number >> n} & 如果{\tf number}非负,则用{\tf number}除以{\tf 2}的{\tf n}次幂\\\hline
    \end{tabular}
  \end{table}
  这些移位运算符类似于在十进制中移动小数点来乘以或除以{\tf 10}。
\end{frame}

\begin{frame}
  移位运算符也用于从较大的单位中提取多组比特位。

  \begin{li}
    假设使用一个{\tf unsigned long}值代表颜色值,其中低位字节存放红色亮度,下一字节存放绿色亮度,第三个字节存放蓝色亮度。如何将每种颜色的亮度存储在各自的{\tf unsigned char}变量中。
  \end{li}
\end{frame}

\begin{frame}[fragile]
  \begin{lstlisting}[language=c,backgroundcolor=\color{red!20}]
#define MASK 0xff
unsigned long color = 0x002a162f;
unsigned char blue, green, red;
red = color & MASK;
green = (color >> 8)  & MASK;
blue  = (color >> 16) & MASK;
\end{lstlisting}
这段代码使用右移运算符将{\tf 8}位颜色值移动到低字节,然后使用掩码技术将低位字节赋给相应的变量。
\end{frame}

\subsection{编程实例}

\begin{frame}[fragile]
  \begin{li}
    使用移位运算符实现一个十进制整数的二进制转码。要求编写一个{\tf itobs()}函数,它接受两个参数,第一个是该整数,另一个是字符串地址。
  \end{li}

\end{frame}

\begin{frame}[fragile,allowframebreaks]
  \lstinputlisting
  [language=c,frame=single,numbers=left]
  {code/binbit.c}
\end{frame}

\begin{frame}[fragile]
\begin{lstlisting}[backgroundcolor=\color{red!20}]
Enter integers and see them in binary.
Non-numeric input terminates program.
7 [enter]
7 is 0000 0000 0000 0000 0000 0000 0000 0111
2017 [enter]
2017 is 0000 0000 0000 0000 0000 0111 1110 0001
-1 [enter]
-1 is 1111 1111 1111 1111 1111 1111 1111 1111
-2017 [enter]
-2017 is 1111 1111 1111 1111 1111 1000 0001 1111
q [enter]
Bye!
\end{lstlisting}
\end{frame}

\begin{frame}[fragile]
  \begin{li}
    编写一个函数,反转一个值中的最后{\tf n}位,参数为{\tf n}和要反转的值。
  \end{li}

\end{frame}

\begin{frame}[fragile]  
  取反运算符{\tf \~{}}可以反转位,但它反转一个值中的所有位,而不是选定的少数位。
  然而,异或运算符{\tf \^{}}可以用与转置单个位。

  \pause \vspace{.1in}
  创建一个掩码,它的最后{\tf n}位设为{\tf 1},其他位设为{\tf 0}。
  然后对该掩码和一个值使用异或运算符{\tf \^{}}就可以反转这个值的最后{\tf n}位,
  同事保留该值的其他位不变。
\end{frame}

\begin{frame}[fragile,allowframebreaks]
\begin{lstlisting}[language=c,backgroundcolor=\color{red!20}]
int invert_end(int num, int bits)
{
  int mask = 0;
  int bitva}l = 1;
  while (bits-- > 0) {
    mask |= bitval;
    bitval <<= 1;
  }
  return num ^ mask;
}
\end{lstlisting}
该代码中,{\tf while}循环创建该掩码。最初,{\tf mask}的所有位都被设为{\tf 0}。
第一次执行该循环将位{\tf 0}设为{\tf 1},然后将{\tf bitval}增加到{\tf 2};
也就是将{\tf bitval}的位{\tf 0}设为{\tf 0},位{\tf 1}设为{\tf 1}。
下次循环时,将{\tf mask}的位{\tf 1}设为{\tf 1},以此类推。
\end{frame}

\begin{frame}[fragile,allowframebreaks]
  \lstinputlisting
  [language=c,frame=single,numbers=left]
  {code/invert4.c}
\end{frame}

\begin{frame}[fragile]
  \begin{lstlisting}[backgroundcolor=\color{red!20}]
Enter integers and see them in binary.
Non-numeric input terminates program.
7
7 is
0000 0000 0000 0000 0000 0000 0000 0111
Inverting the last 4 bits gives
0000 0000 0000 0000 0000 0000 0000 1000
1234
1234 is
0000 0000 0000 0000 0000 0100 1101 0010
Inverting the last 4 bits gives
0000 0000 0000 0000 0000 0100 1101 1101
q
Bye!
  \end{lstlisting}
\end{frame}