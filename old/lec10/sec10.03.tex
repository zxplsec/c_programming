\section{指针与数组}

\begin{frame}[fragile]\ft{\secname}
\begin{itemize}
\item
计算机的硬件指令很大程度上依赖于地址,而指针为你使用地址提供了一种方法。\\[0.1in]
\item
于是,使用指针使你能够以类似于计算机底层的方式来表达你的意愿,从而让程序能更高效地工作。\\[0.1in]
\item 
特别地,指针能很有效的处理数组,实际上数组是一种变相使用指针的形式。
\end{itemize}
\end{frame}


\begin{frame}[fragile]\ft{\secname}
请记住:\textcolor{acolor1}{数组名是数组首元素的地址。}
\pause \vspace{0.2in}

若{\tf array}为一个数组,则以下关系式为真:
\begin{lstlisting}[language=c,backgroundcolor=\color{red!20}]
array == &array[0];
\end{lstlisting}
\end{frame}


\begin{frame}[fragile,allowframebreaks]\ft{\secname}
  \lstinputlisting
  [language=c,numbers=left,frame=single]
  {Code/pnt_add.c}
\end{frame}


\begin{frame}[fragile]\ft{\secname}

\begin{lstlisting}[backgroundcolor=\color{red!20}]
                  short         double
pointers + 0: 0x7fff5fbff7d0 0x7fff5fbff7b0
pointers + 1: 0x7fff5fbff7d2 0x7fff5fbff7b8
pointers + 2: 0x7fff5fbff7d4 0x7fff5fbff7c0
pointers + 3: 0x7fff5fbff7d6 0x7fff5fbff7c8
\end{lstlisting}
\end{frame}


\begin{frame}[fragile]\ft{\secname}
在C中,对指针加1的结果是对该指针增加一个存储单元。对数组而言,地址会增加到下一个元素的地址,而不是下一个字节。
\end{frame}


\begin{frame}[fragile]\ft{\secname}
\begin{center}
\textcolor{acolor1}{\Large 指针定义小结}
\end{center}
\begin{itemize}
\item 
指针的数值就是它所指向的对象的地址。地址的内部表达方式由硬件决定,很多计算机都是以字节编址的。 \\[0.1in]
\item
在指针前用运算符{\tf *}就可以得到该指针所指向的对象的值。\\[0.1in]
\item
对指针加1,等价于对指针的值加上它所指向的对象的字节大小。
\end{itemize}
\end{frame}


\begin{frame}[fragile]\ft{\secname}
\begin{lstlisting}[language=c,backgroundcolor=\color{red!20}]
dates + 2 == &dates[2];    // `相同的地址`
*(dates + 2) == dates[2];  // `相同的值`
\end{lstlisting}

可以用指针标识数组的每个元素,并得到每个元素的值。从本质上讲,这是对同一对象采用了两种不同的符号表示方法。

\end{frame}


\begin{frame}[fragile]\ft{\secname}
在描述数组时,C确实借助了指针的概念。例如,定义{\tf array[n]}时,\vspace{0.1in}

\begin{itemize}
\item
即:{\tf *(array + n)}, \\[0.1in]
\item
含义:“寻址到内存中的{\tf array},然后移动{\tf n}个单元,再取出数值”。
\end{itemize}
\end{frame}


\begin{frame}[fragile]\ft{\secname}
请注意{\tf *(dates+2)}和{\tf *dates+2}的区别。\textcolor{acolor1}{取值运算符{\tf *}的优先级高于{\tf +}},故后者等价于{\tf (*dates)+2}。\vspace{0.1in}

\begin{lstlisting}[language=c,backgroundcolor=\color{red!20}]
*(dates + 2)  // `dates的第三个元素的值`
*dates + 2    // `dates的第一个元素与2相加`
\end{lstlisting}
\end{frame}

\begin{frame}[fragile]\ft{\secname}
  \lstinputlisting
  [language=c,numbers=left,frame=single]  
  {Code/day_mon3.c}
\end{frame}

