\section{变长数组}
\begin{frame}[fragile]\ft{\secname}
对于处理二维数组的函数,数组的行可以在函数调用时传递,但数组的列却只能预置在函数内部。例如函数定义为
\begin{lstlisting}[language=c,backgroundcolor=\color{red!20}]
#define COLS 4
int sum2d(int ar[][COLS], int rows)
{
  int r;
  int c;
  int tot = 0;
  for (r = 0; r < rows; r++)
    for (c = 0; c < COLS; c++)
      tot += ar[r][c];
  return tot;
}
\end{lstlisting}
\end{frame}

\begin{frame}[fragile]\ft{\secname}
假设定义了如下数组
\begin{lstlisting}[language=c,backgroundcolor=\color{red!20}]
int ar1[5][4];
int ar2[100][4];
int ar3[2][4];
\end{lstlisting}
可以使用以下函数调用
\begin{lstlisting}[language=c,backgroundcolor=\color{red!20}]
tot = sum2d(ar1, 5);
tot = sum2d(ar2, 100);
tot = sum2d(ar3, 2);
\end{lstlisting}
\end{frame}

\begin{frame}[fragile]\ft{\secname}
如果要处理6行5列的数组,则需要新创建一个函数,其COLS定义为5。这是因为数组的列数必须是常量:不能用一个变量来代替COLS。
\end{frame}

\begin{frame}[fragile]\ft{\secname}
创建一个处理任意二维数组的函数,也是有可能的,但比较繁琐(这样的函数需要把数组当做一维数组来传递,然后由函数计算每行的起始地址)。
\end{frame}

\begin{frame}[fragile]\ft{\secname}
值得一提的是\textcolor{acolor1}{FORTRAN语言允许在函数调用时指定二维的大小}。虽然FORTRAN是很古老的编程语言,但多年来,数值计算专家们编制出了很多有用的FORTRAN计算库。C正在逐渐代替FORTRAN,因此如何能够简单地转换现有的FORTRAN库将是大有益处的。
\end{frame}

\begin{frame}[fragile]\ft{\secname}
出于以上原因,C99标准引入了变长数组(VLA),它允许使用变量定义数组各维数。\pause

\rule{\textwidth}{.2mm}
 \vspace{.1mm}

你可以使用以下声明
\begin{lstlisting}[language=c,backgroundcolor=\color{red!20}]
int m = 4;
int n = 5;
double array[m][n];
\end{lstlisting}
但变长数组有一些限制,如变长数组必须是自动存储类的,这意味着它们必须在函数内部或作为函数参量声明,而且声明时不可以进行初始化。
\end{frame}

\begin{frame}[fragile]\ft{\secname}
编写一个函数,用于计算任意二维int数组的和。
\end{frame}

\begin{frame}[fragile]\ft{\secname}
\begin{enumerate}
\item[1] 以下代码示范如何声明一个带有二维变长数组参数的函数:
\begin{lstlisting} [language=c,backgroundcolor=\color{red!20}]
int sum2d(int rows, int cols, int ar[rows][cols]);
\end{lstlisting}
请注意
%前面两个参量rows和cols用作数组参量ar的维数,而ar的声明中使用了rows和cols,故
在参量列表中,{\tf rows}和{\tf cols}的声明需要早于{\tf ar}。\\[0.1in]

\item[] 以下原型是错误的(顺序不对):
\begin{lstlisting} [language=c,backgroundcolor=\color{red!20}]
int sum2d(int ar[rows][cols], int rows, int cols);
\end{lstlisting}\vspace{0.05in}

\item[] 可简写为
\begin{lstlisting} [language=c,backgroundcolor=\color{red!20}]
int sum2d(int, int,int ar[*][*]);
\end{lstlisting}
若省略名称,则需用星号代替方括号中省略的维数。
\end{enumerate}
\end{frame}

\begin{frame}[fragile]\ft{\secname}
\begin{enumerate}
\item[2] 函数定义为
\begin{lstlisting} [language=c,backgroundcolor=\color{red!20}]
int sum2d(int rows, int cols, int ar[rows][cols])
{
  int r;
  int c;
  int tot = 0;
  for (r = 0; r < rows; r++)
    for (c = 0; c < cols; c++)
      tot += ar[r][c];
  return tot;
}
\end{lstlisting}
\end{enumerate}
\end{frame}

\begin{frame}[fragile,allowframebreaks]\ft{\secname}
\lstinputlisting
[language=c,numbers=left,frame=single]
{Code/vararr2d.c}
\end{frame}


\begin{frame}[fragile]\ft{\secname}
\begin{lstlisting}[backgroundcolor=\color{red!20}]
3x5 array
Sum of all elements = 80
2x6 array
Sum of all elements = 315
3x10 VLA
Sum of all elements = 270
\end{lstlisting}
\end{frame}

\begin{frame}[fragile]\ft{\secname}
注意,函数定义参量列表中的变长数组声明实际上并没有创建数组。和老语法一样,变长数组名实际上也是指针,也就是说带变长数组参量的函数实际上直接使用原数组,因此它有能力修改原数组。
\end{frame}

\begin{frame}[fragile]\ft{\secname}
以下程序段指出了指针和实际数组是如何声明的。
\begin{lstlisting}[language=c,backgroundcolor=\color{red!20}]
{
  int thing[10][6];
  twoset(10, 6, thing);
  ...
}

void twoset(int n, int m, int ar[n][m])
//`ar是一个指针,它指向由m个int组成的数组`
{
  int temp[n][m]; //`temp是一个nxm的int数组`
  temp[0][0] = 2; //`把temp的第一个元素设置为2`
  ar[0][0] = 2;   //`把thing[0][0]设置为2`
}
  
\end{lstlisting}

\end{frame}
