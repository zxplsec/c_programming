\section{保护数组内容}
\begin{frame}[fragile]\ft{\secname}
在编写处理诸如int这样的基本类型的函数时,可以向函数传递int数值,也可以传递指向int的指针。\vspace{0.05in}

\begin{itemize}
\item 通常是直接传递数值;\\[0.1in]
\item 只有需要在函数中修改该值时,才传递指针。
\end{itemize}
\end{frame}


\begin{frame}[fragile]\ft{\secname}
对于处理数组的函数,只能传递指针,原因是这样能使程序效率更高。\vspace{0.05in}

\begin{itemize}
\item 若通过值向函数传递数组,那么函数中必须分配足够存放一个原数组的拷贝的存储空间,然后把原数组的所有数据复制到这个新数组中;\\[0.1in]
\item 若简单地把数组的地址传递给函数,然后让函数直接读写原数组,程序效率会更高。
\end{itemize}
\end{frame}

\begin{frame}[fragile]\ft{\secname}
传值仅使用原始数据的一份拷贝,这可以保证原数组不会被意外修改;而传址使得函数可以直接操作原始数据,从而能修改原数组。
\end{frame}

\begin{frame}[fragile]\ft{\secname}
\begin{center}
需要修改原数组的例子
\end{center}
以下函数的功能是给数组的每个元素加上同一个数值。
\begin{lstlisting}[language=c,backgroundcolor=\color{red!20}]
void add_to(double arr[], int n, double val)
{
  int i;
  for (i = 0; i < n; i++)
    arr[i] += val;
}
\end{lstlisting} \pause 
该函数改变了数组的内容。之所以可以改变数组内容,是因为函数使用了指针,从而能够直接使用原始数据。
\end{frame}

\begin{frame}[fragile]\ft{\secname}
\begin{center}
不希望修改数据的例子
\end{center}
以下函数的功能是计算数组中所有元素的和,故该函数不希望数组的内容。但由于{\tf ar}实际上是一个指针,故编程的错误可能导致原始数据遭到破坏。
如表达式{\tf arr[i]++}会导致每个元素的值增加1。
\begin{lstlisting}[language=c,backgroundcolor=\color{red!20}]
void sum(int arr[], int n)
{
  int i;
  int sum = 0;
  for (i = 0; i < n; i++)
    sum += arr[i]++;
}
\end{lstlisting}  
\end{frame}

\begin{frame}[fragile]\ft{\secname:对形参使用const}
在ANSI C中,若设计意图是函数不改变数组的内容,那么可以在函数原型和定义中的形参声明中使用关键字{\tf const}。如
\begin{lstlisting}[language=c,backgroundcolor=\color{red!20}]
void sum(const int arr[], int n); // `原型`

void sum(const int arr[], int n)  // `定义`
{
  int i;
  int sum = 0;
  for (i = 0; i < n; i++)
    sum += arr[i];
}
\end{lstlisting} 
这将告知编译器:函数应该把{\tf arr}所指向的数组作为包含常量数据的数组看待。如果你意外的使用了诸如{\tf arr[i]++}之类的表达式,编译器将会发现这个错误并报告之,通知你函数试图修改常量。
\end{frame}

\begin{frame}[fragile]\ft{\secname:对形参使用const}
\begin{itemize}
\item 使用{\tf const}并不要求原始数组固定不变,只是说明函数在处理数组时,应把数组当做是常量数组。\\[0.1in]
\item 使用{\tf const}可以对数组提供保护,可阻止函数修改调用函数中的数据。\\[0.1in]
\item 如果函数想修改数组,那么在声明数组参量时就不要使用{\tf const};如果函数不需要修改数组,那么在声明数组参量时最好使用{\tf const}。
\end{itemize}
\end{frame}

\begin{frame}[fragile,allowframebreaks]\ft{\secname:对形参使用const}
\lstinputlisting
[language=c,numbers=left,frame=single]
{Code/arf.c}
\end{frame}


\begin{frame}[fragile]\ft{\secname:对形参使用const}
\begin{lstlisting}[backgroundcolor=\color{red!20}]
The original dip array:
  20.000   17.660    8.200   15.300   22.220 
The dip array after calling mult_array():
  50.000   44.150   20.500   38.250   55.550 
\end{lstlisting}  
\end{frame}

\begin{frame}[fragile]\ft{\secname:有关const的其它内容}
\begin{enumerate}
\item 使用{\tf const}创建符号常量:
\begin{lstlisting}[language=c,backgroundcolor=\color{red!20}]
const double PI = 3.1415926; 
\end{lstlisting}
也可用{\tf \#define}指令实现:
\begin{lstlisting}[language=c,backgroundcolor=\color{red!20}]
#define PI 3.1415926
\end{lstlisting}
\end{enumerate}
\end{frame}

\begin{frame}[fragile]\ft{\secname:有关const的其它内容}
\begin{enumerate}\setcounter{enumi}{1}
\item 使用{\tf const}还可创建数组常量、指针常量以及指向常量的指针。
\end{enumerate}
\end{frame}

\begin{frame}[fragile]\ft{\secname:有关const的其它内容}
以下代码使用{\tf const}保护数组
\begin{lstlisting}[language=c,backgroundcolor=\color{red!20}]
#define MONTHS 12
...
const int days[MONTHS] = {31,28,31,30,31,30,
                          31,31,30,31,30,31}; 
...
days[9] = 44;   //`编译错误`
\end{lstlisting}
\end{frame}

\begin{frame}[fragile]\ft{\secname:有关const的其它内容}
指向常量的指针不能用于修改数值。
\begin{lstlisting}[language=c,backgroundcolor=\color{red!20}]
double rates[4] = {8.9, 10.1, 9.4, 3.2};
const double * pd = rates;  //`pd指向数组开始处`
*pd = 29.89;                //`不允许`
pd[2] = 222.22;             //`不允许`
rates[0] = 99.99;           //`允许,因为rates不是常量`
pd++;                       //`允许,让pd指向rates[1]`
\end{lstlisting} 
\end{frame}

\begin{frame}[fragile]\ft{\secname:有关const的其它内容}
通常把指向常量的指针用作函数参量,以表明函数不会使用这个指针来修改数据。如
\begin{lstlisting}[language=c,backgroundcolor=\color{red!20}]
void show_array(const double * ar, int n);
\end{lstlisting} 
\end{frame}

\begin{frame}[fragile]\ft{\secname:有关const的其它内容}
关于指针赋值和{\tf const}的一些规则
\rule{\textwidth}{0.2mm} 
\begin{enumerate}[(a)]\setcounter{enumi}{0}
\item 允许将常量或非常量数据的地址赋给指向常量的指针。
\begin{lstlisting}[language=c,backgroundcolor=\color{red!20}]
double rates[4] = {8.9, 10.1, 9.4, 3.2};
const double locked[4] = {0.8, 0.7, 0.2, 0.3};
const double * pc = rates;  //`合法`
pc = locked;                //`合法`
pc = &rates[3];             //`合法`
\end{lstlisting} 
\end{enumerate}
\end{frame}

\begin{frame}[fragile]\ft{\secname:有关const的其它内容}
关于指针赋值和{\tf const}的一些规则
\rule{\textwidth}{0.2mm} 

\begin{enumerate}[(a)]\setcounter{enumi}{1}
\item 只有非常量数据的地址才可以赋给普通指针。
\begin{lstlisting}[language=c,backgroundcolor=\color{red!20}]
double rates[4] = {8.9, 10.1, 9.4, 3.2};
const double locked[4] = {0.8, 0.7, 0.2, 0.3};
double * pnc = rates;       //`合法`
pnc = locked;                //`非法`
pnc = &rates[3];             //`合法`
\end{lstlisting} 
这个规则是合理的,否则你就可以使用指针来修改被认为是常量的数据。
\end{enumerate}
\end{frame}

\begin{frame}[fragile]\ft{\secname:有关const的其它内容}
规则的应用
\rule{\textwidth}{0.2mm} 
像show\_array这样的函数,可以接受普通数组和常量数组的名称作为实参:
\begin{lstlisting}[language=c,backgroundcolor=\color{red!20}]
show_array(rates, 4);    //`合法`
show_array(locked, 4);    //`合法`
\end{lstlisting} \vspace{0.1in}

像mult\_array这样的函数,不能接受常量数组的名称作为参数:
\begin{lstlisting}[language=c,backgroundcolor=\color{red!20}]
mult_array(rates, 4);    //`合法`
mult_array(locked, 4);    //`不允许`
\end{lstlisting}
\vspace{0.1in}

因此,\textcolor{acolor1}{在函数参量定义中使用{\tf const},不仅可以保护数据,而且使函数可以使用声明为{\tf const}的数组。}
\end{frame}

\begin{frame}[fragile]\ft{\secname:有关const的其它内容}
\begin{enumerate}\setcounter{enumi}{2}
\item 使用{\tf const}来声明并初始化指针,以保证指针不会指向别处。
\begin{lstlisting}[language=c,backgroundcolor=\color{red!20}]
double rates[4] = {8.9, 10.1, 9.4, 3.2};
double const * pc = rates;  //`pc指向数组的开始处`
pc = &rates[3];             //`非法`
*pc = 2.2;                  //`合法,允许修改rates[0]的值`
\end{lstlisting}
这样的指针可用于修改数据,但它所指向的地址不能改变。
\end{enumerate}
\end{frame}


\begin{frame}[fragile]\ft{\secname:有关const的其它内容}
\begin{enumerate}\setcounter{enumi}{3}
\item 可使用两个{\tf const}来创建指针,该指针既不可以更改所指向的地址,也不可以修改所指向的数据。
\begin{lstlisting}[language=c,backgroundcolor=\color{red!20}]
double rates[4] = {8.9, 10.1, 9.4, 3.2};
const double const * pc = rates;   
pc = &rates[3];                   //`非法`
*pc = 2.2;                        //`非法`
\end{lstlisting}
\end{enumerate}
\end{frame}
