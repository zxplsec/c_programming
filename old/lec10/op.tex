% -*- coding: utf-8 -*-
% !TEX program = xelatex

\documentclass[10pt,notheorems]{beamer}

\usetheme[style=beta]{epyt} % alpha, beta, delta, gamma, zeta

\usepackage[UTF8,noindent]{ctex}
\usepackage{extarrows}
%\usepackage{courier}
\usepackage{animate}
\usepackage{dcolumn}
\usepackage{pgf}
\usepackage{tikz}
\usetikzlibrary{calc}
\usetikzlibrary{arrows,snakes,backgrounds,shapes,patterns}
\usetikzlibrary{matrix,fit,positioning,decorations.pathmorphing}
\usepackage{listings}
\lstset{
	frame=no,
	keywordstyle=\color{acolor1},  
	basicstyle=\ttfamily\small,
	commentstyle=\color{acolor5},
	breakindent=0pt,
	rulesepcolor=\color{red!20!green!20!blue!20},
	rulecolor=\color{black},
	tabsize=4,
	numbersep=5pt,
	numberstyle=\footnotesize,
	breaklines=true,
	%% backgroundcolor=\color{red!10},
	showspaces=false,
	showtabs=false,
	showstringspaces=false,
	extendedchars=false,
	escapeinside=``
}



\newcommand{\mylead}[1]{\textcolor{acolor1}{#1}}
\newcommand{\mybold}[1]{\textcolor{acolor2}{#1}}
\newcommand{\mywarn}[1]{\textcolor{acolor3}{#1}}

%%%% \renewcommand *****
%\renewcommand{\lstlistingname}{}
\newcommand{\tf}{\ttfamily}
%\newcommand{\ttt}{\texttt}
%\newcommand{\blue}{\textcolor{blue}}
%\newcommand{\red}{\textcolor{red}}
%\newcommand{\purple}{\textcolor{purple}}
\newcommand{\ft}{\frametitle}
\newcommand{\fst}{\framesubtitle}
\newcommand{\bs}{\boldsymbol}
\newcommand{\ds}{\displaystyle}
\newcommand{\vd}{\vdots}
\newcommand{\cd}{\cdots}
\newcommand{\dd}{\ddots}
\newcommand{\id}{\iddots}
\newcommand{\XX}{\mathbf{X}}
\newcommand{\PP}{\mathbf{P}}
\newcommand{\QQ}{\mathbf{Q}}
\newcommand{\xx}{\mathbf{x}}
\newcommand{\yy}{\mathbf{y}}
\newcommand{\bb}{\mathbf{b}}
\newcommand{\abd}{\boldsymbol{a}}

\renewcommand{\proofname}{证明}




%\newtheorem{theorem}{定理}
%\newtheorem{definition}[theorem]{定义}
%\newtheorem{example}[theorem]{例子}
%\newtheorem{dingli}[theorem]{定理}
%\newtheorem{li}[theorem]{例}
%
%\newtheorem*{theorem*}{定理}
%\newtheorem*{definition*}{定义}
%\newtheorem*{example*}{例子}
%\newtheorem*{dingli*}{定理}
%\newtheorem*{li*}{例}

\renewcommand{\proofname}{证明}
\newtheorem*{jie}{解}
\newtheorem*{zhu}{注}
\newtheorem*{dingli}{定理} 
\newtheorem*{dingyi}{定义} 
\newtheorem*{xingzhi}{性质} 
\newtheorem*{tuilun}{推论} 
\newtheorem{li}{例} 
\newtheorem*{jielun}{结论} 
\newtheorem*{zhengming}{证明}
\newtheorem*{wenti}{问题}
\newtheorem*{jieshi}{解释}
\newtheorem{biancheng}{编程}

\renewcommand{\proofname}{证明}

\begin{document}

\title{C语言}
\subtitle{第10次上机\\ 数组与指针}
\author{张晓平}
\institute{武汉大学数学与统计学院}


\begin{frame}[plain]\transboxout
\titlepage
\end{frame}

% \begin{frame}\transboxin
% \begin{center}
% \tableofcontents[]%hideallsubsections]
% \end{center}
% \end{frame}

% \AtBeginSection[]{
% \begin{frame}[allowframebreaks]
% \tableofcontents[currentsection,sectionstyle=show/hide]
% \end{frame}
% }
%\AtBeginSubsection[]{
%\begin{frame}[allowframebreaks]
%\tableofcontents[currentsection,currentsubsection,subsectionstyle=show/shaded/hide]
%\end{frame}
%}


\begin{frame}[fragile]
\begin{li} 
	编写一个函数,对数组按从小到大进行排序。简单排序算法原理:每次从左至右扫描序列,记下最小值的位置。
\end{li}
\end{frame}


\begin{frame}[fragile,allowframebreaks]
\lstinputlisting
[language=c,numbers=left,frame=single]
{Code_OP/ex10_00.c}
\end{frame}


\begin{frame}[fragile]
\begin{li} 
编写一个程序,初始化一个double数组,然后把数组内容复制到另外两个数组。制作第一份拷贝的函数使用数组符号。制作第二份拷贝的函数使用指针符号,并使用指针的增量操作。把目标数组名和要复制的元素个数作为参数传递给函数。
\begin{lstlisting}
double source[5] = {1.1, 2.2, 3.3, 4.4, 5.5};
double target1[5], target2[5];
copy_arr(source, target1, 5);
copy_ptr(source, target2, 5);
\end{lstlisting}
\end{li}
\end{frame}

\begin{frame}[fragile,allowframebreaks]
\lstinputlisting
[language=c,numbers=left,frame=single]
{Code_OP/ex10_02.c}
\end{frame}

 

\begin{frame}[fragile]
\begin{li} 
利用以上函数,把一个包含7个元素的数组中第3到第5个元素复制到一个包含3个元素的数组中。
\end{li}
\end{frame}

\begin{frame}[fragile,allowframebreaks]
\lstinputlisting
[language=c,numbers=left,frame=single]
{Code_OP/ex10_07.c}
\end{frame}


\begin{frame}[fragile]
\begin{li} 
编写一个函数,求一个double数组的最大值及其索引,并写一个简单驱动程序测试它。
\end{li}
\end{frame}

\begin{frame}[fragile,allowframebreaks]
\lstinputlisting
[language=c,numbers=left,frame=single]
{Code_OP/ex10_04.c}
\end{frame}

 

\begin{frame}[fragile]
\begin{li} 
编写一个函数,将两个长度相同的数组相加,结果存储到第三个数组中,并用一个简单的驱动程序测试它。
\end{li}
\end{frame}

\begin{frame}[fragile,allowframebreaks]
\lstinputlisting
[language=c,numbers=left,frame=single]
{Code_OP/ex10_09.c}
\end{frame}


 


\begin{frame}[fragile]
\begin{li} 
编写一个函数,求两个三维向量的内积和外积,并用一个简单的驱动程序测试它。
\end{li}
设
$$
\vec u = (a_1,a_2,a_3)^T, ~~~ 
\vec v = (b_1,b_2,b_3)^T
$$
则内积为
$$
\vec u \cdot \vec v = a_1b_1+a_2b_2+a_3b_3
$$
外积为
$$
\vec u \times \vec v = 
\left|
\begin{array}{ccc}
\mathbf i & \mathbf j & \mathbf k \\
a_1 & a_2 & a_3 \\
b_1 & b_2 & b_3
\end{array}
\right| = (a_2b_3-a_3b_2, a_3b_1-a_1b_3, a_1b_2-a_2b_1)^T.
$$
\end{frame}



\begin{frame}[fragile]
\begin{li} 
编写一个函数,提示用户输入三个数集,每个数集包括5个double值。程序应当实现以下功能:
\begin{enumerate}
\item 把输入信息存储到一个$3\times5$的数组中
\item 计算出每个数集的平均值
\item 计算所有数的平均值
\item 找出这15个数中的最大值
\item 打印出结果
\end{enumerate}
\end{li}
\end{frame}

\begin{frame}[fragile,allowframebreaks]
\lstinputlisting
[language=c,numbers=left,frame=single]
{Code_OP/ex10_12.h}
\end{frame}

\begin{frame}[fragile,allowframebreaks]
\lstinputlisting
[language=c,numbers=left,frame=single]
{Code_OP/ex10_12.c}
\end{frame}

  

\begin{frame}[fragile]
\begin{li} 
用变长数组重写以上程序。
\end{li}
\end{frame}

\begin{frame}[fragile]
\begin{li} 
用一维数组重写以上程序。
\end{li}
\end{frame}






\end{document}
