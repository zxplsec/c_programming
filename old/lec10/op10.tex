\subsection{上机操作}

\begin{frame}[fragile]\ft{\subsecname}
\begin{block}{例1}
编写一个函数,对数组按从小到大进行排序。简单排序算法原理:每次从左至右扫描序列,记下最小值的位置。
\end{block}
\end{frame}


\begin{frame}[fragile]\ft{\subsecname}
\lstinputlisting
[linerange={1-15}]
{Chapters/Ch10/Code/ex10_00.c}
\end{frame}

\begin{frame}[fragile]\ft{\subsecname}
\lstinputlisting
[linerange={17-25}]
{Chapters/Ch10/Code/ex10_00.c}
\end{frame}

\begin{frame}[fragile]\ft{\subsecname}
\lstinputlisting
[linerange={27-42}]
{Chapters/Ch10/Code/ex10_00.c}
\end{frame}

\begin{frame}[fragile]\ft{\subsecname}
\lstinputlisting
[linerange={45-52}]
{Chapters/Ch10/Code/ex10_00.c}
\end{frame}

\begin{frame}[fragile]\ft{\subsecname}
\begin{block}{例2}
编写一个程序,初始化一个double数组,然后把数组内容复制到另外两个数组。制作第一份拷贝的函数使用数组符号。制作第二份拷贝的函数使用指针符号,并使用指针的增量操作。把目标数组名和要复制的元素个数作为参数传递给函数。
\begin{lstlisting}
double source[5] = {1.1, 2.2, 3.3, 4.4, 5.5};
double target1[5], target2[5];
copy_arr(source, target1, 5);
copy_ptr(source, target2, 5);
\end{lstlisting}
\end{block}
\end{frame}

\begin{frame}[fragile]\ft{\subsecname}
\lstinputlisting
[linerange={1-5}]
{Chapters/Ch10/Code/ex10_02.c}
\end{frame}

\begin{frame}[fragile]\ft{\subsecname}
\lstinputlisting
[linerange={7-19}]
{Chapters/Ch10/Code/ex10_02.c}
\end{frame}

\begin{frame}[fragile]\ft{\subsecname}
\lstinputlisting
[linerange={21-28}]
{Chapters/Ch10/Code/ex10_02.c}
\end{frame}

\begin{frame}[fragile]\ft{\subsecname}
\lstinputlisting
[linerange={30-37}]
{Chapters/Ch10/Code/ex10_02.c}
\end{frame}

\begin{frame}[fragile]\ft{\subsecname}
\lstinputlisting
[linerange={39-46}]
{Chapters/Ch10/Code/ex10_02.c}
\end{frame}

\begin{frame}[fragile]\ft{\subsecname}
\begin{block}{例3}
利用以上函数,把一个包含7个元素的数组中第3到第5个元素复制到一个包含3个元素的数组中。
\end{block}
\end{frame}

\begin{frame}[fragile]\ft{\subsecname}
\lstinputlisting
[linerange={1-15}]
{Chapters/Ch10/Code/ex10_07.c}
\end{frame}


\begin{frame}[fragile]\ft{\subsecname}
\begin{block}{例4}
编写一个函数,求一个double数组的最大值及其索引,并写一个简单驱动程序测试它。
\end{block}
\end{frame}

\begin{frame}[fragile]\ft{\subsecname}
\lstinputlisting
[linerange={1-17}]
{Chapters/Ch10/Code/ex10_04.c}
\end{frame}

\begin{frame}[fragile]\ft{\subsecname}
\lstinputlisting
[linerange={18-32}]
{Chapters/Ch10/Code/ex10_04.c}
\end{frame}

\begin{frame}[fragile]\ft{\subsecname}
\begin{block}{例6}
编写一个函数,将两个长度相同的数组相加,结果存储到第三个数组中,并用一个简单的驱动程序测试它。
\end{block}
\end{frame}

\begin{frame}[fragile]\ft{\subsecname}
\lstinputlisting
[linerange={1-18}]
{Chapters/Ch10/Code/ex10_09.c}
\end{frame}



\begin{frame}[fragile]\ft{\subsecname}
\lstinputlisting
[linerange={19-33}]
{Chapters/Ch10/Code/ex10_09.c}
\end{frame}


\begin{frame}[fragile]\ft{\subsecname}
\begin{block}{例7}
编写一个函数,求两个三维向量的内积和外积,并用一个简单的驱动程序测试它。
\end{block}
设
$$
\vec u = (a_1,a_2,a_3)^T, ~~~ 
\vec v = (b_1,b_2,b_3)^T
$$
则内积为
$$
\vec u \cdot \vec v = a_1b_1+a_2b_2+a_3b_3
$$
外积为
$$
\vec u \times \vec v = 
\left|
\begin{array}{ccc}
\mathbf i & \mathbf j & \mathbf k \\
a_1 & a_2 & a_3 \\
b_1 & b_2 & b_3
\end{array}
\right| = (a_2b_3-a_3b_2, a_3b_1-a_1b_3, a_1b_2-a_2b_1)^T.
$$
\end{frame}



\begin{frame}[fragile]\ft{\subsecname}
\begin{block}{例7}
编写一个函数,提示用户输入三个数集,每个数集包括5个double值。程序应当实现以下功能:
\begin{enumerate}
\item 把输入信息存储到一个$3\times5$的数组中
\item 计算出每个数集的平均值
\item 计算所有数的平均值
\item 找出这15个数中的最大值
\item 打印出结果
\end{enumerate}
\end{block}
\end{frame}

\begin{frame}[fragile]\ft{\subsecname}
\lstinputlisting[title=ex10\_12.h]
{Chapters/Ch10/Code/ex10_12.h}
\end{frame}

\begin{frame}[fragile]\ft{\subsecname}
\lstinputlisting
[
title=ex10\_12.c,
linerange={1-14},
]
{Chapters/Ch10/Code/ex10_12.c}
\end{frame}

\begin{frame}[fragile]\ft{\subsecname}
\lstinputlisting
[
title=ex10\_12.c,
linerange={16-27},
]
{Chapters/Ch10/Code/ex10_12.c}
\end{frame}

\begin{frame}[fragile]\ft{\subsecname}
\lstinputlisting
[
title=ex10\_12.c,
linerange={29-35},
]
{Chapters/Ch10/Code/ex10_12.c}
\end{frame}

\begin{frame}[fragile]\ft{\subsecname}
\lstinputlisting
[
title=ex10\_12.c,
linerange={37-50},
]
{Chapters/Ch10/Code/ex10_12.c}
\end{frame}

\begin{frame}[fragile]\ft{\subsecname}
\lstinputlisting
[
title=ex10\_12.c,
linerange={52-61},
]
{Chapters/Ch10/Code/ex10_12.c}
\end{frame}

\begin{frame}[fragile]\ft{\subsecname}
\lstinputlisting
[
title=ex10\_12.c,
linerange={63-70},
]
{Chapters/Ch10/Code/ex10_12.c}
\end{frame}

\begin{frame}[fragile]\ft{\subsecname}
\lstinputlisting
[
title=ex10\_12.c,
linerange={72-82},
]
{Chapters/Ch10/Code/ex10_12.c}
\end{frame}

\begin{frame}[fragile]\ft{\subsecname}
\lstinputlisting
[
title=ex10\_12.c,
linerange={84-95},
]
{Chapters/Ch10/Code/ex10_12.c}
\end{frame}

\begin{frame}[fragile]\ft{\subsecname}
\begin{block}{例8}
用变长数组重写以上程序。
\end{block}
\end{frame}

\begin{frame}[fragile]\ft{\subsecname}
\begin{block}{例9}
用一维数组重写以上程序。
\end{block}
\end{frame}


