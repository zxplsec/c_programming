% -*- coding: utf-8 -*-
% !TEX program = xelatex

\documentclass[12pt,notheorems]{beamer}

\usetheme[style=beta]{epyt} % alpha, beta, delta, gamma, zeta

\usepackage[UTF8,noindent]{ctex}
\usepackage{etex}
\usepackage{pgf}
\usepackage{tikz}
\usetikzlibrary{calc}
\usetikzlibrary{arrows,snakes,backgrounds,shapes,shadows}
\usetikzlibrary{matrix,fit,positioning,decorations.pathmorphing}
\usepackage{CJK} 
\usepackage{amsmath,amssymb,amsfonts}
\usepackage{mathdots}
\usepackage{caption}
\usepackage{verbatim,color,xcolor}
\usepackage{graphicx}
\usepackage{manfnt}
\usepackage{fancybox}
\usepackage{textcomp}
\usepackage{multirow,multicol}
\usepackage{parcolumns}
\usepackage{framed}
\usepackage{threeparttable}
\usepackage{extarrows}
\usepackage{fourier} 
\usepackage{listings}
\lstset{
frame=no,
keywordstyle=\color{acolor1},  
basicstyle=\ttfamily\small,
commentstyle=\color{acolor5},
breakindent=0pt,
rulesepcolor=\color{red!20!green!20!blue!20},
rulecolor=\color{black},
tabsize=4,
numbersep=5pt,
numberstyle=\footnotesize,
breaklines=true,
%% backgroundcolor=\color{red!10},
showspaces=false,
showtabs=false,
showstringspaces=false,
extendedchars=false,
escapeinside=``
}
%% \usepackage[utf8]{inputenc}
%% \usepackage[upright]{fourier}   %


%\usepackage{xcolor}
%\usepackage{pgf}
%\usepackage{tikz}
%\usepackage{pgfplots}
%\usetikzlibrary{calc}
%\usetikzlibrary{arrows,snakes,backgrounds,shapes}
%\usetikzlibrary{matrix,fit,positioning,decorations.pathmorphing}
%\usepackage{CJK}               
%\usepackage[italian,american]{babel}
%\usepackage[applemac]{inputenc}
%\usepackage[T1]{fontenc}
%\usepackage{amsmath,amssymb,amsthm}
%\usepackage{varioref}
%\usepackage[style=philosophy-modern,hyperref,square,natbib]{biblatex}
%\usepackage{chngpage}
%\usepackage{calc}
%\usepackage{listings}
%\usepackage{graphicx}
%\usepackage{subfigure}
%\usepackage{multicol}
%\usepackage{makeidx}
%\usepackage{fixltx2e}
%\usepackage{relsize}
%\usepackage{lipsum}
%\usepackage{xifthen}
%%% \usepackage[eulerchapternumbers,subfig,beramono,eulermath,pdfspacing,listings]{classicthesis}
%%% \usepackage{arsclassica}        
%\usepackage{titlesec} %设置标题
%\usepackage{titletoc}
%\usepackage{extarrows}
%\usepackage{enumerate}
%% \usepackage[T1]{fontenc} % Needed for Type1 Concrete
%% %% \usepackage{concrete} % Loads Concrete + Euler VM
%% %% \usepackage{pxfonts} % Or palatino or mathpazo
%% \usepackage{eulervm} %
%% %% \usepackage{kerkis} % Kerkis roman and sans
%% %% \usepackage{kmath} % Kerkis math
%% \usepackage{fourier}
\usepackage{courier}
\usepackage{animate}
\usepackage{dcolumn}



\newcommand{\mylead}[1]{\textcolor{acolor1}{#1}}
\newcommand{\mybold}[1]{\textcolor{acolor2}{#1}}
\newcommand{\mywarn}[1]{\textcolor{acolor3}{#1}}

%%%% \renewcommand *****
%\renewcommand{\lstlistingname}{}
\newcommand{\tf}{\ttfamily}
%\newcommand{\ttt}{\texttt}
%\newcommand{\blue}{\textcolor{blue}}
%\newcommand{\red}{\textcolor{red}}
%\newcommand{\purple}{\textcolor{purple}}
\newcommand{\ft}{\frametitle}
\newcommand{\fst}{\framesubtitle}
\newcommand{\bs}{\boldsymbol}
\newcommand{\ds}{\displaystyle}
\newcommand{\vd}{\vdots}
\newcommand{\cd}{\cdots}
\newcommand{\dd}{\ddots}
\newcommand{\id}{\iddots}
\newcommand{\XX}{\mathbf{X}}
\newcommand{\PP}{\mathbf{P}}
\newcommand{\QQ}{\mathbf{Q}}
\newcommand{\xx}{\mathbf{x}}
\newcommand{\yy}{\mathbf{y}}
\newcommand{\bb}{\mathbf{b}}
\newcommand{\abd}{\boldsymbol{a}}

\renewcommand{\proofname}{证明}




%\newtheorem{theorem}{定理}
%\newtheorem{definition}[theorem]{定义}
%\newtheorem{example}[theorem]{例子}
%\newtheorem{dingli}[theorem]{定理}
%\newtheorem{li}[theorem]{例}
%
%\newtheorem*{theorem*}{定理}
%\newtheorem*{definition*}{定义}
%\newtheorem*{example*}{例子}
%\newtheorem*{dingli*}{定理}
%\newtheorem*{li*}{例}

\renewcommand{\proofname}{证明}
\newtheorem*{jie}{解}
\newtheorem*{zhu}{注}
\newtheorem*{dingli}{定理} 
\newtheorem*{dingyi}{定义} 
\newtheorem*{xingzhi}{性质} 
\newtheorem*{tuilun}{推论} 
\newtheorem*{li}{例} 
\newtheorem*{jielun}{结论} 
\newtheorem*{zhengming}{证明}
\newtheorem*{wenti}{问题}
\newtheorem*{jieshi}{解释}


\renewcommand{\proofname}{证明}

\begin{document}

\title{C语言}
\subtitle{第九讲、函数}
\author{张晓平}
\institute{武汉大学数学与统计学院}


\begin{frame}[plain]\transboxout
\titlepage
\end{frame}

\begin{frame}[allowframebreaks]\transboxin
\begin{center}
\tableofcontents[hideallsubsections]
\end{center}
\end{frame}

%% \AtBeginSection[]{
%% \begin{frame}[allowframebreaks]
%% \tableofcontents[currentsection]%,sectionstyle=show/hide]
%% \end{frame}
%% }
%\AtBeginSubsection[]{
%\begin{frame}[allowframebreaks]
%\tableofcontents[currentsection,currentsubsection,subsectionstyle=show/shaded/hide]
%\end{frame}
%}

\section{函数概述}

\begin{frame}\ft{为什么使用函数?}
\begin{itemize}
\item 使用函数可以减少代码的重复。若程序需要多次使用某种特定的功能,只需编写一个合适的函数,然后程序可以在任何需要的地方调用该函数。\\[0.2in]
\item 即使某种功能在程序中只使用一次,将其以函数的形式实现也有必要,因为函数使得程序更加模块化,从而有利于程序的阅读、修改和完善。
\end{itemize}
\end{frame}

\begin{frame}\ft{为什么使用函数?}
假设你想编写一个程序,以实现如下功能:\vspace{0.1in}

\begin{itemize}
\item  读入一行数字 \\[0.1in]
\item  对数字进行排序 \\[0.1in]
\item  求他们的平均值 \\[0.1in]
\item  打印出一个柱状图
\end{itemize}
\end{frame}

\begin{frame}[fragile]\ft{为什么使用函数?}
\begin{lstlisting}[language=c,numbers=left,frame=single]
#include <stdio.h>
#define SIZE 50
int main(void)
{
  float list[SIZE];  
  readlist(list, SIZE);
  sort(list, SIZE);
  average(list, SIZE);
  bargragh(list, SIZE);  
  return 0;
}
\end{lstlisting}
\end{frame}

\begin{frame}[fragile]\ft{为什么使用函数?}
如何实现这四个函数需要你自行完成。描述性的函数名可以清楚地表明程序的功能和组织结构,然后对每个函数进行独立设计,若这些函数足够通用化,则可以在其他程序中调用它们。
\end{frame}

\begin{frame}[fragile]\ft{为什么使用函数?}
\begin{itemize}
\item
函数可看做是一个“黑盒子”,你只需关注函数的功能及使用方法,而其内部行为你无需考虑,除非你是该函数的编写者。\\[0.1in]
\item
如我们在使用{\tf printf()}时,只需输入一个控制字符串,或者还有其它一些参数,就可以预测{\tf printf()}的执行结果,而无须了解{\tf printf()}内部的代码。
\\[0.1in]
\item
以这种方式看待函数,有助于集中精力投入到程序的整体设计而不是实现细节。
\end{itemize}
\end{frame}

\begin{frame}[fragile]\ft{对于函数需要了解些什么?}
\begin{itemize}
\item 如何正确定义函数\\[0.1in]
\item 如何调用函数\\[0.1in]
\item 如何建立函数间的通信
\end{itemize}

\end{frame}

\begin{frame}[fragile]\ft{一个简单的例子}
请打印一个简单的信头:
\begin{lstlisting}[backgroundcolor=\color{red!10}]
****************************************
Wuhan University
299 Bayi Road Wuchang District,
Wuhan, PR China 430072
****************************************
\end{lstlisting}
\end{frame}

\begin{frame}[fragile,allowframebreaks]\ft{一个简单的例子}
\lstinputlisting
[language=c,numbers=left,frame=single]
{Code/letterhead1.c}
\end{frame}


\begin{frame}[fragile]\ft{程序分析}
{\tf starbar}在不同位置出现了三次:\vspace{0.1in}

\begin{itemize}
\item 函数原型{\tf (function prototype)}:告知编译器starbar的函数类型\\[0.1in]
\item 函数调用{\tf (function call)}:使函数执行\\[0.1in]
\item 函数定义{\tf (function definition)}:实现函数的具体功能
\end{itemize}
\end{frame}

\begin{frame}[fragile]\ft{程序分析}

函数同变量一样有多种类型。 
函数在被使用之前都要声明其类型,故{\tf main()}之前出现了代码
\begin{lstlisting}[backgroundcolor=\color{red!10}]
void starbar(void);
\end{lstlisting}
\begin{itemize}
\item
圆括号表明{\tf starbar}是一个函数名。\\[0.1in]
\item 
第一个{\tf void}指的是函数类型,表明该函数没有返回值。\\[0.1in]
\item 
第二个{\tf void}表明该函数不接受任何参数。\\[0.1in]
\item 
分号表示该语句是进行函数声明,而不是函数定义。 
\end{itemize}
\end{frame}

\begin{frame}[fragile]\ft{程序分析}
函数原型也可以放在main函数内变量声明的任何位置,故以下两种写法都正确:
\begin{lstlisting}[language=c,backgroundcolor=\color{red!10}]
...
void starbar(void);
int main(void)
{
  ...
}
\end{lstlisting}

\begin{lstlisting}[language=c,backgroundcolor=\color{red!10}]
...
int main(void)
{
  void starbar(void);
}
\end{lstlisting}

\end{frame}

\begin{frame}[fragile]\ft{程序分析}
程序在{\tf main()}中通过使用以下方式调用{\tf starbar()}:
\begin{lstlisting}
starbar();
\end{lstlisting}
\begin{itemize}
\item
当程序执行到该语句时,它找到{\tf starbar()}并执行其中的指令。\\[0.1in]
\item 
执行完{\tf starbar()}中的代码后,程序将返回到调用函数{\tf (calling function)}的下一条语句继续执行。 
\end{itemize}
\end{frame}

\begin{frame}[fragile]\ft{程序分析}
\begin{itemize}
\item
程序中,{\tf starbar()}和{\tf main()}有相同的定义格式,即首先以类型、名称和圆括号开始,接着是开始花括号、变量声明、函数语句定义以及结束花括号。\\[0.1in]
\item 
注意此处的{\tf starbar()}后跟花括号,告诉编译器这是在定义函数,而不是调用它或声明其原型。
\end{itemize}

\end{frame}

\begin{frame}[fragile]\ft{程序分析}
\begin{itemize}
\item
该程序中,{\tf starbar()}和{\tf main()}在同一个文件中,也可以将它们放在不同文件中。\\[0.1in]
\item 
单文件形式比较容易编译,而使用多个文件则有利于在不同的程序中使用相同的函数。\\[0.1in]
\item 
若使用多文件形式,则每个文件中都必须包含{\tf \#define}和{\tf \#include}指令。
\end{itemize}

\end{frame}

\begin{frame}[fragile]\ft{程序分析}
\begin{itemize}
\item
{\tf starbar()}中的变量{\tf count}是一个局部变量,这意味着该变量只在{\tf starbar()}中可用。\\[0.1in]
\item 
即使你在其它函数中使用名称{\tf count},也不会出现任何冲突。
\end{itemize}

\end{frame}

\begin{frame}[fragile]\ft{函数参数}
改写以上程序,让信头的文字居中,形如
\begin{lstlisting}
****************************************
            Wuhan University
    299 Bayi Road, Wuchang District,
         Wuhan, PR China 430072
****************************************
\end{lstlisting}
\end{frame}

\begin{frame}[fragile]\ft{如何做到?}
假设一行是40个字符宽度。\vspace{0.1in} \pause 

\begin{enumerate}
\item
打印一行星号很容易做到,直接输出40个星号即可。\\[0.1in] \pause 
\item 
如何让Wuhan University居中呢?。
\\[0.1in] \pause 
\item[]
在输出文字之前输出若干空格即可。\\[0.1in] \pause 
\item 
那到底输出多少个空格呢?。
\\[0.1in] \pause 
\item[]
设文字长度为$l$,则一行中除文字外还需$40-l$个空格。想要文字居中,左边应输出$(40-l)/2$个空格。
\end{enumerate}

\end{frame}

\begin{frame}[fragile,allowframebreaks]\ft{程序实现}
\lstinputlisting
[language=c,numbers=left,frame=single]
{Code/letterhead2.c}
\end{frame}


\begin{frame}[fragile]\ft{定义带参数的函数(形式参数,简称“形参”)}
\begin{lstlisting}[language=c,backgroundcolor=\color{red!10},title=函数头]
  void show_n_char(char ch, int num)
\end{lstlisting}
\begin{itemize}
\item
这行代码告诉编译器,{\tf show\_n\_char()}使用了两个参数{\tf ch}和{\tf num},它们的类型分别为{\tf char}和{\tf int}。\\[0.1in]
\item
变量{\tf ch}和{\tf num}被称为形式参数{\tf (formal argument)}或形式参量{\tf (formal parameter)}。\\[0.1in]
\item 形式参量是局部变量,为函数所私有,这意味着可以在其它函数中使用相同的变量名。\\[0.1in]
\item 调用函数时,形式参量会被赋值。
\end{itemize}
\end{frame}

\begin{frame}[fragile]\ft{定义带参数的函数(形式参数,简称“形参”)}
必须在每个形参前声明其类型,不能像通常的变量声明那样使用变量列表来声明同一类型的变量。比如
\begin{lstlisting}[language=c,backgroundcolor=\color{red!10}]
void func1(int x, y, z)  // wrong 
void func2(int x, int y, int z)  // right
\end{lstlisting}

\end{frame}

\begin{frame}[fragile]\ft{定义带参数的函数(形式参数,简称“形参”)}
古老的函数定义方式1:
\begin{lstlisting}[language=c,backgroundcolor=\color{red!10}]
void show_n_char(ch, num)
char ch;
int num;
{
  ...
}
\end{lstlisting}
\end{frame}

\begin{frame}[fragile]\ft{定义带参数的函数(形式参数,简称“形参”)}
古老的函数定义方式2:
\begin{lstlisting}[language=c,backgroundcolor=\color{red!10}]
void func1(x, y, z)
int x, y, z;
{
  ... 
}
\end{lstlisting}
\end{frame}

\begin{frame}[fragile]\ft{带参数函数的声明}
\begin{itemize}
\item
使用函数之前需要用ANSI原型声明该函数
\begin{lstlisting}[language=c,backgroundcolor=\color{red!10}]
void show_n_char(char ch, int num);
\end{lstlisting}
\vspace{0.1in}

\item
当函数接受参数时,函数原型通过使用一个逗号分隔的类型列表指明参数的个数和类型。在函数原型中可根据你的喜好省略变量名:
\begin{lstlisting}[language=c,backgroundcolor=\color{red!10}]
void show_n_char(char, int);
\end{lstlisting}
\vspace{0.1in}

\item
在原型中使用变量名并没有实际地创建变量。
\end{itemize}

\end{frame}

\begin{frame}[fragile]\ft{带参数函数的声明}
ANSI C也支持旧的函数声明形式,即圆括号内不带任何参数:
\begin{lstlisting}[language=c,backgroundcolor=\color{red!10}]
void show_n_char();
\end{lstlisting}
该方式请不要使用。了解该形式的主要原因只是为了让你能正确识别并理解以前的代码。
\end{frame}

\begin{frame}[fragile]\ft{调用带参数的函数:实际参数,简称“实参”}
函数调用中,通过使用实际参数{\tf (actual argument)}对{\tf ch}和{\tf num}赋值。
\begin{itemize}
\item
第一次调用中
\begin{lstlisting}[language=c,backgroundcolor=\color{red!10}]
show_n_char(SPACE, 12);
\end{lstlisting}
实参是空格字符和{\tf 12},它们被赋给{\tf show\_n\_char()}中相应的形参:{\tf ch}和{\tf num}。\\[0.1in]
\item \textcolor{acolor1}{实参可以是常量、变量或一个复杂的表达式。}\\[0.1in]
\item 但无论何种形式的实参,执行时首先要计算其值,然后将该值赋值给被调函数中相应的形参。
\end{itemize}
\end{frame}

\begin{frame}[fragile]\ft{调用带参数的函数:实际参数,简称“实参”}
实参赋值给形参,被调函数使用的值是从调用函数中复制而来的,故不管在被调函数中对赋值数值进行了什么操作,调用函数中的原数值不受影响。
\end{frame}

\begin{frame}[fragile]\ft{使用return从函数中返回一个值}
\begin{itemize}
\item
将实参赋值给形参,实现了从调用函数到被调函数的通信。\\[0.1in]
\item
而想从被调函数往调用函数传递信息,可以使用函数返回值。
\end{itemize}

\end{frame}

\begin{frame}[fragile]\ft{使用return从函数中返回一个值}
  \begin{wenti}
    编写函数,比较两个整数的大小,并返回较小值。同时编制一个驱动程序来测试该函数。
  \end{wenti}
\end{frame}

\begin{frame}[fragile,allowframebreaks]\ft{使用return从函数中返回一个值}
\lstinputlisting
[language=c,numbers=left,frame=single]
{Code/lesser.c}
\end{frame}

\begin{frame}[fragile]\ft{使用return从函数中返回一个值}

\begin{lstlisting}[backgroundcolor=\color{red!10}]
Enter two integers (q to quit):
509 333
The lesser of 509 and 333 is 333.
Enter two integers (q to quit):
-9333 6
The lesser of -9333 and 6 is -9333.
Enter two of integers (q to quit):
q
Bye.
\end{lstlisting}
\end{frame}

\begin{frame}[fragile]\ft{使用return从函数中返回一个值}
\begin{itemize}
\item
关键字{\tf return}指明了其后的表达式的值即为该函数的返回值。\\[0.1in]
\item 
{\tf imin()}中的变量{\tf min}是其私有的,但{\tf return}语句将它的值返回给了调用函数。\\[0.1in]
\item 
语句
\begin{lstlisting}[language=c,backgroundcolor=\color{red!10}]
lesser = imin(m, n);
\end{lstlisting}
相当于把min的值赋给了lesser。
\\[0.1in]
\item
能否这么写?
\begin{lstlisting}[language=c,backgroundcolor=\color{red!10}]
imin(m, n);
lesser = min;
\end{lstlisting} \pause
{\Huge 当然不行的啦!!!}
\end{itemize}
\end{frame}

\begin{frame}[fragile]\ft{使用return从函数中返回一个值}
返回值不仅可以被赋给一个变量,也可以被用作表达式的一部分。如
\begin{lstlisting}[language=c,backgroundcolor=\color{red!10}]
answer = 2*imin(m, n) + 5;
printf("%d\n", imin(answer+2, LIMIT));
\end{lstlisting}

\end{frame}

\begin{frame}[fragile]\ft{使用return从函数中返回一个值}
返回值可以由任何表达式计算而得到,而不仅仅来自于一个变量。如imin函数可以改写为
\begin{lstlisting}[language=c,backgroundcolor=\color{red!10}]
int imin(int n,int m)
{  
  return ((n < m) ? n : m);
}
\end{lstlisting}
这里并不要求使用圆括号,但如果想让程序更清晰,可以把添上一个圆括号。
\end{frame}

\begin{frame}[fragile]\ft{使用return从函数中返回一个值}
观察以下代码:
\begin{lstlisting}[language=c,backgroundcolor=\color{red!10}]
int what_if(int n)
{  
  double z = 100.0 / (double) n;
  return z;
}
\end{lstlisting}
这里,返回值的类型和声明的类型不一致,{\Large What will happen?}
\pause 
\vspace{0.1in}

\textcolor{acolor1}{将把{\tf doule}型变量{\tf z}的值强制转换为{\tf int}型。}

\end{frame}

\begin{frame}[fragile]\ft{使用return从函数中返回一个值}
{\tf return}的另一个作用是终止函数的执行,并把控制返回给调用函数的下一条语句,即使{\tf return}语句不在函数尾部。如{\tf imin()}可以写成
\begin{lstlisting}[language=c,backgroundcolor=\color{red!10}]
int imin(int n, int m)
{  
  if (n < m)
    return n;
  else
    return m;  
  printf("Oh my god!\n");    
}
\end{lstlisting}
{\tf return}语句使得{\tf printf}语句永远不会执行。
\end{frame}

\begin{frame}[fragile]\ft{使用return从函数中返回一个值}
也可以使用语句
\begin{lstlisting}[language=c,backgroundcolor=\color{red!10}]
return;
\end{lstlisting}
该语句会终止执行函数,并把控制返回给调用函数。此时,{\tf return}后没有任何表达式,故没有返回值,该形式只能用于{\tf void}类型的函数。
\end{frame}

\begin{frame}[fragile]\ft{函数类型}
\begin{itemize}
\item
函数应该进行类型声明,同时其类型应和返回值类型相同。\\[0.1in]
\item
无返回值的函数应该被声明为{\tf void}类型。\\[0.1in]
\item
类型声明是函数定义的一部分,该类型指的是返回值类型。如函数头
\begin{lstlisting}[language=c,backgroundcolor=\color{red!10}]
double klink(int a, int b)
\end{lstlisting}
表示函数使用两个{\tf int}型的参数,而返回值类型为{\tf double}。
\end{itemize}
\end{frame}

\begin{frame}[fragile]\ft{函数类型}
为正确使用函数,程序在首次调用函数之前需要知道该函数的类型。
\begin{itemize}
\item 方式一:
调用之前给出完整的函数定义。\\[0.1in]
\begin{lstlisting}[language=c,backgroundcolor=\color{red!10}]
int imin(int n, int m)
{
  ... 
}

int main(void)
{
  ...
  n = imin(n1, n2);
  ...
}
\end{lstlisting}
\end{itemize}
\end{frame}

\begin{frame}[fragile]\ft{函数类型}
\begin{itemize}
\item 方式二:
对函数进行声明,以便将函数信息通知编译器。\\[0.1in]
\begin{lstlisting}[language=c,backgroundcolor=\color{red!10}]
int imin(int, int);

int main(void)
{
  int n1, n2, lesser;
  ...
  n = imin(n1, n2);
  ...
}

int imin(int n, int m)
{
  ... 
}
\end{lstlisting}
\end{itemize}
\end{frame}

\begin{frame}[fragile]\ft{函数类型}
也可将函数声明放在调用函数内部。 
\begin{lstlisting}[language=c,backgroundcolor=\color{red!10}]
int main(void)
{
  int imin(int, int);
  int n1, n2, lesser;
  ...
  n = imin(n1, n2);
  ...
}

int imin(int n, int m)
{
  ... 
}
\end{lstlisting}
\end{frame}

\begin{frame}[fragile]\ft{函数类型}
在ANSI C标准库中,函数被分为几个系列,每一系列都有各自的头文件,这些头文件中包含了本系列函数的声明部分。
\end{frame}

\begin{frame}[fragile]\ft{函数类型}
  \begin{lstlisting}[language=c,backgroundcolor=\color{red!10}]
// stdio.h 
int getchar();
int putchar(int c);
int printf(const char *format , ... );
int scanf (const char *format , ... );
\end{lstlisting}
\end{frame}

\begin{frame}[fragile]\ft{函数类型}
  \begin{lstlisting}[language=c,backgroundcolor=\color{red!10}]
// math.h
double sin(double);   
double cos(double);   
double tan(double);   
double asin(double);  
double acos(double); 
double atan(double); 
double log(double);  
double log10(double); 
double pow(double x,double y); 
double exp(double); 
double sqrt(double); 
int abs(int);  
double fabs(double); 
\end{lstlisting}
\end{frame}
                  
%\section{多维数组}

\begin{frame}[fragile]\ft{\secname} 
编制程序,计算出年降水总量、年降水平均量,以及月降水平均量。
\end{frame}

\begin{frame}[fragile]\ft{\secname} 
\lstinputlisting
[
basicstyle=\footnotesize\ttfamily,
linerange={3-18}
]
{Chapters/Ch10/Code/rain.c}
\end{frame}

\section{递归}

\begin{frame}[fragile]\ft{\secname}
C允许一个函数调用其自身,这种调用过程被称为递归(recursion)。 
\vspace{0.1in}

\begin{itemize}
\item 递归一般可用循环代替。有些情况使用循环会比较好,而有时使用递归更有效。\\[0.1in]
\item 递归虽然可使程序结构优美,但其执行效率却没循环语句高。
\end{itemize}
\end{frame}


\begin{frame}[fragile,allowframebreaks]\ft{\secname}
  \lstinputlisting
  [language=c,numbers=left,frame=single]
  {Code/recur.c}
\end{frame}


\begin{frame}[fragile]\ft{\secname}
\begin{lstlisting}[backgroundcolor=\color{red!10}]
Level 1: n location 0x7fff5fbff7bc
Level 2: n location 0x7fff5fbff79c
Level 3: n location 0x7fff5fbff77c
Level 4: n location 0x7fff5fbff75c
LEVEL 4: n location 0x7fff5fbff75c
LEVEL 3: n location 0x7fff5fbff77c
LEVEL 2: n location 0x7fff5fbff79c
LEVEL 1: n location 0x7fff5fbff7bc
\end{lstlisting}
\end{frame}


\begin{frame}[fragile]\ft{\secname}
{\tf \&}为地址运算符,{\tf \&n}表示存储n的内存地址,{\tf printf()}使用占位符{\tf \%p}来指示地址。
\end{frame}


\begin{frame}[fragile]\ft{\secname:程序分析}
\begin{itemize}
\item 首先,{\tf main()}使用实参1调用{\tf up\_and\_down()},打印语句{\tf \#1}输出{\tf Level 1}。\\[0.1in]
\item 然后,由于$n<4$,故{\tf up\_and\_down()}(第1级)使用实参2调用{\tf up\_and\_down()}(第2级),打印语句{\tf \#1}输出{\tf Level 2}。\\[0.1in]
\item 类似地,下面的两次调用打印{\tf Level 3}和{\tf Level 4}。
\end{itemize}
\end{frame}

\begin{frame}[fragile]\ft{\secname:程序分析}
\begin{itemize}
\item 当开始执行第4级调用时,$n$的值为4,故if语句不满足条件,不再继续调用{\tf up\_and\_down()},接着执行打印语句{\tf \#2},输出{\tf Level 4},至此第4级调用结束,把控制返回给第3级调用函数。\\[0.1in]
\item 第3级调用函数中前一个执行过的语句是在if语句中执行第4级调用,因此,它开始执行后续代码,即执行打印语句{\tf \#2},输出{\tf Level 3}。\\[0.1in]
\item 当第3级调用结束后,第2级调用函数开始继续执行,输出{\tf Level 2}。以此类推。
\end{itemize}
\end{frame}

\begin{frame}[fragile]\ft{\secname:递归的基本原理}
\begin{itemize}
\item
\textcolor{acolor1}{每一级的递归都使用其私有变量n。}\\[0.1in]
\item
每一次函数调用都会有一次返回。当程序执行到某一级递归的结尾处时,它会转移到前一级递归继续执行。
\end{itemize}
\end{frame}

\begin{frame}[fragile]\ft{\secname:递归的基本原理}
\begin{itemize}
\item
递归函数中,位于递归调用前的语句和各级被调函数具有相同的执行次序。\\[0.1in]
\item[] 如打印语句\#1位于递归调用语句之前,它按递归调用的顺序执行4次,即依次为第1级、第2级、第3级和第4级。\\[0.1in]
\item 
递归函数中,位于递归调用后的语句和各级被调函数具有相反的执行次序。\\[0.1in]
\item[] 
如打印语句{\tf \#2}位于递归调用语句之后,执行次序为:第4级、第3级、第2级和第1级。
\end{itemize}
\end{frame}

\begin{frame}[fragile]\ft{\secname:递归的基本原理}
\begin{itemize}
\item 递归函数中,必须包含可以终止递归调用的语句。
\end{itemize}

\end{frame}

\begin{frame}[fragile]\ft{\secname:尾递归}
最简单的递归方式是\textcolor{acolor1}{把递归调用语句放在函数结尾,return语句之前。}这种形式被称为\textcolor{acolor1}{尾递归(tail recursion)}。尾递归的作用相当于一条循环语句,它是最简单的递归形式。
\end{frame}

\begin{frame}[fragile]\ft{\secname:尾递归}
分别使用循环和尾递归编写函数计算阶乘,然后用一个驱动程序测试它们。
\end{frame}

\begin{frame}[fragile,allowframebreaks]\ft{\secname:尾递归}
  \lstinputlisting
  [language=c,numbers=left,frame=single]
  {Code/factor.c}
\end{frame}


\begin{frame}[fragile]\ft{\secname:尾递归}
\begin{lstlisting}
This program calculates factorials.
Enter a value in the range 0-12 (q to quit):
5
loop:      5! = 120
recursion: 5! = 120
Enter a value in the range 0-12 (q to quit):
10
loop:      10! = 3628800
recursion: 10! = 3628800
Enter a value in the range 0-12 (q to quit):
12
loop:      12! = 479001600
recursion: 12! = 479001600
Enter a value in the range 0-12 (q to quit):
q
Bye.
\end{lstlisting}
\end{frame}

\begin{frame}[fragile]\ft{\secname:尾递归}
{\Large 选用循环还是递归?}\pause 一般来说,选择循环更好一些。
\pause
\vspace{0.1in}

\begin{itemize}
\item 每次递归调用都有自己的变量集合,需要占用较多的内存。每次递归调用需要把新的变量集合存储在堆栈中。\\[0.1in]
\item 每次函数调用都要花费一定的时间,故递归的执行速度较慢。
\end{itemize}
\end{frame}

\begin{frame}[fragile]\ft{\secname:尾递归}
{\Large 那为什么要学习递归呢?}\pause 
\vspace{0.1in}

\begin{itemize}
\item 尾递归非常简单,易于理解。\\[0.1in]
\item 某些情况下,不能使用简单的循环语句代替递归,所以有必要学习递归。
\end{itemize}

\end{frame}

\begin{frame}[fragile]\ft{\secname:递归与反向计算}
编写程序,将一个整数转换为二进制形式。
\end{frame}

\begin{frame}[fragile]\ft{\secname:递归与反向计算}
对于奇数,其二进制形式的末位为1;而对于偶数,其二进制形式的末位为0。于是,\textcolor{acolor1}{对于n,其二进制数的末位为n\%2。}
\begin{lstlisting}[backgroundcolor=\color{red!10}]
628
628%10=8  628/10=62  62%10=2   62/10=6   6%10=6
     8                     2                  6
\end{lstlisting}

\begin{lstlisting}[backgroundcolor=\color{red!10}]
5  
 5%2=1  5/2=2  2%2=0  2/2=1  1%2=1
     1             0             1 
10
10%2=0  10/2=5  5%2=1  5/2=2  2%2=0  
     0              1             0
2/2=1  1%2=1
           1
\end{lstlisting}

\end{frame}

\begin{frame}[fragile]\ft{\secname:递归与反向计算}
规律:
\vspace{0.1in}

\begin{itemize}
\item 
在递归调用之前,计算{\tf n\%2}的值,在递归调用之后输出。\\[0.1in]
\item
为算下一个数字,需把原数值除以2。若此时得出的为偶数,则下一个二进制位为0;若得出的是奇数,则下一个二进制位为1。
\end{itemize}
\end{frame}

\begin{frame}[fragile,allowframebreaks]\ft{\secname:递归与反向计算}
\lstinputlisting
  [language=c,numbers=left,frame=single]
  {Code/binary.c}
\end{frame}



\begin{frame}[fragile]\ft{\secname:递归与反向计算}
\begin{lstlisting}[backgroundcolor=\color{red!10}]
Enter an integer (q to quit):
9
Binary equivalent: 1001
Enter an integer (q to quit):
255
Binary equivalent: 11111111
Enter an integer (q to quit):
1024
Binary equivalent: 10000000000
Enter an integer (q to quit):
q
Done.
\end{lstlisting}
\end{frame}


\begin{frame}[fragile]\ft{\secname:递归的优缺点}
 

\begin{itemize}
\item 优点:
\item[] 
为某些编程问题提供了最简单的解决办法。\\[0.1in]
\item 缺点:
\item[]
一些递归算法会很快地耗尽计算机的内存资源,同时递归程序难于阅读和维护。
\end{itemize}

\end{frame}


\begin{frame}[fragile]\ft{\secname:递归的优缺点}
编写程序,计算斐波那契数列。
$$
\begin{aligned}
&F_1=F_2=1,\\[0.1in]
&F_n=F_{n-1}+F_{n-2}, \quad n=3,4,\cd.
\end{aligned}
$$
\end{frame}


\begin{frame}[fragile]\ft{\secname:递归的优缺点}
\begin{lstlisting}
long Fibonacci(int n)
{
  if (n > 2)
    return Fibonacci(n-1) + Fibonacci(n-2);
  else
    return 1;
}
\end{lstlisting}

该函数使用了双重递归(double recursion),即函数对本身进行了两次调用。这会导致一个弱点。{\Huge What?}
\end{frame}


\begin{frame}[fragile]\ft{\secname:递归的优缺点}
每级调用的变量数会呈指数级增长:
\begin{table}
\centering
\caption{每级调用中变量n的个数}
\begin{tabular}{cc}\hline
Level & number of n\\\hline
$1$ & $1$\\
$2$ & $2$\\
$3$ & $2^2$\\
$4$ & $2^3$\\
$\vd$ & $\vd$ \\
$l$ & $2^{l-1}$\\\hline
\end{tabular}
\end{table}
\end{frame}

\subsection{重定向与文件}



\begin{frame}[fragile]\ft{\subsecname}
\end{frame}
%  \section{地址运算符:\&}
\begin{frame}[fragile]\ft{\secname}
C最重要的、也是最复杂的一个概念是指针(pointer),即\textcolor{acolor1}{用来存储地址的变量}。

\end{frame}

\begin{frame}[fragile]\ft{\secname}
\begin{itemize}
\item
scanf函数使用地址作为参数。\\[0.1in]
\item
更一般地,若想在无返回值的被调函数中修改调用函数的某个数据,必须使用地址参数。
\end{itemize}
\end{frame}

\begin{frame}[fragile]\ft{\secname}
\&为单目运算符,可以取得变量的存储地址。 
\pause 
\vspace{0.4in}

设var为一个变量,则\&var为该变量的地址。

\vspace{0.1in}

\textcolor{acolor1}{一个变量的地址就是该变量在内存中的地址。} 

\end{frame}

\begin{frame}[fragile]\ft{\secname}
设有如下语句
\begin{lstlisting}
var = 24;
\end{lstlisting}
并假定var的存储地址为07BC,则执行语句
\begin{lstlisting}
printf("%d %p\n", var, &var);
\end{lstlisting}
的结果为
\begin{lstlisting}
24 07BC
\end{lstlisting}
\end{frame}

\begin{frame}[fragile,allowframebreaks]\ft{\secname}
  \lstinputlisting
  [language=c,numbers=left,frame=single]
  {Code/loccheck.c}
\end{frame}


\begin{frame}[fragile]\ft{\secname}
\begin{lstlisting}[backgroundcolor=\color{red!10}]
In main(), var1 =  2 and &var1 = 0x7fff5fbff7d8
In main(), var2 =  5 and &var2 = 0x7fff5fbff7d4
In func(), var1 = 10 and &var1 = 0x7fff5fbff7a8
In func(), var2 =  5 and &var2 = 0x7fff5fbff7ac
\end{lstlisting}
\end{frame}

\begin{frame}[fragile]\ft{\secname}
\begin{itemize}
\item 两个{\tf var1}变量具有不同的地址,两个{\tf var2}变量也是如此。\\[0.1in]
\item 调用{\tf func}函数时,把实参({\tf main}函数中的{\tf var2})的值5传递给了形参({\tf func}函数中的{\tf var2})。但这种传递只是进行了数值传递,两个变量仍是独立的。
\end{itemize}
\end{frame}



\section{改变调用函数中的变量}
\begin{frame}[fragile]\ft{\secname}
有些时候,我们需要用一个函数改变另一个函数的变量。如排序问题中,一个常见的任务是交换两个变量的值。
\end{frame}

\begin{frame}[fragile]\ft{\secname}
以下代码能否交换变量x和y的值{\Large ?}
\begin{lstlisting}[backgroundcolor=\color{red!10}]
x = y;
y = x;
\end{lstlisting}
\pause \vspace{0.1in}

\begin{center}
{\Large NO!}
\end{center}
\pause\vspace{0.1in}

\begin{center}
{\Large Why?} 
\end{center}
\end{frame}

\begin{frame}[fragile]\ft{\secname}
那以下代码能否交换变量x和y的值{\Large ?}
\begin{lstlisting}[backgroundcolor=\color{red!10}]
temp = y;
x = y;
y = temp;
\end{lstlisting}
\pause \vspace{0.1in}

\begin{center}
{\Large OK!}
\end{center}

\end{frame}

\begin{frame}[fragile,allowframebreaks]\ft{\secname}
  \lstinputlisting
  [language=c,numbers=left,frame=single]
  {Code/swap1.c}
\end{frame}


\begin{frame}[fragile]\ft{\secname}
\begin{lstlisting}[backgroundcolor=\color{red!10}]
Originally: x = 5, y = 10.
Now       : x = 5, y = 10.
\end{lstlisting}
\pause \vspace{0.1in}

\begin{center}
{\Large Why not interchanged?}
\end{center}


\end{frame}

\begin{frame}[fragile,allowframebreaks]\ft{\secname}
  \lstinputlisting
  [language=c,numbers=left,frame=single]
  {Code/swap2.c}
\end{frame}

\begin{frame}[fragile]\ft{\secname}
\begin{lstlisting}[backgroundcolor=\color{red!10}]
Originally: x =  5, y = 10.
Originally: u =  5, v = 10.
Now       : u = 10, v =  5.
Now       : x =  5, y = 10.
\end{lstlisting}
\pause \vspace{0.1in}

\begin{itemize}
\item
函数interchange中,u和v的值确实得到了交换。问题出在了把执行结果传递给函数main的时候。\\[0.1in]
\item
函数interchange中的变量独立于函数main,因此交换u和v的值对x和y的值没有任何影响。
\end{itemize}

\end{frame}

\begin{frame}[fragile]\ft{\secname}
能否使用return?如
\begin{lstlisting}[backgroundcolor=\color{red!10}]
int main(void)
{
  ...
  x = interchange(x, y);
  ...
}
int interchange(int u, int v)
{
  int temp;
  temp = u;
  u = v;
  v = temp;
  return u;
}
\end{lstlisting}
\end{frame}

\begin{frame}[fragile]\ft{\secname}
此时,x的值得以更新,但y的值仍未做改变。因为\textcolor{acolor1}{return语句只能把一个数值传递给调用函数},而现在却需要传递两个数值。
\pause \vspace{0.1in}

\begin{center}
{\Large 怎么办?}
\end{center}
\pause\vspace{0.1in}

\begin{center}
{\Large 用指针!} 
\end{center}
\end{frame}

 \section{指针}
\begin{frame}[fragile]\ft{\secname}
指针是一个变量,其值为一个地址。
\end{frame}

\begin{frame}[fragile]\ft{\secname}
假如你把某个指针变量命名为{\tf ptr},就可以使用以下语句
\begin{lstlisting}[backgroundcolor=\color{blue!10}]
ptr = &var;
\end{lstlisting}
即把变量{\tf var}的地址赋给指针变量{\tf ptr},称为\textcolor{acolor1}{\tf ptr“指向”var}。
\pause \vspace{0.1in}

{\tf ptr}和{\tf \&var}的区别在于,前者为一变量,后者是一个常量。
\end{frame}

\begin{frame}[fragile]\ft{\secname}
{\tf ptr}可以指向任何地址,即可以把任何地址赋值给{\tf ptr}:
\begin{lstlisting}[backgroundcolor=\color{blue!10}]
ptr = &var1;
\end{lstlisting}
\end{frame}

\begin{frame}[fragile]\ft{\secname}
  \begin{wenti}
    如何创建一个指针变量?
  \end{wenti}
  \pause \vskip.1in
  
  首先需要声明其类型。在介绍其类型之前,我们先介绍一个新运算符{\tf *}。
\end{frame}

\begin{frame}[fragile]\ft{\secname:间接运算符或取值运算符:*}
假定{\tf ptr}指向{\tf var},即
\begin{lstlisting}[backgroundcolor=\color{blue!10}]
ptr = &var;
\end{lstlisting}
就可以用间接运算符{\tf *}来获取{\tf var}中存放的数值:
\begin{lstlisting}[backgroundcolor=\color{blue!10}]
value = *ptr;
\end{lstlisting}
\pause \vspace{0.1in}

\begin{minipage}{0.4\textwidth}
\begin{lstlisting}[backgroundcolor=\color{blue!10}]
ptr = &var;
value = *ptr;
\end{lstlisting}
\end{minipage}
~~~$\Longleftrightarrow$~~~
\begin{minipage}{0.4\textwidth}
\begin{lstlisting}[backgroundcolor=\color{blue!10}]
value = var;
\end{lstlisting}
\end{minipage}
\end{frame}

\begin{frame}[fragile]\ft{\secname:指针声明}
能否如以下方式声明一个指针?
\begin{lstlisting}[backgroundcolor=\color{blue!10}]
pointer ptr;
\end{lstlisting}
\pause \vspace{0.1in}

\begin{center}
{\Large NO!}
\end{center}
\pause\vspace{0.1in}

\begin{center}
{\Large Why?} 
\end{center}
\end{frame}

\begin{frame}[fragile]\ft{\secname:指针声明}
原因在于,仅声明一个变量为指针是不够的,还需说明指针所指向变量的类型。
\vspace{0.1in}

\begin{itemize}
\item 不同的变量类型占用的存储空间大小不同,而有些指针需要知道变量类型所占用的存储空间。\\[0.1in]
\item 程序也需要知道地址中存储的是何种数据。
\end{itemize}
\end{frame}

\begin{frame}[fragile]\ft{\secname:指针声明}
  \begin{lstlisting}[backgroundcolor=\color{blue!10}]
// `正确的指针声明方式`    
int * pi;           // pi`是指向一个整型变量的指针`
char * pc;          // pc`是指向一个字符变量的指针`
float * pf, * pg;   // pf`和`pg`是指向浮点变量的指针`
\end{lstlisting} \pause \vspace{0.1in}

\begin{itemize}
\item
类型标识符表明了被指向变量的类型,{\tf *}表示该变量为一个指针。\\[0.1in]
\item 
声明{\tf int * pi;}的含义是:{\tf pi}是一个指针,且{\tf *pi}是{\tf int}类型的。\\[0.1in]
\item 
{\tf *}与指针名之间的空格可选。通常在声明中使用空格,在指向变量时将其省略。
\end{itemize}
\end{frame}

\begin{frame}[fragile]\ft{\secname:指针声明}
\begin{itemize}
\item
{\tf pc}所指向的值{\tf (*pc)}是{\tf char}类型的,而{\tf pc}本身是“指向{\tf char}的指针”类型。\\[0.1in]
\item 
{\tf pc}的值是一个地址,在大多数系统中,它由一个无符号整数表示。但这并不表示可以把指针看做是整数类型。\\[0.1in]
\item 
一些处理整数的方法不能用来处理指针,反之亦然。如两个整数可以相乘,但指针不能。\\[0.1in]
\item 
指针是一种新的数据类型,而不是一种整数类型。
\end{itemize}
\end{frame}

\begin{frame}[fragile]\ft{\secname:使用指针在函数间通信}
这里将重点介绍如何通过指针解决函数间的通信问题。
\end{frame}

\begin{frame}[fragile,allowframebreaks]\ft{\secname:使用指针在函数间通信}
  \lstinputlisting
  [language=c,numbers=left,frame=single]  
  {Code/swap3.c}
\end{frame}

\begin{frame}[fragile]\ft{\secname}
\begin{lstlisting}[backgroundcolor=\color{blue!10}]
Originally: x =  5, y = 10.
Now       : x = 10, y =  5.
\end{lstlisting}
\pause \vspace{0.1in}

\begin{center}
{\Large Oh Ye!!!}
\end{center}
\end{frame}

\begin{frame}[fragile]\ft{\secname:使用指针在函数间通信}
\begin{itemize}
\item 
函数调用语句为
\begin{lstlisting}[backgroundcolor=\color{blue!10}]
interchange(&x, &y);
\end{lstlisting}
故函数传递的是{\tf x}和{\tf y}的地址而不是它们的值。\\[0.15in]
\item 
函数声明为
\begin{lstlisting}[backgroundcolor=\color{blue!10}]
void interchange(int * u, int * v);
\end{lstlisting}
也可简化为
\begin{lstlisting}[backgroundcolor=\color{blue!10}]
void interchange(int *, int *);
\end{lstlisting}
\end{itemize}
\end{frame}

\begin{frame}[fragile]\ft{\secname:使用指针在函数间通信}
\begin{itemize}
\item
函数体中声明了一个临时变量
\begin{lstlisting}[backgroundcolor=\color{blue!10}]
int temp;
\end{lstlisting}
\item
为了把{\tf x}的值存在{\tf temp}中,需使用以下语句
\begin{lstlisting}[backgroundcolor=\color{blue!10}]
temp = *u; 
\end{lstlisting}
因{\tf u}的值为{\tf \&x},即{\tf x}的地址,故{\tf *u}代表了{\tf x}的值。\\[0.1in]
\item 
同理,为了把{\tf y}的值赋给{\tf x},需用以下语句
\begin{lstlisting}[backgroundcolor=\color{blue!10}]
*u = *v;
\end{lstlisting}
\end{itemize}
\end{frame}

\begin{frame}[fragile]\ft{\secname:使用指针在函数间通信}
该例中,用一个函数实现了{\tf x}和{\tf y}的数值交换。\vspace{0.1in}

\begin{itemize}
\item
首先函数使用{\tf x}和{\tf y}的地址作为参数,这使得它可以访问{\tf x}和{\tf y}变量。\\[0.1in]
\item 
通过使用指针和运算符{\tf *},函数可以获得相应存储地址的数据,从而就可以改变这些数据。
\end{itemize}
\end{frame}

\begin{frame}[fragile]\ft{\secname:使用指针在函数间通信}
通常情况下,可以把变量的两类信息传递给一个函数,即传值与传址。
\end{frame}

\begin{frame}[fragile]\ft{\secname:传值}
\begin{itemize}
\item 调用方式为
\begin{lstlisting}[backgroundcolor=\color{blue!10}]
function1(x);
\end{lstlisting}
\item 定义方式为
\begin{lstlisting}[backgroundcolor=\color{blue!10}]
int function1(int num)
\end{lstlisting}
\item 适用范围:使用函数进行数据计算等操作。
\end{itemize}

\end{frame}

\begin{frame}[fragile]\ft{\secname:传址}
\begin{itemize}
\item 调用方式为
\begin{lstlisting}[backgroundcolor=\color{blue!10}]
function2(&x);
\end{lstlisting}
\item 定义方式为
\begin{lstlisting}[backgroundcolor=\color{blue!10}]
int function2(int * ptr)
\end{lstlisting}
\item 适用范围:改变调用函数中的多个变量的值。
\end{itemize}

\end{frame}




\end{document}
