% -*- coding: utf-8 -*-
% !TEX program = xelatex

\documentclass[12pt,notheorems]{beamer}

\usetheme[style=beta]{epyt} % alpha, beta, delta, gamma, zeta

\usepackage[UTF8,noindent]{ctex}
\usepackage{etex}
\usepackage{pgf}
\usepackage{tikz}
\usetikzlibrary{calc}
\usetikzlibrary{arrows,snakes,backgrounds,shapes,shadows}
\usetikzlibrary{matrix,fit,positioning,decorations.pathmorphing}
\usepackage{CJK} 
\usepackage{amsmath,amssymb,amsfonts}
\usepackage{mathdots}
\usepackage{caption}
\usepackage{verbatim,color,xcolor}
\usepackage{graphicx}
\usepackage{manfnt}
\usepackage{fancybox}
\usepackage{textcomp}
\usepackage{multirow,multicol}
\usepackage{parcolumns}
\usepackage{framed}
\usepackage{threeparttable}
\usepackage{extarrows}
\usepackage{fourier} 
\usepackage{listings}
\lstset{
frame=no,
keywordstyle=\color{acolor1},  
basicstyle=\ttfamily\small,
commentstyle=\color{acolor5},
breakindent=0pt,
rulesepcolor=\color{red!20!green!20!blue!20},
rulecolor=\color{black},
tabsize=4,
numbersep=5pt,
numberstyle=\footnotesize,
breaklines=true,
%% backgroundcolor=\color{red!10},
showspaces=false,
showtabs=false,
showstringspaces=false,
extendedchars=false,
escapeinside=``
}
%% \usepackage[utf8]{inputenc}
%% \usepackage[upright]{fourier}   %


%\usepackage{xcolor}
%\usepackage{pgf}
%\usepackage{tikz}
%\usepackage{pgfplots}
%\usetikzlibrary{calc}
%\usetikzlibrary{arrows,snakes,backgrounds,shapes}
%\usetikzlibrary{matrix,fit,positioning,decorations.pathmorphing}
%\usepackage{CJK}               
%\usepackage[italian,american]{babel}
%\usepackage[applemac]{inputenc}
%\usepackage[T1]{fontenc}
%\usepackage{amsmath,amssymb,amsthm}
%\usepackage{varioref}
%\usepackage[style=philosophy-modern,hyperref,square,natbib]{biblatex}
%\usepackage{chngpage}
%\usepackage{calc}
%\usepackage{listings}
%\usepackage{graphicx}
%\usepackage{subfigure}
%\usepackage{multicol}
%\usepackage{makeidx}
%\usepackage{fixltx2e}
%\usepackage{relsize}
%\usepackage{lipsum}
%\usepackage{xifthen}
%%% \usepackage[eulerchapternumbers,subfig,beramono,eulermath,pdfspacing,listings]{classicthesis}
%%% \usepackage{arsclassica}        
%\usepackage{titlesec} %设置标题
%\usepackage{titletoc}
%\usepackage{extarrows}
%\usepackage{enumerate}
%% \usepackage[T1]{fontenc} % Needed for Type1 Concrete
%% %% \usepackage{concrete} % Loads Concrete + Euler VM
%% %% \usepackage{pxfonts} % Or palatino or mathpazo
%% \usepackage{eulervm} %
%% %% \usepackage{kerkis} % Kerkis roman and sans
%% %% \usepackage{kmath} % Kerkis math
%% \usepackage{fourier}
\usepackage{courier}
\usepackage{animate}
\usepackage{dcolumn}



\newcommand{\mylead}[1]{\textcolor{acolor1}{#1}}
\newcommand{\mybold}[1]{\textcolor{acolor2}{#1}}
\newcommand{\mywarn}[1]{\textcolor{acolor3}{#1}}

%%%% \renewcommand *****
%\renewcommand{\lstlistingname}{}
\newcommand{\tf}{\ttfamily}
%\newcommand{\ttt}{\texttt}
%\newcommand{\blue}{\textcolor{blue}}
%\newcommand{\red}{\textcolor{red}}
%\newcommand{\purple}{\textcolor{purple}}
\newcommand{\ft}{\frametitle}
\newcommand{\fst}{\framesubtitle}
\newcommand{\bs}{\boldsymbol}
\newcommand{\ds}{\displaystyle}
\newcommand{\vd}{\vdots}
\newcommand{\cd}{\cdots}
\newcommand{\dd}{\ddots}
\newcommand{\id}{\iddots}
\newcommand{\XX}{\mathbf{X}}
\newcommand{\PP}{\mathbf{P}}
\newcommand{\QQ}{\mathbf{Q}}
\newcommand{\xx}{\mathbf{x}}
\newcommand{\yy}{\mathbf{y}}
\newcommand{\bb}{\mathbf{b}}
\newcommand{\abd}{\boldsymbol{a}}

\renewcommand{\proofname}{证明}




%\newtheorem{theorem}{定理}
%\newtheorem{definition}[theorem]{定义}
%\newtheorem{example}[theorem]{例子}
%\newtheorem{dingli}[theorem]{定理}
%\newtheorem{li}[theorem]{例}
%
%\newtheorem*{theorem*}{定理}
%\newtheorem*{definition*}{定义}
%\newtheorem*{example*}{例子}
%\newtheorem*{dingli*}{定理}
%\newtheorem*{li*}{例}

\renewcommand{\proofname}{证明}
\newtheorem*{jie}{解}
\newtheorem*{zhu}{注}
\newtheorem*{dingli}{定理} 
\newtheorem*{dingyi}{定义} 
\newtheorem*{xingzhi}{性质} 
\newtheorem*{tuilun}{推论} 
\newtheorem{li}{例} 
\newtheorem*{jielun}{结论} 
\newtheorem*{zhengming}{证明}
\newtheorem*{wenti}{问题}
\newtheorem*{jieshi}{解释}
\newtheorem{biancheng}{编程}

\renewcommand{\proofname}{证明}

\begin{document}

\title{C语言}
\subtitle{第八次上机\\ 字符输入输出}
\author{张晓平}
\institute{武汉大学数学与统计学院}


\begin{frame}[plain]\transboxout
\titlepage
\end{frame}

% \begin{frame}\transboxin
% \begin{center}
% \tableofcontents[]%hideallsubsections]
% \end{center}
% \end{frame}

% \AtBeginSection[]{
% \begin{frame}[allowframebreaks]
% \tableofcontents[currentsection,sectionstyle=show/hide]
% \end{frame}
% }
%\AtBeginSubsection[]{
%\begin{frame}[allowframebreaks]
%\tableofcontents[currentsection,currentsubsection,subsectionstyle=show/shaded/hide]
%\end{frame}
%}


\begin{frame}[fragile]
\begin{li}
编写一个程序,把输入作为字符流读取,直至遇到EOF。令其报告输入中的大写字母个数和小写字母个数。
\end{li}
\end{frame}

\begin{frame}[fragile]
\begin{li}
改写猜数程序:猜1-100中的某个数字z,按二分法进行。
\begin{itemize}
\item 
假设你最初猜50,让其询问z是大于、小于还是等于猜测值。
\item
若小于50,则令下一次猜测值为50和100的平均值75。
\item
若大于75,则令下一次猜测值为50和75的平均值62。
\end{itemize}
\end{li}
\end{frame}

\begin{frame}[fragile]
\begin{li}
编写一个程序,显示一个菜单,提供加法、减法、乘法或除法的选项。获取选择后,该程序请求两个数,然后执行选择的操作。
\begin{itemize}
\item 
该程序应该只接受所提供的菜单选项,应使用float类型的数,并且如果用户未能输入数字应允许其重新输入。
\item
在除法的情况下,如果用户输入0作为第二个数,该程序应该提示用户输入一个新的值。
\end{itemize}
\end{li}
\end{frame}

\begin{frame}[fragile]
\begin{lstlisting}
Enter the operation of your choice:
a. add         b. substract
c. multiply    d. divide
q. quit
a
Enter first number: 22.4
Enter second number: one
one is not a number.
Please enter a number,
such as 2.5, -1.78E8, or 3: 1
22.40 + 1.00 = 23.40.
\end{lstlisting}
\end{frame}

\begin{frame}[fragile]
\begin{lstlisting}[backgroundcolor=\color{blue!20}]
Enter the operation of your choice:
a. add         b. substract
c. multiply    d. divide
q. quit
d
Enter first number: 1
Enter second number: 0
Enter a number other than 0: 0.2
1.00 / 0.20 = 5.00.
Enter the operation of your choice:
a. add         b. substract
c. multiply    d. divide
q. quit
q
Bye.
\end{lstlisting}
\end{frame}



\end{document}
