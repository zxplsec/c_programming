\documentclass[10pt,a4paper%,twoside,openright,titlepage,fleqn,%
%headinclude,footinclude,BCOR5mm,%
%numbers=noenddot,cleardoublepage=empty,%
tablecaptionabove]{article}

\usepackage{geometry}
\geometry{left=2.5cm,right=2.5cm,top=2.5cm,bottom=2.5cm}

\usepackage{amsmath,amssymb,amsthm}

%% -----------------设置数学公式字体-------------------------
%% Font style 1
%% \newcommand\ibinom[2]{\genfrac\lbrace\rbrace{0pt}{}{#1}{#2}}
%% \usepackage{bm}

%% Font style 2
%% \newcommand\ibinom[2]{\genfrac\lbrace\rbrace{0pt}{}{#1}{#2}} 
%% \usepackage[boldsans]{ccfonts} 
%% \usepackage{bm} 

%% Font style 3
\newcommand\ibinom[2]{\genfrac\lbrace\rbrace{0pt}{}{#1}{#2}}
\usepackage[euler-digits]{eulervm}
\usepackage{bm}

%% Font style 4
%% \usepackage{fourier}
%% \newcommand\ibinom[2]{\genfrac\lbrace\rbrace{0pt}{}{#1}{#2}}
%% \usepackage{bm}

%% Font style 5
%% \newcommand\ibinom[2]{\genfrac\lbrace\rbrace{0pt}{}{#1}{#2}}
%% \usepackage{mathptmx}
%% \usepackage{bm} 


%% %% Font style 6
%% \newcommand\ibinom[2]{\genfrac\lbrace\rbrace{0pt}{}{#1}{#2}}
%% \usepackage{txfonts}
%% \usepackage{bm}
%% -----------------------------------------------------------

\usepackage{titlesec} %设置标题
\usepackage{titletoc}

\usepackage{extarrows}
\usepackage{verbatim,color,xcolor}
\usepackage{pgf}
\usepackage{tikz}
\usetikzlibrary{calc}
\usetikzlibrary{arrows,snakes,backgrounds,shapes,patterns}
\usetikzlibrary{matrix,fit,positioning,decorations.pathmorphing}
%% \usepackage{classicthesis}
\usepackage{CJK}
\usepackage{mathdots}

\usepackage{listings}
\lstset{
  keywordstyle=\color{blue!70},
  frame=single,
  basicstyle=\ttfamily\small,
  commentstyle=\small\color{red},
  breakindent=0pt,
  rulesepcolor=\color{red!20!green!20!blue!20},
  rulecolor=\color{black},
  tabsize=4,
  numbersep=5pt,
  breaklines=true,
  %% backgroundcolor=\color{red!10},
  showspaces=false,
  showtabs=false,
  extendedchars=false,
  escapeinside=``,
  frame=no,
}


\newcommand{\blue}{\textcolor{blue}}
\newcommand{\red}{\textcolor{red}}
\newcommand{\purple}{\textcolor{electricpurple}}
\newcommand{\ds}{\displaystyle}
\newcommand{\cd}{\cdots}
\newcommand{\dd}{\ddots}
\newcommand{\vd}{\vdots}
\newcommand{\id}{\iddots}

\newcommand{\R}{\mathbb R}
\def\nn{{\boldsymbol{n}}}
\def\xx{{\boldsymbol{x}}}
\def\F{{\boldsymbol{F}}}
\def\div{{\mathrm{div}}}
\def\tf{\ttfamily}


\begin{document}

\begin{CJK}{UTF8}{gkai}
 

\newtheorem{li}{例}
\newtheorem{jielun}{结论}
\newtheorem{dingli}{定理}
\newtheorem{mingti}{{命题}} 
\newtheorem{yinli}{{引理}} 
\newtheorem{tuilun}{{推论}}
\newtheorem{dingyi}{{定义}} 
\newtheorem{example}{{例}}
\newtheorem*{example*}{{例}}
\newtheorem*{jie}{{解}}
\newtheorem*{zhengming}{{证明}}
\newtheorem{zhu}{{注}}
\newtheorem*{zhu*}{{注}}
\newtheorem{xingzhi}{{性质}}
\newtheorem{wenti}{{问题}}
\newtheorem{rem}{{Remark}}
\newtheorem{lem}{{Lemma}}
\pagenumbering{roman}
\pagestyle{plain}

\pagenumbering{arabic}

\title{C与C++中的逗号}
%\author{张晓平}
%\date{}                                           % Activate to display a given date or no date
\maketitle

在C 与 C++中,逗号有两层含义:
\begin{enumerate}
\item 逗号充当运算符。
\item[] 逗号运算符为一元运算符,先计算第一个操作数并舍弃之,然后计算第二个操作数并返回该值。逗号运算符具有最低优先级,并且是一个顺序点。
  \begin{lstlisting}[language=c,backgroundcolor=\color{red!10}]
/* comma as an operator */
int i = (5, 10);  /* 10 is assigned to i*/
int j = (f1(), f2());  /* f1() is called (evaluated) first followed by f2(). 
                      The returned value of f2() is assigned to j */    
  \end{lstlisting}
\item 逗号充当分隔符
\item[] 逗号作为分隔符,通常用于函数调用与定义,函数宏,变量声明,enum声明以及结构体中。
  \begin{lstlisting}[language=c,backgroundcolor=\color{red!10}]
/* comma as a separator */
int a = 1, b = 2;
void fun(x, y);    
  \end{lstlisting}
\end{enumerate}


%% The use of comma as a separator should not be confused with the use as an operator. For example, in below statement, f1() and f2() can be called in any order.
\begin{lstlisting}[language=c,backgroundcolor=\color{red!10}]
/* Comma acts as a separator here and doesn't enforce any sequence. 
    Therefore, either f1() or f2() can be called first */
void fun(f1(), f2());  
\end{lstlisting}

%% You can try below programs to check your understanding of comma in C.
\lstinputlisting[language=c,backgroundcolor=\color{red!10}]{Code/comma1.c}
\lstinputlisting[language=c,backgroundcolor=\color{red!10}]{Code/comma2.c}
\lstinputlisting[language=c,backgroundcolor=\color{red!10}]{Code/comma3.c}

\end{CJK}
\end{document}
