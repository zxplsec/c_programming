\subsection{字符串函数}

\begin{frame}[fragile]\ft{\subsecname} 
C库提供了许多处理字符串的函数,在头文件string.h中给出其函数原型。
\begin{enumerate}
\item strlen 
\item strcat 
\item strcmp 
\item strncmp 
\item strcpy 
\item strncpy
\end{enumerate}
\end{frame}


\begin{frame}[fragile]\ft{\subsecname:strlen函数} 
\begin{lstlisting}[title=strlen函数原型, basicstyle=\ttfamily]
int * strlen(const char * s);
\end{lstlisting}
\rule{\textwidth}{0.3mm} \vspace{0.3mm}

\blue{功能:}返回字符串s的长度,即s中的字符数(不包含空字符)。
\end{frame}

\begin{frame}[fragile]\ft{\subsecname:strlen函数}
\lstinputlisting
[basicstyle=\ttfamily,
 linerange={1-16}]
{Chapters/Ch11/Code/test_fit.c}
\end{frame}

\begin{frame}[fragile]\ft{\subsecname:strlen函数}
\lstinputlisting
[
 basicstyle=\ttfamily,
 linerange={17-21}
]
{Chapters/Ch11/Code/test_fit.c}
\end{frame}

\begin{frame}[fragile]\ft{\subsecname:strlen函数}
\begin{lstlisting}[basicstyle=\ttfamily]
Hold on to your hats, hackers.
Hold on
Let's look at some more of the string.
to your hats, hackers.
\end{lstlisting}
\end{frame}

\begin{frame}[fragile]\ft{\subsecname:strcat函数} 
\begin{lstlisting}[title=strcat函数原型, basicstyle=\ttfamily]
char * strcat(char * s1, const char * s2);
\end{lstlisting}
\rule{\textwidth}{0.3mm} \vspace{0.3mm}

\begin{itemize}
\item
\blue{功能:}把字符串s2(包括空字符)追加到字符串s1的结尾,字符串s2的第一个字符覆盖字符串s1中的空字符。\\[0.1in]
\item
s1成为一个新的字符串,s2没有改变。\\[0.1in]
\item 
strcat函数为char *类型,返回s1。
\end{itemize}
\end{frame}

\begin{frame}[fragile]\ft{\subsecname:strcat函数}
\lstinputlisting
[title=str\_cat.c,linerange={1-16}]
{Chapters/Ch11/Code/str_cat.c}
\end{frame}

\begin{frame}[fragile]\ft{\subsecname:strcat函数}
\begin{lstlisting}[basicstyle=\ttfamily]
What is your favorite flower?
Rose
Roses smell like old shoes.
s smell like old shoes.
\end{lstlisting}
\end{frame}

\begin{frame}[fragile]\ft{\subsecname:strncat函数} 
strcat并不检查第一个数组是否能够容纳第二个字符串。如果没有为第一个数组分配足够大的空间,多出的字符溢出到相邻单元时就会出现问题。\vspace{.1in}

该问题可通过使用strncat函数加以解决,该函数需要另一个参数来指明最多允许添加的字符数目。


\end{frame}

\begin{frame}[fragile]\ft{\subsecname:strncat函数}
\begin{lstlisting}[title=strncat函数原型, basicstyle=\ttfamily]
char * strcat(char * s1, const char * s2, size_t n);
\end{lstlisting}
\rule{\textwidth}{0.3mm} \vspace{0.3mm}

\begin{itemize}
\item 
\blue{功能:}把字符串s2的前n个字符或直到空字符为止的字符追加到字符串s1的结尾,s2的第一个字符覆盖s1中的空字符,总在最后添加一个空字符。\\[0.1in]
\item
strncat函数为char *类型,返回s1。
\end{itemize}
\end{frame}

\begin{frame}[fragile]\ft{\subsecname:strncat函数}
\lstinputlisting
[title=join\_chk.c,linerange={1-10}]
{Chapters/Ch11/Code/join_chk.c}
\end{frame}

\begin{frame}[fragile]\ft{\subsecname:strncat函数}
\lstinputlisting
[title=join\_chk.c,linerange={12-16}]
{Chapters/Ch11/Code/join_chk.c}
\end{frame}

\begin{frame}[fragile]\ft{\subsecname:strncat函数}
\lstinputlisting
[title=join\_chk.c,linerange={18-25}]
{Chapters/Ch11/Code/join_chk.c}
\end{frame}

\begin{frame}[fragile]\ft{\subsecname:strncat函数}
\begin{lstlisting}[basicstyle=\ttfamily]
What is your favorite flower?
Rose
Roses smell like old shoes.
What is your favorite bug?
Aphid
Aphids smell
\end{lstlisting}

\end{frame}

\begin{frame}[fragile]\ft{\subsecname:strcmp函数} 
\begin{lstlisting}[title=strcmp函数原型, basicstyle=\ttfamily]
int strcmp(const char * s1, const char * s2);
\end{lstlisting}
\rule{\textwidth}{0.3mm} \vspace{0.3mm}

\blue{功能:}比较字符串s1和s2。若字符串相同,则返回0;否则就比较两个字符串的第一个不匹配的字符对(使用ASCII码进行比较)。\vspace{.05in}

\begin{itemize}
\item
若第一个字符串小于第二个则返回一个负数;
\item
若第一个字符串较大就返回一个整数。
\end{itemize}
\end{frame}

\begin{frame}[fragile]\ft{\subsecname:strcmp函数}
\lstinputlisting
[title=compback.c,linerange={2-18}]
{Chapters/Ch11/Code/compare.c}
\end{frame}

\begin{frame}[fragile]\ft{\subsecname:strcmp函数}
\lstinputlisting
[title=compback.c,linerange={2-16}]
{Chapters/Ch11/Code/compback.c}
\end{frame}

\begin{frame}[fragile]\ft{\subsecname:strcmp函数}
\lstinputlisting
[title=compback.c,linerange={18-25}]
{Chapters/Ch11/Code/compback.c}
\end{frame}

\begin{frame}[fragile]\ft{\subsecname:strcmp函数}
我的系统的输出结果为
\begin{lstlisting}[basicstyle=\ttfamily]
strcmp("A", "A") is 0
strcmp("A", "B") is -1
strcmp("B", "A") is 1
strcmp("C", "A") is 2
strcmp("Z", "a") is -7
strcmp("apples", "apple") is 115
\end{lstlisting}
而有些系统输出结果为
\begin{lstlisting}[basicstyle=\ttfamily]
strcmp("A", "A") is 0
strcmp("A", "B") is -1
strcmp("B", "A") is 1
strcmp("C", "A") is 1
strcmp("Z", "a") is -1
strcmp("apples", "apple") is 1
\end{lstlisting}
\end{frame}

\begin{frame}[fragile]\ft{\subsecname:strncmp函数} 
\begin{lstlisting}[title=strncmp函数原型, basicstyle=\ttfamily]
int strncmp(const char * s1, const char * s2, size_t n);
\end{lstlisting}
\rule{\textwidth}{0.3mm} \vspace{0.3mm}

\blue{功能:}比较字符串s1和s2的前n个字符或直到第一个空字符为止。返回结果与strcmp类似。
\end{frame}

\begin{frame}[fragile]\ft{\subsecname:strncmp函数}
如果想搜索以"astro"开头的字符串,可以限定搜索前5个字符。
\lstinputlisting
[title=starsrch.c,linerange={2-13}]
{Chapters/Ch11/Code/starsrch.c}

\end{frame}

\begin{frame}[fragile]\ft{\subsecname:strncmp函数}
\lstinputlisting
[title=starsrch.c,linerange={15-25}]
{Chapters/Ch11/Code/starsrch.c}
\end{frame}

\begin{frame}[fragile]\ft{\subsecname:strncmp函数}
\begin{lstlisting}[basicstyle=\ttfamily]
Found: astronomy
Found: astrophysics
The list contained 2 words beginning with astro.
\end{lstlisting}


\end{frame}

\begin{frame}[fragile]\ft{\subsecname:strcpy函数} 
\begin{lstlisting}[title=strcpy函数原型, basicstyle=\ttfamily]
char * strcpy(char * s1, const char * s2);
\end{lstlisting}
\rule{\textwidth}{0.3mm} \vspace{0.3mm}

\begin{itemize}
\item 
\blue{功能:}把字符串s2(包括空字符)复制到s1指向的位置,返回s1。\\[0.1in]
\item
s2称为源(source)字符串,s1称为目标(target)字符串。
\\[0.1in]
\item
指针s2可以是一个已声明的指针、数组名或字符串常量。\\[0.1in]
\item
指针s1应指向空间大到足够容纳字符串s2的数组。\\[0.1in]
\item[]
\red{谨记:声明一个数组将为数据分配存储空间,而声明一个指针值为一个地址分配存储空间。}
\end{itemize}
\end{frame}

\begin{frame}[fragile]\ft{\subsecname:strcpy函数}
\lstinputlisting
[title=copy1.c,linerange={2-10}]
{Chapters/Ch11/Code/copy1.c}

\end{frame}

\begin{frame}[fragile]\ft{\subsecname:strcpy函数}
\lstinputlisting
[title=copy1.c,linerange={12-27}]
{Chapters/Ch11/Code/copy1.c}
\end{frame}

\begin{frame}[fragile]\ft{\subsecname:strcpy函数}
\begin{lstlisting}[basicstyle=\ttfamily]
Enter 5 words beginning with q:
quit
quarter
quite
quotient
nomore
nomore doesn't begin with q!
quiz
Here are the words accepted:
quit
quarter
quite
quotient
quiz
\end{lstlisting}

\end{frame}

\begin{frame}[fragile]\ft{\subsecname:strcpy函数} 
strcpy还有两个重要的属性:\vspace{0.1in}
\begin{itemize}
\item 它是char *类型,返回第一个参数的值; \\[0.1in]
\item 第一个参数不需要指向数组的开始,这样就可以复制到目标字符串的指定位置。
\end{itemize}
\end{frame}

\begin{frame}[fragile]\ft{\subsecname:strcpy函数}
\lstinputlisting
[title=copy2.c,linerange={1-16}]
{Chapters/Ch11/Code/copy2.c}
\end{frame}

\begin{frame}[fragile]\ft{\subsecname:strcpy函数}
\begin{lstlisting}[basicstyle=\ttfamily]
beast
Be the best that you can be.
Be the beast
beast
\end{lstlisting}

\end{frame}

\begin{frame}[fragile]\ft{\subsecname:strncpy函数} 
\begin{lstlisting}[title=strncpy函数原型, basicstyle=\ttfamily]
char * strcpy(char * s1, const char * s2, size_t n);
\end{lstlisting}
\rule{\textwidth}{0.3mm} \vspace{0.3mm}

\blue{功能:}把字符串s2的前n个字符或直到空字符为止的字符复制到s1指向的位置,第三个参数n用于指明最大可复制的字符数。
\begin{itemize}
\item 
若源字符串的字符数小于n,则整个字符串都被复制过来,包括空字符;
\item
复制的字符数不能超过n,必须要给空字符留位置。处于这个原因,调用该函数时,n一般设置为目标数组长度减1。
\item
函数返回s1。
\end{itemize}
\end{frame}


\begin{frame}[fragile]\ft{\subsecname:strncpy函数}
\lstinputlisting
[title=copy3.c,linerange={1-10}]
{Chapters/Ch11/Code/copy3.c}
\end{frame}

\begin{frame}[fragile]\ft{\subsecname:strncpy函数}
\lstinputlisting
[title=copy3.c,linerange={12-28}]
{Chapters/Ch11/Code/copy3.c}
\end{frame}


\begin{frame}[fragile]\ft{\subsecname:strncpy函数}
\begin{lstlisting}[basicstyle=\ttfamily]
Enter 5 words begin with q:
quack
quadratic
quisling
quota
quagga
Here are the words accepted:
quack
quadra
quisli
quota
quagga
\end{lstlisting}
\end{frame}


\begin{frame}[fragile]\ft{\subsecname:sprintf函数}
sprintf函数在stdio.h中声明。 \vspace{.05in}

\begin{itemize}
\item 
作用同prinf函数一样,但它写到字符串中而不是输出显示。 \\[0.1in]
\item
第一个参数是目标字符串的地址,其余参数同printf。
\end{itemize}
\end{frame}


\begin{frame}[fragile]\ft{\subsecname:sprintf函数}
\lstinputlisting
[title=format.c,linerange={1-8}]
{Chapters/Ch11/Code/format.c}
\end{frame}

\begin{frame}[fragile]\ft{\subsecname:sprintf函数}
\lstinputlisting
[title=format.c,linerange={10-21}]
{Chapters/Ch11/Code/format.c}
\end{frame}

\begin{frame}[fragile]\ft{\subsecname:sprintf函数}
\begin{lstlisting}[basicstyle=\ttfamily]
Enter your first name:
warning: this program uses gets(), which is unsafe.
Teddy
Enter your last name:
Bear
Enter your prize money:
2000
Bear, Teddy              : $2000.00

\end{lstlisting}

\pause \rule{\textwidth}{0.3mm}\vspace{0.3mm}

sprintf函数获取输入,并把输入格式化为标准形式后存放在字符串format中。

\end{frame}