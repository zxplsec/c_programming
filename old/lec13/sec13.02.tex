\subsection{存储类说明符}

\begin{frame}[fragile]\ft{\subsecname} 
  关键字{\tf static}与{\tf extern}的含义随上下文而不同。{\tf C}语言中有5个作为存储类说明符的关键字:
  \begin{itemize}
  \item {\tf auto}
  \item {\tf register}
  \item {\tf static}
  \item {\tf extern}
  \item {\tf typedef}:它与内存存储无关,由于语法原因被归入此类。
  \end{itemize}
\end{frame}


\begin{frame}[fragile]\ft{\subsecname}
  \begin{zhu}
    不可以在一个声明中使用一个以上存储类说明符,这意味着\textcolor{acolor3}{不能将其它任一存储类说明符作为{\tf typedef}的一部分}。
  \end{zhu}
\end{frame}


\begin{frame}[fragile]\ft{\subsecname}
  {\tf auto}表明一个变量具有自动存储时期,它只能用在具有代码块作用域的变量声明中。使用它仅用于明确指出意图,使程序更易读。
\end{frame}

\begin{frame}[fragile]\ft{\subsecname}
  {\tf register}也只能用在具有代码块作用域的变量声明中。它将一个变量归入寄存器存储类,这相当于请求将该变量存储在一个寄存器内,以更快地存取。

  {\tf register}的使用将导致不能获取变量的地址。
\end{frame}

\begin{frame}[fragile]\ft{\subsecname}
  对于{\tf static},
  \begin{itemize}
  \item 用于具有代码块作用域的变量声明时,使该变量具有静态存储时期,从而得以在程序运行期间存在并保留其值。此时,变量仍具有代码作用域和空链接。
  \item 用于具有文件作用域的变量声明时,表明该变量具有内部链接。
  \end{itemize}
\end{frame}

\begin{frame}[fragile]\ft{\subsecname}
  {\tf extern}表明你在声明一个已经在别处定义的了变量,
  \begin{itemize}
  \item 若该声明具有文件作用域,所指向的变量必然具有外部链接;
  \item 若该声明具有代码块作用域,所指向的变量可能具有外部链接也可能具有内部链接,这取决于该变量的定义声明。
  \end{itemize}
\end{frame}

\begin{frame}[fragile]\ft{总结}
  \begin{itemize}
  \item 自动变量具有代码块作用域、空链接和自动存储时期。它们是局部的,为定义它们的代码所私有。

  \item 寄存器变量与自动变量具有相同的属性,但编译器可能使用速度更快的内存或寄存器来存储它们。无法获取一个寄存器变量的地址。
  \end{itemize}

  
\end{frame}

\begin{frame}[fragile]\ft{总结}
  具有静态存储时期的变量可能具有外部链接、内部链接或空链接。
    \begin{itemize}
    \item 当变量在文件的所有函数之外声明时,它是一个具有文件作用域的外部变量,具有外部链接和静态存储时期。\\[0.1in]
    \item 若在这样的声明中再加上{\tf static},将获得一个具有静态存储时期、文件作用域和内部链接的变量。\\[0.1in]
    \item 若在一个函数内使用关键字{\tf static}声明变量,变量将具有静态存储时期、代码块作用域和空链接。
    \end{itemize}
 
\end{frame}

\begin{frame}[fragile]\ft{总结}
  \begin{itemize}
  \item 当程序执行到包含变量声明的代码块时,给具有自动存储时期的变量分配内存,并在代码块结束时释放内存。如果没有初始化,该变量将是垃圾值。\\[0.1in]
  \item 在程序编译时给具有静态存储时期的变量分配内存,并在程序运行时一直保持。若没有初始化,将被设置为0。
  \end{itemize}
\end{frame}

\begin{frame}[fragile]\ft{总结}
  \begin{itemize}
  \item 具有代码块作用域的变量局部于包含变量声明的代码块。\\[0.1in]
  \item 具有文件作用域的变量对文件中在它声明之后的所有函数可见。
    \begin{itemize}
    \item 若一个文件作用域变量具有外部链接,则它可被程序中的其他文件使用;
    \item 若一个文件作用域变量具有内部链接,则它只能在声明它的文件中使用。
    \end{itemize}
  \end{itemize}
\end{frame}

