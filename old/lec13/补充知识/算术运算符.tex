\documentclass[10pt,a4paper%,twoside,openright,titlepage,fleqn,%
%headinclude,footinclude,BCOR5mm,%
%numbers=noenddot,cleardoublepage=empty,%
tablecaptionabove]{article}

\usepackage{geometry}
\geometry{left=2.5cm,right=2.5cm,top=2.5cm,bottom=2.5cm}

\usepackage{amsmath,amssymb,amsthm}

%% -----------------设置数学公式字体-------------------------
%% Font style 1
%% \newcommand\ibinom[2]{\genfrac\lbrace\rbrace{0pt}{}{#1}{#2}}
%% \usepackage{bm}

%% Font style 2
%% \newcommand\ibinom[2]{\genfrac\lbrace\rbrace{0pt}{}{#1}{#2}} 
%% \usepackage[boldsans]{ccfonts} 
%% \usepackage{bm} 

%% Font style 3
\newcommand\ibinom[2]{\genfrac\lbrace\rbrace{0pt}{}{#1}{#2}}
\usepackage[euler-digits]{eulervm}
\usepackage{bm}

%% Font style 4
%% \usepackage{fourier}
%% \newcommand\ibinom[2]{\genfrac\lbrace\rbrace{0pt}{}{#1}{#2}}
%% \usepackage{bm}

%% Font style 5
%% \newcommand\ibinom[2]{\genfrac\lbrace\rbrace{0pt}{}{#1}{#2}}
%% \usepackage{mathptmx}
%% \usepackage{bm} 


%% %% Font style 6
%% \newcommand\ibinom[2]{\genfrac\lbrace\rbrace{0pt}{}{#1}{#2}}
%% \usepackage{txfonts}
%% \usepackage{bm}
%% -----------------------------------------------------------

\usepackage{titlesec} %设置标题
\usepackage{titletoc}

\usepackage{extarrows}
\usepackage{verbatim,color,xcolor}
\usepackage{pgf}
\usepackage{tikz}
\usetikzlibrary{calc}
\usetikzlibrary{arrows,snakes,backgrounds,shapes,patterns}
\usetikzlibrary{matrix,fit,positioning,decorations.pathmorphing}
%% \usepackage{classicthesis}
\usepackage{CJK}
\usepackage{mathdots}

\usepackage{listings}
\lstset{
  keywordstyle=\color{blue!70},
  frame=single,
  basicstyle=\ttfamily\small,
  commentstyle=\small\color{red},
  breakindent=0pt,
  rulesepcolor=\color{red!20!green!20!blue!20},
  rulecolor=\color{black},
  tabsize=4,
  numbersep=5pt,
  breaklines=true,
  %% backgroundcolor=\color{red!10},
  showspaces=false,
  showtabs=false,
  extendedchars=false,
  escapeinside=``,
  frame=no,
}


\newcommand{\blue}{\textcolor{blue}}
\newcommand{\red}{\textcolor{red}}
\newcommand{\purple}{\textcolor{electricpurple}}
\newcommand{\ds}{\displaystyle}
\newcommand{\cd}{\cdots}
\newcommand{\dd}{\ddots}
\newcommand{\vd}{\vdots}
\newcommand{\id}{\iddots}

\newcommand{\R}{\mathbb R}
\def\nn{{\boldsymbol{n}}}
\def\xx{{\boldsymbol{x}}}
\def\F{{\boldsymbol{F}}}
\def\div{{\mathrm{div}}}
\def\tf{\ttfamily}


\begin{document}

\begin{CJK}{UTF8}{gkai}
 

\newtheorem{li}{例}
\newtheorem{jielun}{结论}
\newtheorem{dingli}{定理}
\newtheorem{mingti}{{命题}} 
\newtheorem{yinli}{{引理}} 
\newtheorem{tuilun}{{推论}}
\newtheorem{dingyi}{{定义}} 
\newtheorem{example}{{例}}
\newtheorem*{example*}{{例}}
\newtheorem*{jie}{{解}}
\newtheorem*{zhengming}{{证明}}
\newtheorem{zhu}{{注}}
\newtheorem*{zhu*}{{注}}
\newtheorem{xingzhi}{{性质}}
\newtheorem{wenti}{{问题}}
\newtheorem{rem}{{Remark}}
\newtheorem{lem}{{Lemma}}
\pagenumbering{roman}
\pagestyle{plain}

\pagenumbering{arabic}

\title{算术运算符}
%\author{张晓平}
%\date{}                                           % Activate to display a given date or no date
\maketitle

运算符是任何程序语言的基础,运算符使得我们能对操作数做不同的运算。在C中,运算符可分为以下几类:
\begin{enumerate}
\item 算术运算符{ \tf (+, -, *, /, \%, 后缀自增,前缀自增,后缀自减,前缀自减)}
\item 复制运算符{ \tf (=,+=,-=,*=,/=,\%=,...)}
\item 关系运算符{ \tf (==,!=,>,<,>=,<=)}
\item 逻辑运算符{ \tf (\&\&,||,!)}
\item 位运算符{ \tf (\&,|,\^{},~,>>,<<)}
\item 其他运算符{\tf (条件运算符,逗号运算符,sizeof,取址运算符\&,取值运算符*)}
\end{enumerate}
算术运算符用于对操作数执行算术/数学操作。二元操作符包括:
\begin{itemize}
\item 加法:运算符 {\tf +} 对两个操作数相加。如{\tf x + y}。
\item 减法:运算符 {\tf -} 将第一个操作数减去第二个操作数。如{\tf x - y}。
\item 乘法:运算符 {\tf +} 对两个操作数相加。如{\tf x * y}。
\item 除法:运算符 {\tf /} 将第一个操作数除以第二个操作数。如{\tf x / y}。
\item 求余:运算符 {\tf \%} 求第一个操作数除以第二个操作数的余数。如{\tf x \% y}。
\end{itemize}
\begin{lstlisting}[language=c,backgroundcolor=\color{red!10}]
// C program to demonstrate working of binary arithmetic operators
#include<stdio.h>
 
int main()
{
    int a = 10, b = 4, res;
 
    //printing a and b
    printf("a is %d and b is %d\n", a, b);
 
    res = a+b; //addition
    printf("a+b is %d\n", res);
 
    res = a-b; //subtraction
    printf("a-b is %d\n", res);
 
    res = a*b; //multiplication
    printf("a*b is %d\n", res);
 
    res = a/b; //division
    printf("a/b is %d\n", res);
 
    res = a%b; //modulus
    printf("a%%b is %d\n", res);
 
    return 0;
}  
\end{lstlisting}
\begin{lstlisting}[backgroundcolor=\color{red!10}]
Output:
a is 10 and b is 4
a+b is 14
a-b is 6
a*b is 40
a/b is 2
a\%b is 2  
\end{lstlisting}
一元运算符包括:
\begin{itemize}
\item 自增: 运算符 {\tf ++} 使得一个整型值加1。 当它作用于变量名之前时(称为前缀自增运算符),其值立即自增1;而当它作用于变量名之后时(称为后缀自增运算符),其值暂时保持不变,直至所在语句执行完毕,而在下一条语句执行之前将自增1。
\item 自减:运算符 {\tf --} 使得一个整型值减1。 当它作用于变量名之前时(称为前缀自减运算符),其值立即自减1;而当它作用于变量名之后时(称为后缀自减运算符),其值暂时保持不变,直至所在语句执行完毕,而在下一条语句执行之前将自减1。
\end{itemize}
\begin{lstlisting}[language=c,backgroundcolor=\color{red!10}]
// C program to demonstrate working of Unary arithmetic operators
#include<stdio.h>
 
int main()
{
    int a = 10, b = 4, res;
 
    // post-increment example:
    // res is assigned 10 only, a is not updated yet
    res = a++;
    printf("a is %d and res is %d\n", a, res); //a becomes 11 now
 
 
    // post-decrement example:
    // res is assigned 11 only, a is not updated yet
    res = a--;
    printf("a is %d and res is %d\n", a, res);  //a becomes 10 now
 
 
    // pre-increment example:
    // res is assigned 11 now since a is updated here itself
    res = ++a;
    // a and res have same values = 11
    printf("a is %d and res is %d\n", a, res);
 
 
    // pre-decrement example:
    // res is assigned 10 only since a is updated here itself
    res = --a;
    // a and res have same values = 10
    printf("a is %d and res is %d\n",a,res); 
 
    return 0;
}
  
\end{lstlisting}
\begin{lstlisting}[backgroundcolor=\color{red!10}]
Output:
a is 11 and res is 10
a is 10 and res is 11
a is 11 and res is 11
a is 10 and res is 10  
\end{lstlisting}


\end{CJK}
\end{document}
