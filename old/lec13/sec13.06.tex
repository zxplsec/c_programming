\section{内存的动态分配}

\begin{frame}[fragile]\ft{\secname}
  本节主要介绍四个函数
  \begin{itemize}
  \item \tf{malloc()}
  \item \tf{calloc()}
  \item \tf{realloc()}
  \item \tf{free()}
  \end{itemize}
\end{frame}


\begin{frame}[fragile]\ft{\secname}
回顾一下,对于如下声明
\begin{lstlisting}[language=c,backgroundcolor=\color{red!20}]
float x;
char place[]="Beijing'';  
\end{lstlisting}
此时系统将留出足够存储{\tf float}或字符串的内存空间。而声明
\begin{lstlisting}[language=c,backgroundcolor=\color{red!20}]
char plates[100];  
\end{lstlisting}
将留出{\tf 100}个内存位置,每个位置可存储一个{\tf int}值。
\end{frame}

\begin{frame}[fragile]\ft{\secname}
  C的功能远不止如此,可以在程序运行时分配更多的内存。  
  主要工具是{\tf malloc()},
  \begin{itemize}
  \item 
    它接受一个参数:\textcolor{acolor1}{所需内存字节数}。
  \item
    然后{\tf malloc()}找到可用内存中一个大小合适的块。
  \item
    内存是匿名的,也就是说,{\tf malloc()}分配了内存,
    但没有为它指定名字。但它可以返回那块内存中第一个字节的地址。
  \item
    于是,你可以把该地址赋值给一个指针变量,并使用该指针来访问那块内存。
  \end{itemize}
\end{frame}


\begin{frame}[fragile]\ft{\secname}

  \begin{itemize}
  \item 
    因{\tf char}代表一个字节,故传统上曾将{\tf malloc()}定义为指向{\tf char}的指针类型。
  \item
    {\tf ANSI C}标准使用了一个新类型:指向{\tf void}的指针,并允许将{\tf void}指针赋值给其他类型的指针。
  \item
    若{\tf malloc()}找不到所需的空间,则返回空指针。
  \end{itemize}
  
\end{frame}


\begin{frame}[fragile]\ft{\secname}
  以下展示如何用{\tf malloc()}创建一个数组。
 \begin{lstlisting}[language=c,backgroundcolor=\color{red!20}]
 double * ptd;
 ptd = (double *) malloc(30 * sizeof(double));
 \end{lstlisting}
\rule{\textwidth}{0.3mm} \pause \vspace{0.3mm}

这段代码请求{\tf 30}个{\tf double}值的空间,并把{\tf ptd}指向该空间所在位置。
\end{frame}

\begin{frame}[fragile]\ft{\secname}
  需要注意的是,{\tf ptd}是作为指向一个{\tf double}值的指针声明的,而不是指向{\tf 30}个{\tf double}值的数据块的指针。\vspace{0.2in}

  而数组的名字是其第一个元素的地址,因此,如果令{\tf ptd}指向一个内存块的第一个元素,就可以像使用数组名一样使用它。
\end{frame}

\begin{frame}[fragile]\ft{\secname}
  创建数组的三种方法:
  \begin{itemize}
  \item 声明一个数组,声明时用常量表达式指定数组维数,然后用数组名访问各元素;\\[0.1in]
  \item 声明一个变长数组,声明时用变量表达式指定数组维数,然后用数组名访问各元素;\\[0.1in]
  \item 声明一个指针,调用{\tf malloc()},然后使用该指针来访问数组元素。
  \end{itemize}
  \rule{\textwidth}{0.3mm} \pause \vspace{0.3mm}

  使用第二种或第三种方法可以创建一个动态数组。
\end{frame}

\begin{frame}[fragile]\ft{\secname}
  在{\tf C99}之前,不允许这么做:
 \begin{lstlisting}[language=c,backgroundcolor=\color{red!20}]
 double item[n];
 \end{lstlisting} \vspace{0.1in}

 然而,即使在{\tf C99}之前,也可以这么做
 \begin{lstlisting}[language=c,backgroundcolor=\color{red!20}]
 ptd = (double *) malloc(n * sizeof(double));
 \end{lstlisting}  
\end{frame}


\begin{frame}[fragile]\ft{\secname}
  一般地,分配了内存,就应该释放内存。释放内存的主要工具是{\tf free()}。
  \begin{itemize}
  \item {\tf free()}的参数是{\tf malloc()}返回的地址,它释放先前分配的地址。
  \item 所分配内存的持续时间从调用{\tf malloc()}分配内存开始,到调用{\tf free()}释放内存结束。
  \end{itemize}
\end{frame}


\begin{frame}[fragile]\ft{\secname}
  设想{\tf malloc()}函数与{\tf free()}管理一个内存池。每次调用{\tf malloc()}函数分配内存供程序使用,每次调用{\tf free()}将内存归还到池中。
\end{frame}

\begin{frame}[fragile,allowframebreaks]\ft{\secname}
  \lstinputlisting
  [language=c,numbers=left,frame=single]
  {code/dyn_arr.c}
\end{frame}


\begin{frame}[fragile]\ft{\secname}
  \begin{lstlisting}[language=c,backgroundcolor=\color{red!20}] 
What is the maximum number of type double entries?
8
Enter the values (q to quit): 
1 2 3 4 5 6  7 8
Here are your 8 entries:
   1.00    2.00    3.00    4.00    5.00    6.00    7.00 
   8.00 
Done.   
  \end{lstlisting}
\end{frame}

\begin{frame}[fragile]\ft{\secname}
  注意在程序末尾的{\tf free()},用于释放{\tf malloc()}函数分配的内存。{\tf free()}只释放其参数所指向的内存块。
  \vspace{0.1in}

  该例中,{\tf free()}不是必须的,因为程序终止后所有已分配的内存都将被自动释放。
\end{frame}


\begin{frame}[fragile]\ft{\secname}
  为什么会使用动态数组?主要是获得了程序的灵活性。\vspace{0.1in}

  假定某程序在大多数时候需要的数组元素不超过100个;而在某些情况下,却需要10000个元素。在声明数组时,不得不考虑最坏情形并声明一个长度为10000的数组。这样一来,在大多数情况下,程序将浪费内存。
\end{frame}


\begin{frame}[fragile,allowframebreaks]\ft{\secname:{\tf free()}的重要性}
  观察以下代码:
  \begin{lstlisting}[language=c,numbers=left,frame=single]
int main(void)
{
  double glad[2000];
  int i;
  ...
  for (i = 0; i < 1000; i++)
    gobble(glad, 2000);
  ...  
}

void gobble(double ar[], int n)
{
  double * tmp = (double *) malloc(n * sizeof(double));
  ...
  // free(tmp);
}
  \end{lstlisting}
\end{frame}

\begin{frame}[fragile]\ft{\secname:{\tf free()}的重要性}
  第一次调用{\tf gobble()}时,它创建了指针{\tf tmp},并使用{\tf malloc()}为之分配了{\tf 16000}字节的内存。
  
  若没有调用{\tf free()},当函数终止时,指针{\tf tmp}作为一个自动变量消失了,但它所指向的{\tf 16000}字节的内存依然存在。我们无法访问这些内存,因为地址不见了。
\end{frame}

\begin{frame}[fragile]\ft{\secname:{\tf free()}的重要性}
  第二次调用{\tf gobble()}时,它又创建了指针{\tf tmp},再次使用{\tf malloc()}为之分配了{\tf 16000}字节的内存。

  第一个{\tf 16000}字节的块已不可用,因此{\tf malloc()}不得不再找一个{\tf 16000}字节的块。当函数终止时,这个内存块也无法访问,不可再用。
\end{frame}

\begin{frame}[fragile]\ft{\secname:{\tf free()}的重要性}
   循环执行了{\tf 1000}次,当循环结束时,已经有{\tf 1600}万字节的内存从内存池中移走。事实上,在到达这一步之前,程序可能已经内存溢出了。这类问题被称为内存泄露{\tf (memory leak)},可以在函数末尾处调用{\tf free()}防止该问题出现。
\end{frame}

\begin{frame}[fragile]\ft{\secname:calloc函数}
  内存分配还可以使用{\tf calloc()},典型应用如
  \begin{lstlisting}[language=c,backgroundcolor=\color{red!20}]
long * newmem;
newmem = (long *) calloc(100, sizeof(long));    
  \end{lstlisting}
  \rule{\textwidth}{0.3mm} \pause \vspace{0.3mm}
  
  \begin{itemize}
  \item   {\tf calloc()}的第一个参数是所需内存单元的数量,第二个参数是每个单元的字节大小,返回类型同{\tf malloc()}函数。\\[0.1in]
  \item
    {\tf calloc()}的另一个特性:\textcolor{acolor3}{它将块中的全部位置都设置为0。}\\[0.1in]
  \item
    {\tf free()}也可用来释放由{\tf calloc()}分配的内存。
  \end{itemize}
\end{frame}

\begin{frame}[fragile]\ft{\secname:realloc函数}
可以使用{\tf realloc()}给一个已经分配了地址的指针重新分配空间,其函数原型声明为
  \begin{lstlisting}[language=c,backgroundcolor=\color{red!20}]
void * realloc(void * ptr, unsigned newsize)    
  \end{lstlisting}
  \rule{\textwidth}{0.3mm} \pause \vspace{0.3mm}
  
  \begin{itemize}
  \item 参数{\tf ptr}为原有的空间地址;
  \item 参数{\tf newsize}是重新申请的地址长度。
  \end{itemize}
\end{frame}

\begin{frame}[fragile]\ft{\secname:realloc函数}
  {\tf realloc()}可以对给定的指针所指的空间进行扩大或者缩小,无论是扩张或是缩小,原有内存的中内容将保持不变。当然,对于缩小,则被缩小的那一部分的内容会丢失。

  \textcolor{acolor1}{{\tf realloc()}并不保证调整后的内存空间和原来的内存空间保持同一内存地址。}相反,{\tf realloc()}返回的指针很可能指向一个新的地址。
\end{frame}

\begin{frame}[fragile]\ft{\secname:realloc函数}
  {\tf realloc()}是从堆上分配内存的。

  当扩大一块内存空间时,{\tf realloc()}试图直接从堆上现存的数据后面的那些字节中获得附加的字节,
  \begin{itemize}
  \item  如果能够满足,自然天下太平;
  \item  如果数据后面的字节不够,问题就出来了,那么就使用堆上第一个有足够大小的自由块,现存的数据然后就被拷贝至新的位置,而老块则放回到堆上。
  \end{itemize}
\textcolor{acolor1}{这句话传递的一个重要的信息就是数据可能被移动。}
\end{frame}

\begin{frame}[fragile]\ft{\secname:realloc函数}
  \lstinputlisting
  [language=c,numbers=left,frame=single]
  {code/realloc1.c}
\end{frame}

\begin{frame}[fragile]\ft{\secname:realloc函数}
  \begin{lstlisting}[backgroundcolor=\color{red!20}] 
p = 0x110a010
q = 0x110a010
\end{lstlisting}
{\tf realloc()}后,内存地址没有发生改变。
\end{frame}

\begin{frame}[fragile]\ft{\secname:realloc函数}
  \lstinputlisting
  [language=c,numbers=left,frame=single]
  {code/realloc2.c}
\end{frame}

\begin{frame}[fragile]\ft{\secname:realloc函数}
  \begin{lstlisting}[backgroundcolor=\color{red!20}] 
p = 0x7fc2c9716010
q = 0x1574010
\end{lstlisting}
{\tf realloc()}后,内存地址发生了改变。
\end{frame}



\begin{frame}[fragile]\ft{\secname:动态内存分配与变长数组}
  VLA与{\tf malloc()}在功能上有些一致,比如它们都可以用来创建一个大小在运行时决定的数组:
  \begin{lstlisting}[language=c,backgroundcolor=\color{red!20}]
int vlamal()
{
  int n;
  int * pi;
  scanf(``%d", &n);
  pi = (int *) malloc(n * sizeof(int));

  int ar[n];
  pi[2] = ar[2] = 5;
  ...
}    
  \end{lstlisting}
\end{frame}


\begin{frame}[fragile]\ft{\secname:动态内存分配与变长数组}
  区别一:\vspace{0.05in}

  \begin{itemize}
  \item VLA是自动存储的,其结果之一是VLA所用内存空间在运行后会自动释放,不必使用{\tf free()}。\\[0.1in]
  \item {\tf malloc()}创建的数组不必局限于在一个函数中。例如,函数可以创建一个数组并返回指针,供调用它的函数访问。
    然后,后者可以在它结束时调用{\tf free()}。
  \end{itemize}
\end{frame}


\begin{frame}[fragile]\ft{\secname:动态内存分配与变长数组}
  区别二:\vspace{0.05in}

  \begin{itemize}
  \item  VLA对多维数组来说更为方便。  \\[0.1in]
  \item  也可以使用{\tf malloc()}来定义一个二维数组,但语法较麻烦。
  \end{itemize}

  \begin{lstlisting}[language=c,backgroundcolor=\color{red!20}]
int n = 5;
int m = 6;
int ar1[n][m];   // VLA
int (* p1) [6];  // Before C99
int (* p2) [m];  // C99    
  \end{lstlisting}
\end{frame}


\begin{frame}[fragile,allowframebreaks]\ft{\secname:动态内存分配与变长数组}
  \lstinputlisting
  [language=c,numbers=left,frame=single]
  {code/dyn_ar2d.c}
\end{frame}



\begin{frame}[fragile]\ft{\secname}
  \begin{lstlisting}[language=c,backgroundcolor=\color{red!20}]
   0    1    2    3    4 
   1    2    3    4    5 
   2    3    4    5    6 
   3    4    5    6    7 
  \end{lstlisting}
\end{frame}
