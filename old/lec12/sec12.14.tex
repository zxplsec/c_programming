\section{指针函数与函数指针}

\subsection{指针函数}
\begin{frame}[fragile]\ft{\subsecname}
先看下面的函数声明,注意,此函数有返回值,返回值为{\tf int *},即返回值是指针类型的。
  \begin{lstlisting}[language=c,backgroundcolor=\color{red!20}]
int * f (int a, int b);    
  \end{lstlisting}
\end{frame}


\begin{frame}[fragile,allowframebreaks]\ft{\subsecname}
\lstinputlisting
[language=c,numbers=left,frame=single]
{code/ptr_fun.c}
\end{frame}


\begin{frame}[fragile]\ft{\subsecname}
  \begin{lstlisting}[backgroundcolor=\color{blue!20}]
The memeory address of p1 = (nil) 
The memeory address of p = 0x12c0010 
*p = 3 
The memeory address of p1 = 0x12c0010 
*p1 = 3     
  \end{lstlisting}
\end{frame}


\subsection{函数指针}
\begin{frame}[fragile]\ft{\subsecname}
函数指针说的就是一个指针,但这个指针指向函数,不是普通的基本数据类型或者类对象。
  \begin{lstlisting}[language=c,backgroundcolor=\color{red!20}]
int (* f) (int a, int b);    
  \end{lstlisting}
  \begin{itemize}
  \item
    函数指针与指针函数的最大区别是函数指针的函数名是一个指针,即函数名前面有一个{\tf *}。 
  \item
    上面的函数指针定义为一个指向一个返回值为整型,有两个参数并且两个参数的类型都是整型的函数。
  \end{itemize}
\end{frame}

\begin{frame}[fragile,allowframebreaks]\ft{\subsecname}
\lstinputlisting
[language=c,numbers=left,frame=single]
{code/fun_ptr.c}
\end{frame}


\begin{frame}[fragile]\ft{\subsecname}
  \begin{lstlisting}[backgroundcolor=\color{blue!20}]
The max value is 2 
The min value is 1 
  \end{lstlisting}
\end{frame}
