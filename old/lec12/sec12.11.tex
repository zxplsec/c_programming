\section{枚举类型}

\begin{frame}[fragile]\ft{\secname}
可使用枚举类型{\tf (enumerated type)}声明代表整数常量的符号名称。用关键字{\tf enum},可创建一个新“类型”并指定它可以具有的值。
实际上,{\tf enum}常量是{\tf int}类型,故在使用{\tf int}类型的任何地方都可使用它。\vspace{0.1in}

枚举类型的目的是为了提高程序可读性,其语法与结构相同。
\end{frame}

\begin{frame}[fragile]\ft{\secname}
  声明方式如下:
  \begin{lstlisting}[language=c,backgroundcolor=\color{red!20}]
enum spectrum {red, orange, yellow, green, blue};
enum spectrum color;
  \end{lstlisting}
  \begin{itemize}
  \item 第一个声明设置{\tf spectrum}为标记名,从而允许你把{\tf enum spectrum}作为一个类型名使用。\\[0.1in]
  \item 第二个声明使得{\tf color}成为该类型的一个变量。花括号中的标识符枚举了{\tf spectrum}变量可能有的值。
  \end{itemize}
\end{frame}

\begin{frame}[fragile]\ft{\secname}
可以使用以下语句:
  \begin{lstlisting}[language=c,backgroundcolor=\color{red!20}]
int c;
color = blue;
if (color == yellow)
  ...
for (color = red; color <= blue; color++)
  ...
  \end{lstlisting}
实际上,为{\tf spectrum}枚举的常量在{\tf 0}到{\tf 5}之间。
\end{frame}

\begin{frame}[fragile]\ft{\secname}
执行以下代码:
  \begin{lstlisting}[language=c,backgroundcolor=\color{red!20}]
printf("red = %d, orange = %d\n", red, orange);
  \end{lstlisting}
结果为
  \begin{lstlisting}[language=c,backgroundcolor=\color{red!20}]
red = 0, orange = 1
  \end{lstlisting}
{\tf red}为一个代表整数{\tf 0}的命名常量,其他标识符分别是代表{\tf 1}到{\tf 5}的命名常量。
\end{frame}

\begin{frame}[fragile]\ft{\secname}
默认时,枚举列表中的常量被指定为整数值{\tf 0、1、2}等。故,以下声明使得{\tf nina}具有值{\tf 3}:
  \begin{lstlisting}[language=c,backgroundcolor=\color{red!20}]
enum kids {nippy, slats, skippy, nina, liz};
  \end{lstlisting}

\end{frame}

\begin{frame}[fragile]\ft{\secname}
  \begin{itemize}
  \item 
也可指定常量具有特定的整数值:
  \begin{lstlisting}[language=c,backgroundcolor=\color{red!20}]
enum levels {low = 100, medium = 500, high = 2000};
  \end{lstlisting}
\item 若只对一个常量赋值,而没对后面的常量赋值,则后面的常量会被赋予后续的值:
  \begin{lstlisting}[language=c,backgroundcolor=\color{red!20}]
enum feline {cat, lynx = 10, puma, tiger};
  \end{lstlisting}
则{\tf cat}的默认值为{\tf 0},{\tf lynx}、{\tf puma}、{\tf tiger}的默认值分别为{\tf 10、11、12}。
  \end{itemize}
\end{frame}

\begin{frame}[fragile]\ft{\secname:enum的用法}
枚举类型的目的是为了提高程序可读性。如果是处理颜色,采用{\tf red}和{\tf blue}要比使用{\tf 0}和{\tf 1}更显而易见。
\end{frame}

\begin{frame}[fragile,allowframebreaks]\ft{\secname:enum的用法}
\lstinputlisting
[language=c,numbers=left,frame=single]
{code/enum.c}
\end{frame}


\begin{frame}[fragile,allowframebreaks]\ft{\secname:enum的用法}
  \begin{lstlisting}[backgroundcolor=\color{blue!20}]
Enter a color (empty line to quit):
orange
Poppies are orange.
Next color, please (empty line to quit): 
blue
Bluebells are blue.
Next color, please (empty line to quit): 
red
Roes are red.
Next color, please (empty line to quit): 
sdf
I don't know about the color sdf.
Next color, please (empty line to quit): 

Goodbye!
\end{lstlisting}
\end{frame}

