\section{结构体数组}

\subsection{示例程序}

\begin{frame}[fragile,allowframebreaks]\ft{\subsecname}
\lstinputlisting
[language=c,numbers=left,frame=single]
{code/manybook.c}
\end{frame}

\begin{frame}[fragile]\ft{\subsecname}
  \begin{lstlisting}[backgroundcolor=\color{blue!20}]
Enter the book title.
Press [enter] at the start of a line to stop.
C primer plus[enter]
Enter the author.
Stephan Prata[enter]
Enter the value.
80[enter]
Enter the next title.
C programing language[enter]
Enter the author.
Dennis Ritchie[enter]
Enter the value.
40[enter]
Enter the next title.

  \end{lstlisting}
\end{frame}

\begin{frame}[fragile]\ft{\subsecname}
  \begin{lstlisting}[backgroundcolor=\color{blue!20}]
Here is the list of your book:
C primer plus by Stephan Prata: 80.00
C programing language by Dennis Ritchie: 40.00    
  \end{lstlisting}
\end{frame}

\subsection{声明结构体数组}
\begin{frame}[fragile]\ft{\subsecname}
  \begin{lstlisting}
struct book library[MAXBOOK];
  \end{lstlisting}
声明一个具有{\tf MAXBOOK}个元素的数组,每个元素都是一个{\tf book}类型的结构体。\vspace{0.1in}

注意:{\tf library}本身不是结构体变量名,它是元素类型为{\tf struct book}的数组名。
\end{frame}

\subsection{标识结构体数组的成员}
\begin{frame}[fragile]\ft{\subsecname}
  为了标识结构体数组的成员,可这么做
  \begin{lstlisting}[language=c,backgroundcolor=\color{red!20}]
library[0].value;     //`第1个数组元素的value成员`
library[4].title;     //`第5个数组元素的title成员`
library[2].title[4];  //`第3个数组元素的title成员`
                      //`   的第5个字符`
  \end{lstlisting}
\end{frame}

