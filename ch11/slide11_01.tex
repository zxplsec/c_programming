\section{字符串表示}

\begin{frame}[fragile]\ft{\secname}
\begin{itemize}
\item
字符串是以空字符结尾的\lstinline| char |数组,故关于数组与指针的知识也适用于字符串。
\\[0.15in]
\item
由于字符串的使用非常广泛,C提供了很多专为字符串设计的函数。
\end{itemize}
\end{frame}

\begin{frame}[fragile]\ft{\secname:字符串常量}

字符串常量是指被一对双引号括起来的任何字符。\vspace{0.1in}

\begin{lstlisting}[basicstyle=\ttfamily,showstringspaces=true]
char greeting[50] = "How" " are" " you?";
\end{lstlisting}
等价于
\begin{lstlisting}[basicstyle=\ttfamily,showstringspaces=true]
char greeting[50] = "How are you?";
\end{lstlisting}
\end{frame}

\begin{frame}[fragile]\ft{\secname:字符串常量}
若想在字符串中使用双引号,可在双引号前加反斜杠\lstinline| \ |。\vspace{0.1in}

\begin{lstlisting}[basicstyle=\ttfamily,showstringspaces=true]
printf("\"Ready, go!\" exclaimed John.");
\end{lstlisting}\vspace{0.1in}

输出结果为
\begin{lstlisting}[basicstyle=\ttfamily,showstringspaces=true]
"Ready, go!" exclaimed John.
\end{lstlisting}
\end{frame}

\begin{frame}[fragile]\ft{\secname:字符串常量}
字符串常量属于静态存储(static storage)类。\vspace{0.1in}

\red{所谓静态存储是指如果在一个函数中使用字符串常量,即使多次调用了这个函数,该字符串在程序的运行过程中只存储一份。}\vspace{0.1in}

整个引号中的内容作为指向该字符存储位置的指针。
\end{frame}

\begin{frame}[fragile]\ft{\secname:字符串常量}
\lstinputlisting
[language=c,numbers=left,frame=single,showstringspaces=true]
{ch11/code/quotes.c}
\pause \vspace{0.1in}

输出结果为
\begin{lstlisting}[basicstyle=\ttfamily,showspaces=true]
We, 0x100000f81, s
\end{lstlisting}
\end{frame}

\begin{frame}[fragile]\ft{\secname:字符串数组及其初始化}
初始化方式一:指定一个足够大的数组来容纳字符串。
\begin{lstlisting}[language=c]
char s[40] = "Hello world!"
\end{lstlisting} \pause \vspace{0.1in}

这相比于标准的数组初始化要简洁很多:
\begin{lstlisting}[language=c]
char s[] = {'H', 'e', 'l', 'l', 'o', ' ', 
            'w', 'o', 'r', 'l', 'd', '!', 
            '\0'};
\end{lstlisting}
如果没有空字符,则得到的是一个字符数组而不是一个字符串。
\end{frame}

\begin{frame}[fragile]\ft{\secname:字符串数组及其初始化}
\red{指定数组大小时,要确保数组长度比字符串长度至少多1,未被初始化的元素均被自动设置为空字符。}
\end{frame}

\begin{frame}[fragile]\ft{\secname:字符串数组及其初始化}
初始化方式二:省略数组大小,让编译器决定。
\begin{lstlisting}[language=c,showstringspaces=true]
char s[] = "Hello world!"
\end{lstlisting} \pause \vspace{0.1in}

初始化方式三:使用指针符号建立字符串。
\begin{lstlisting}[language=c,showstringspaces=true]
char * s = "Hello world!"
\end{lstlisting} \pause \vspace{0.1in}

这两种方式都声明\lstinline| s |是一个指向给定字符串的指针。
\end{frame}

\begin{frame}[fragile]\ft{\secname:数组与指针的差别}
考虑如下两个声明:
\begin{lstlisting}[language=c,showstringspaces=true]
char heart[] = "I love C!";
char * head = "I love math!";
\end{lstlisting}
\vspace{0.1in}

主要的差别在于数组名\lstinline| heart |是个常量,而指针\lstinline| head |则是个变量。
\end{frame}

\begin{frame}[fragile]\ft{\secname:数组与指针的差别}
(1)、两者都可以使用数组符号。
\begin{lstlisting}[language=c,showstringspaces=true]
for (i = 0; i < 6; i++)
  putchar(heart[i]);
putchar('\n');
for (i = 0; i < 6; i++)
  putchar(head[i]);
putchar('\n');
\end{lstlisting}
\vspace{0.1in}

输出结果为
\begin{lstlisting}[basicstyle=\ttfamily,showstringspaces=true]
I love
I love
\end{lstlisting}
\end{frame}

\begin{frame}[fragile]\ft{\secname:数组与指针的差别}
(2)、两者都可以使用指针加法。
\begin{lstlisting}[language=c,showstringspaces=true]
for (i = 0; i < 6; i++)
  putchar(*(heart + i));
putchar('\n');
for (i = 0; i < 6; i++)
  putchar(*(head + i));
putchar('\n');
\end{lstlisting}
\vspace{0.1in}

输出结果仍为
\begin{lstlisting}[basicstyle=\ttfamily,showstringspaces=true]
I love
I love
\end{lstlisting}
\end{frame}

\begin{frame}[fragile]\ft{\secname:数组与指针的差别}
(3)、但只有指针可以使用增量运算符。
\begin{lstlisting}[language=c,showstringspaces=true]
while (*head != '\0')
  putchar(*(head++));
\end{lstlisting}
\vspace{0.1in}

输出结果
\begin{lstlisting}[basicstyle=\ttfamily,showstringspaces=true]
I love math!
\end{lstlisting}
\end{frame}

\begin{frame}[fragile]\ft{\secname:数组与指针的差别}
(4)、可让head指针指向heart数组的首元素,即
\begin{lstlisting}[language=c,showstringspaces=true]
head = heart;
\end{lstlisting}
\vspace{0.05in}

但不能这么做:
\begin{lstlisting}[basicstyle=\ttfamily,showstringspaces=true]
heart = head;
\end{lstlisting}
\end{frame}

\begin{frame}[fragile]\ft{\secname:数组与指针的差别}
(5)、可改变heart数组中的信息。
\begin{lstlisting}[language=c,showstringspaces=true]
heart[7] = 'R';  OR  *(heart + 7) = 'R';
\end{lstlisting}
\vspace{0.05in}

因数组元素是变量,但数组名是常量。
\end{frame}

\begin{frame}[fragile]\ft{\secname:数组与指针的差别}
(5)、可改变heart数组中的信息。
\begin{lstlisting}[basicstyle=\ttfamily,showstringspaces=true]
heart[7] = 'R';  OR  *(heart + 7) = 'R';
\end{lstlisting}
\vspace{0.05in}

因数组元素是变量,但数组名是常量。
\end{frame}

\begin{frame}[fragile]\ft{\secname:字符串数组}
\begin{lstlisting}[language=c,showstringspaces=true]
const char * fruit[4] = {
  "Apple",
  "Pear",
  "Orange",
  "Peach"
};
\end{lstlisting}
\vspace{0.05in}

该语句声明了一个长度为4的数组\lstinline| fruit |,每个元素都是指向\lstinline| char |的指针。
\vspace{0.05in}

\begin{itemize}
\item 第一个指针是\lstinline| fruit[0] |,它指向第一个字符串的第一个字符;
\item 第二个指针是\lstinline| fruit[1] |,它指向第二个字符串的第一个字符;
\item ...
\end{itemize}
\end{frame}