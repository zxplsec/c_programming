\section{字符串函数}

\begin{frame}[fragile]\ft{\secname} 
C库提供了许多处理字符串的函数,在\lstinline| string.h |中给出其函数原型。
\begin{enumerate}
\item \lstinline| strlen() |
\item \lstinline| strcat() |
\item \lstinline| strncat() |
\item \lstinline| strcmp() |
\item \lstinline| strncmp()| 
\item \lstinline| strcpy() |
\item \lstinline| strncpy()|
\end{enumerate}
\end{frame}


\begin{frame}[fragile]\ft{\secname:\lstinline| strlen() |}
  \begin{itemize}
  \item \red{函数原型}
    \begin{lstlisting}[language=c,backgroundcolor=\color{red!20}]
int * strlen(const char * s);
    \end{lstlisting}
  \item    \red{功能}
  \item[] 返回字符串\lstinline| s |的长度,即\lstinline| s |中的字符数(不包含空字符)。
  \end{itemize}

\end{frame}

\begin{frame}[fragile,allowframebreaks]\ft{\secname:\lstinline| strlen() |}
\lstinputlisting
[language=c,numbers=left,frame=single]
{ch11/code/test_fit.c}
\end{frame}

\begin{frame}[fragile]\ft{\secname:\lstinline| strlen() |}
\begin{lstlisting}[backgroundcolor=\color{blue!20}]
Hold on to your hats, hackers.
Hold on
Let's look at some more of the string.
to your hats, hackers.
\end{lstlisting}
\end{frame}

\begin{frame}[fragile]\ft{\secname:{\tf strcat()}}
  \begin{itemize}
    \item \red{函数原型}
      \begin{lstlisting}[backgroundcolor=\color{red!20}]
char * strcat(char * s1, 
        const char * s2);
      \end{lstlisting}
    \item \red{功能}
    \item[] 把字符串\lstinline| s2 |(包括空字符)追加到字符串\lstinline| s1 |的结尾,字符串\lstinline| s2 |的第一个字符覆盖字符串\lstinline| s1 |中的空字符。
    \end{itemize}
\end{frame}

\begin{frame}[fragile]\ft{\secname:\lstinline| strcat() |} 
  \begin{itemize}
    \item
      \lstinline| s1 |成为一个新的字符串,\lstinline| s2 |没有改变。\\[0.1in]
    \item 
      \lstinline| strcat() |为\lstinline| char * |类型,返回\lstinline| s1 |。
    \end{itemize}
\end{frame}

\begin{frame}[fragile,allowframebreaks]\ft{\secname:\lstinline| strcat() |}
\lstinputlisting
[language=c,numbers=left,frame=single]
{ch11/code/str_cat.c}
\end{frame}

\begin{frame}[fragile]\ft{\secname:\lstinline| strcat() |}
\begin{lstlisting}[backgroundcolor=\color{blue!20}]
What is your favorite flower?
Rose
Roses smell like old shoes.
s smell like old shoes.
\end{lstlisting}
\end{frame}

\begin{frame}[fragile]\ft{\secname:{\tf strcat()}} 
\lstinline| strcat() |并不检查第一个数组是否能够容纳第二个字符串。如果没有为第一个数组分配足够大的空间,多出的字符溢出到相邻单元时就会出现问题。\vspace{.1in}

该问题可通过使用\lstinline| strncat() |加以解决,该函数需要另一个参数来指明最多允许添加的字符数目。


\end{frame}

\begin{frame}[fragile]\ft{\secname:\lstinline| strncat() |}
  \begin{itemize}
  \item \red{函数原型}
    \begin{lstlisting}[language=c,backgroundcolor=\color{red!20}]
char * strcat(char * s1, 
        const char * s2, 
              size_t n);
    \end{lstlisting}
  \item 
    \red{功能}
  \item[]
    把字符串\lstinline| s2 |的前\lstinline| n |个字符或直到空字符为止的字符追加到字符串\lstinline| s1 |的结尾,\lstinline| s2 |的第一个字符覆盖\lstinline| s1 |中的空字符,总在最后添加一个空字符。\\[0.1in]
  \item
    \lstinline| strncat() |为\lstinline| char * |类型,返回\lstinline| s1 |。
  \end{itemize}
\end{frame}

\begin{frame}[fragile,allowframebreaks]\ft{\secname:\lstinline| strncat() |}
  \lstinputlisting
  [language=c,numbers=left,frame=single]
  {ch11/code/join_chk.c}
\end{frame}


\begin{frame}[fragile]\ft{\secname:\lstinline| strncat() |}
\begin{lstlisting}[language=c]
What is your favorite flower?
Rose
Roses smell like old shoes.
What is your favorite bug?
Aphid
Aphids smell
\end{lstlisting}

\end{frame}

\begin{frame}[fragile]\ft{\secname:\lstinline| strcmp() |} 
  \begin{itemize}
  \item \red{函数原型}
    \begin{lstlisting}[language=c,backgroundcolor=\color{red!20}]
int strcmp(const char * s1, 
           const char * s2);
    \end{lstlisting}
  \item
    \red{功能}
  \item[]
    比较字符串\lstinline| s1 |和\lstinline| s2 |。若字符串相同,则返回\lstinline| 0 |;否则就比较两个字符串的第一个不匹配的字符对(使用ASCII码进行比较)。
  \end{itemize}
\end{frame}

\begin{frame}[fragile]\ft{\secname:\lstinline| strcmp() |} 
  \begin{itemize}
  \item
    若第一个字符串小于第二个则返回一个负数;
  \item
    若第一个字符串较大就返回一个整数。
  \end{itemize}
\end{frame}

\begin{frame}[fragile,allowframebreaks]\ft{\secname:\lstinline| strcmp() |}
\lstinputlisting
[language=c,numbers=left,frame=single]
{ch11/code/compare.c}
\end{frame}


\begin{frame}[fragile]\ft{\secname:\lstinline| strcmp() |}
我的系统的输出结果为
\begin{lstlisting}[backgroundcolor=\color{red!20}]  
strcmp("A", "A") is 0
strcmp("A", "B") is -1
strcmp("B", "A") is 1
strcmp("C", "A") is 2
strcmp("Z", "a") is -7
strcmp("apples", "apple") is 115
\end{lstlisting}
\end{frame}


\begin{frame}[fragile]\ft{\secname:\lstinline| strcmp() |}
而有些系统输出结果为
\begin{lstlisting}[backgroundcolor=\color{red!20}]  
strcmp("A", "A") is 0
strcmp("A", "B") is -1
strcmp("B", "A") is 1
strcmp("C", "A") is 1
strcmp("Z", "a") is -1
strcmp("apples", "apple") is 1
\end{lstlisting}
\end{frame}

\begin{frame}[fragile]\ft{\secname:\lstinline| strncmp() |}
\begin{itemize}
\item \red{函数原型}
\begin{lstlisting}[language=c,backgroundcolor=\color{red!20}]
  int strncmp(const char * s1, 
              const char * s2,
              size_t n);
\end{lstlisting}
\item \red{功能}
\item[]
  比较字符串\lstinline| s1 |和\lstinline| s2 |的前\lstinline| n |个字符或直到第一个空字符为止。返回结果与\lstinline| strcmp() |类似。  
\end{itemize}    
\end{frame}

\begin{frame}[fragile]\ft{\secname:\lstinline| strncmp() |}
如果想搜索以\lstinline| "astro" |开头的字符串,可以限定搜索前5个字符。
\end{frame}

\begin{frame}[fragile,allowframebreaks]\ft{\secname:\lstinline| strncmp() |}
\lstinputlisting
[language=c,numbers=left,frame=single]
{ch11/code/starsrch.c}
\end{frame}

\begin{frame}[fragile]\ft{\secname:\lstinline| strncmp() |}
\begin{lstlisting}[language=c]
Found: astronomy
Found: astrophysics
The list contained 2 words beginning with astro.
\end{lstlisting}
\end{frame}

\begin{frame}[fragile]\ft{\secname:\lstinline| strcpy() |} 
\begin{itemize}
\item \red{函数原型}
\begin{lstlisting}[language=c,backgroundcolor=\color{red!20}]
char * strcpy(char * s1, 
        const char * s2);
\end{lstlisting}
\item 
  \red{功能}
\item[]把字符串\lstinline| s2 |(包括空字符)复制到\lstinline| s1 |指向的位置,返回\lstinline| s1 |。
\end{itemize}
\end{frame}

\begin{frame}[fragile]\ft{\secname:\lstinline| strcpy() |} 
\begin{itemize}
\item
\lstinline| s2 |称为源(source)字符串,\lstinline| s1 |称为目标(target)字符串。
\\[0.1in]
\item
指针\lstinline| s2 |可以是一个已声明的指针、数组名或字符串常量。\\[0.1in]
\item
指针\lstinline| s1 |应指向空间大到足够容纳字符串\lstinline| s2 |的数组。\\[0.1in]
\item[]
\red{谨记:声明一个数组将为数据分配存储空间,而声明一个指针值为一个地址分配存储空间。}
\end{itemize}
\end{frame}

\begin{frame}[fragile,allowframebreaks]\ft{\secname:\lstinline| strcpy() |}
\lstinputlisting
[language=c,numbers=left,frame=single]
{ch11/code/copy1.c}

\end{frame}


\begin{frame}[fragile]\ft{\secname:\lstinline| strcpy() |}
\begin{lstlisting}[backgroundcolor=\color{blue!20}]
Enter 5 words beginning with q:
quit
quarter
quite
quotient
nomore
nomore doesn't begin with q!
quiz
Here are the words accepted:
quit
quarter
quite
quotient
quiz
\end{lstlisting}

\end{frame}

\begin{frame}[fragile]\ft{\secname:\lstinline| strcpy() |} 
\lstinline| strcpy() |还有两个重要的属性:\vspace{0.1in}
\begin{itemize}
\item 它是\lstinline| char * |类型,返回第一个参数的值; \\[0.1in]
\item 第一个参数不需要指向数组的开始,这样就可以复制到目标字符串的指定位置。
\end{itemize}
\end{frame}

\begin{frame}[fragile,allowframebreaks]\ft{\secname:\lstinline| strcpy() |}
\lstinputlisting
[language=c,numbers=left,frame=single]
{ch11/code/copy2.c}
\end{frame}

\begin{frame}[fragile]\ft{\secname:\lstinline| strcpy() |}
\begin{lstlisting}[backgroundcolor=\color{blue!20}]
beast
Be the best that you can be.
Be the beast
beast
\end{lstlisting}

\end{frame}

\begin{frame}[fragile]\ft{\secname:\lstinline| strncpy() |} 
\begin{itemize}
\item \red{函数原型}
  \begin{lstlisting}[language=c,backgroundcolor=\color{red!20}]
    char * strcpy(char * s1, const char * s2,
                  size_t n);
\end{lstlisting}
\item 
  \red{功能}
\item[]
  把字符串\lstinline| s2 |的前\lstinline| n |个字符或直到空字符为止的字符复制到\lstinline| s1 |指向的位置,第三个参数\lstinline| n |用于指明最大可复制的字符数。
\end{itemize}
\end{frame}

\begin{frame}[fragile]\ft{\secname:\lstinline| strncpy() |} 
\begin{itemize}
\item 
若源字符串的字符数小于\lstinline| n |,则整个字符串都被复制过来,包括空字符;
\item
复制的字符数不能超过\lstinline| n |,必须要给空字符留位置。处于这个原因,调用该函数时,\lstinline| n |一般设置为目标数组长度减1。
\item
函数返回\lstinline| s1 |。
\end{itemize}
\end{frame}


\begin{frame}[fragile,allowframebreaks]\ft{\secname:\lstinline| strncpy() |}
\lstinputlisting
[language=c,numbers=left,frame=single]
{ch11/code/copy3.c}
\end{frame}


\begin{frame}[fragile]\ft{\secname:\lstinline| strncpy() |}
\begin{lstlisting}[backgroundcolor=\color{blue!20}]
Enter 5 words begin with q:
quack
quadratic
quisling
quota
quagga
Here are the words accepted:
quack
quadra
quisli
quota
quagga
\end{lstlisting}
\end{frame}


\begin{frame}[fragile]\ft{\secname:\lstinline| sprintf() |}
\lstinline| sprintf() |在{\tf stdio.h}中声明。 \vspace{.05in}

\begin{itemize}
\item 
作用同{\tf printf()}一样,但它写到字符串中而不是输出显示。 \\[0.1in]
\item
第一个参数是目标字符串的地址,其余参数同{\tf printf()}。
\end{itemize}
\end{frame}


\begin{frame}[fragile,allowframebreaks]\ft{\secname:\lstinline| sprintf() |}
\lstinputlisting
[language=c,numbers=left,frame=single]
{ch11/code/format.c}
\end{frame}


\begin{frame}[fragile]\ft{\secname:\lstinline| sprintf() |}
\begin{lstlisting}[backgroundcolor=\color{blue!20}]
Enter your first name:
warning: this program uses gets(), which is unsafe.
Teddy
Enter your last name:
Bear
Enter your prize money:
2000
Bear, Teddy              : $2000.00

\end{lstlisting}

\pause \rule{\textwidth}{0.3mm}\vspace{0.3mm}

\lstinline| sprintf() |获取输入,并把输入格式化为标准形式后存放在字符串{\tf format}中。

\end{frame}