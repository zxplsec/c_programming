\section{字符串输入}

% \begin{frame}[fragile]\ft{\secname} 
% 把一个字符串读到程序中,必须首先预留存储字符串的空间,然后通过输入函数来获取该字符串。
% \end{frame}

\begin{frame}[fragile]\ft{\secname:创建存储空间} 
在读入字符串之前,必须分配足够大的存储区来存放读入的字符串。\red{不要指望计算机读取字符串时自动计算其长度,然后为其分配空间。} 例如,如果你这么做
\begin{lstlisting}[language=c,showstringspaces=true]
char * name;
scanf("%s", name);
\end{lstlisting}
虽然编译会通过,但后果可能会非常严重,因{\tf name}可能指向任何地方。
\end{frame}

\begin{frame}[fragile]\ft{\secname:创建存储空间} 
创建存储空间,有两种方式:
\begin{itemize}
\item 
最简单的方法就是在声明中明确指出数组大小:
\begin{lstlisting}[language=c,showstringspaces=true]
char name[81];
\end{lstlisting}
\vspace{0.05in}
\item 
另一种方式就是使用C库中分配存储空间的函数,如\lstinline| malloc() |。
\end{itemize}
\end{frame}

\begin{frame}[fragile]\ft{\secname:创建存储空间} 
为字符串预留空间后,就可以读取字符串了。C库提供了三种读取字符串的函数:
\begin{itemize} 
\item \lstinline| scanf() |
\item \lstinline| gets() |
\item \lstinline| fgets() |
\end{itemize}
\end{frame}

\begin{frame}[fragile]\ft{\secname:\lstinline| gets() |} 
函数\lstinline| gets() |从键盘获取一个字符串,其函数原型为
\begin{lstlisting}
char * gets ( char * str );
\end{lstlisting}
\vspace{0.2in}

\red{它读取换行符之前(不包括换行符)的所有字符,在这些字符后添加一个空字符,然后把这个字符串交给调用它的程序。}

\vspace{0.2in}
它将\red{读取换行符并将其丢弃},这样下一次读取就会在新的一行开始。
\end{frame}

\begin{frame}[fragile]\ft{\secname:\lstinline| gets() |} 
\lstinputlisting
[language=c,numbers=left,frame=single]
{ch11/code/name1.c}
\end{frame}

\begin{frame}[fragile]\ft{\secname:\lstinline| gets() |} 
\begin{lstlisting}[]
Hi, what's your name?
warning: this program uses gets(), which is unsafe.
Ming Li
Nice name, Ming Li.
\end{lstlisting}

\rule{\textwidth}{0.3mm} \pause \vspace{.1in}
由于\lstinline| gets() |不检查目标数组是否能够容纳输入,故编译器会给出警告,提醒用户使用\lstinline| gets() |会不安全。
\end{frame}

\begin{frame}[fragile]\ft{\secname:\lstinline| gets() |} 
\lstinputlisting
[language=c,showstringspaces=true,numbers=left,frame=single]
{ch11/code/name2.c}
\end{frame}

\begin{frame}[fragile]\ft{\secname:\lstinline| gets() |} 
\begin{lstlisting}[]
Hi, what's your name?
warning: this program uses gets(), which is unsafe.
Xiaoping Zhang
Xiaoping Zhang? Ah! Xiaoping Zhang!
\end{lstlisting}
\end{frame}


\begin{frame}[fragile]\ft{\secname:\lstinline| fgets() |} 
由于\lstinline| gets() |不检查目标数组是否能够容纳输入,多出来的字符简单溢出到相邻的内存区。\lstinline| fgets() |改进了该不足,它让用户指定最大读入字符数。\vspace{.1in}

\lstinline| fgets() |的函数原型为
\begin{lstlisting}
char * fgets (char * str, int num, FILE * stream);
\end{lstlisting}
\end{frame}

\begin{frame}[fragile]\ft{\secname:\lstinline| fgets() |} 
\lstinline| gets() |与\lstinline| fgets() |的三个不同: \vspace{.1in}

\begin{enumerate}
\item \lstinline| fgets() |用第二个参数来说明最大读入字符数。
\item[] 若该参数值为{\tf n},则\lstinline| fgets() |会读取最多{\tf n-1}个字符或读完第一个换行符为止。\\[0.1in]
\item 若\lstinline| fgets() |读取换行符,并将它存到字符串中,而\lstinline| gets() |读取并丢弃它。\\[0.1in]
\item \lstinline| fgets() |用第三个参数来说明读哪一个文件,其中\lstinline| FILE * |表示文件指针,第13章会专门介绍。从键盘输入时,可以使用\lstinline| stdin |作为该参数。
\end{enumerate}
\end{frame}

\begin{frame}[fragile]\ft{\secname:\lstinline| fgets() |} 
\lstinputlisting
[language=c,showstringspaces=true,numbers=left,frame=single]
{ch11/code/name3.c}
\end{frame}

\begin{frame}[fragile]\ft{\secname:\lstinline| gets() |} 
\begin{lstlisting}[]
Hi, what's your name?
Michael Jordan
Michael Jordan
? Ah! Michael Jordan
!
\end{lstlisting}
\rule{\textwidth}{0.3mm} \pause \vspace{.1in}

该显示表明了\lstinline| fgets() |的不足,原因在于\lstinline| fgets() |把换行符存储到字符串中,导致每次显示字符串时就会显示换行符。 
\end{frame}

\begin{frame}[fragile]\ft{\secname:\lstinline| scanf() |} 

可用带\lstinline| %s |的\lstinline| scanf() |来读取一个字符串。 \vspace{.1in}

\lstinline| scanf() |与\lstinline| gets() |的主要差别在于字符串何时结束。\vspace{.1in}

\lstinline| scanf() |更基于获取单词而不是获取字符串,而\lstinline| gets() |会读取所有字符,直到遇到第一个换行符为止。
\end{frame}

\begin{frame}[fragile]\ft{\secname:\lstinline| scanf() |} 

\lstinline| scanf() |使用以下两种方法决定输入何时结束。(但无论哪种方法,都从第一个非空白字符开始读取。)

\begin{enumerate}
\item 若使用占位符\lstinline| %s |,字符串读到(但不包括)下一个空白字符。\\[0.1in]
\item 若指定字段宽度,如\lstinline| %10s |,则\lstinline| scanf() |会读入\lstinline| 10 |个字符或直到遇到第一个空白字符。
\end{enumerate}
\end{frame}

\begin{frame}[fragile]\ft{\secname:\lstinline| scanf() |}
  \begin{footnotesize}
\begin{table}
\centering
% \caption{以\lstinline| _ |代表空格}
\begin{tabular}{c|l|l|l}\hline
输入语句&原始输入队列&name中的内容&剩余队列\\\hline
  \lstinline|scanf("%s", name);|  & \lstinline|Fleebert Hup| & \lstinline|Fleebert| & \lstinline| Hup| \\\hline
  \lstinline| scanf("%5s", name);| & \lstinline|Fleebert Hup| & \lstinline|Fleeb|    & \lstinline|ert Hup|\\\hline
  \lstinline| scanf("%5s", name);| & \lstinline|Ann Ular| & \lstinline|Ann| & \lstinline| Ular|\\\hline
\end{tabular}
\end{table}
  \end{footnotesize}
\end{frame}

\begin{frame}[fragile]\ft{\secname:\lstinline| scanf() |} 
\lstinputlisting
[language=c,showstringspaces=true,numbers=left,frame=single]
{ch11/code/scan_str.c}
\end{frame}

\begin{frame}[fragile]\ft{\secname:\lstinline| scanf() |} 
\begin{lstlisting}[backgroundcolor=\color{blue!20}]
Please enter 2 names.
Jesse Jukes
I read the 2 names Jesse and Jukes.
\end{lstlisting}

\begin{lstlisting}[backgroundcolor=\color{blue!20}]
Please enter 2 names.
Liza Applebottham
I read the 2 names Liza and Applebotth.
\end{lstlisting}

\begin{lstlisting}[backgroundcolor=\color{blue!20}]
Please enter 2 names.
Portensia Callowit
I read the 2 names Porte and nsia.
\end{lstlisting}

\end{frame}