\section{指针}
\begin{frame}[fragile]\ft{\secname}
指针是一个变量,其值为一个地址。
\end{frame}

\begin{frame}[fragile]\ft{\secname}
假如你把某个指针变量命名为\lstinline| ptr |,就可以使用以下语句
\begin{lstlisting}[backgroundcolor=\color{blue!10}]
ptr = &var;
\end{lstlisting}
即把变量\lstinline| var |的地址赋给指针变量\lstinline| ptr |,称为\red{\lstinline| ptr |“指向”\lstinline| var |}。
\pause \vspace{0.1in}

\lstinline| ptr |和\lstinline| &var |的区别在于,前者为一变量,后者是一个常量。
\end{frame}

\begin{frame}[fragile]\ft{\secname}
\lstinline| ptr |可以指向任何地址,即可以把任何地址赋值给\lstinline| ptr |:
\begin{lstlisting}[backgroundcolor=\color{blue!10}]
ptr = &var1;
\end{lstlisting}
\end{frame}

\begin{frame}[fragile]\ft{\secname}
  \begin{wenti}
    如何创建一个指针变量?
  \end{wenti}
  \pause \vskip.1in
  
  首先需要声明其类型。在介绍其类型之前,我们先介绍一个新运算符\lstinline| * |。
\end{frame}

\begin{frame}[fragile]\ft{\secname:间接运算符或取值运算符:*}
假定\lstinline| ptr |指向\lstinline| var |,即
\begin{lstlisting}[backgroundcolor=\color{blue!10}]
ptr = &var;
\end{lstlisting}
就可以用间接运算符\lstinline| * |来获取\lstinline| var |中存放的数值:
\begin{lstlisting}[backgroundcolor=\color{blue!10}]
value = *ptr;
\end{lstlisting}
\pause \vspace{0.1in}

\begin{minipage}{0.4\textwidth}
\begin{lstlisting}[backgroundcolor=\color{blue!10}]
ptr = &var;
value = *ptr;
\end{lstlisting}
\end{minipage}
~~~$\Longleftrightarrow$~~~
\begin{minipage}{0.4\textwidth}
\begin{lstlisting}[backgroundcolor=\color{blue!10}]
value = var;
\end{lstlisting}
\end{minipage}
\end{frame}

\begin{frame}[fragile]\ft{\secname:指针声明}
能否如以下方式声明一个指针?
\begin{lstlisting}[backgroundcolor=\color{blue!10}]
pointer ptr;
\end{lstlisting}
\pause \vspace{0.1in}

\begin{center}
{\Large NO!}
\end{center}
\pause\vspace{0.1in}

\begin{center}
{\Large Why?} 
\end{center}
\end{frame}

\begin{frame}[fragile]\ft{\secname:指针声明}
原因在于,仅声明一个变量为指针是不够的,还需说明指针所指向变量的类型。
\vspace{0.1in}

\begin{itemize}
\item 不同的变量类型占用的存储空间大小不同,而有些指针需要知道变量类型所占用的存储空间。\\[0.1in]
\item 程序也需要知道地址中存储的是何种数据。
\end{itemize}
\end{frame}

\begin{frame}[fragile]\ft{\secname:指针声明}
  \begin{lstlisting}[backgroundcolor=\color{blue!10}]
int * pi;           // pi is a pointer to an integer variable
char * pc;          // pc is a pointer to a character variable
float * pf, * pg;   // pf, pg are pointers to float variables
\end{lstlisting} \pause \vspace{0.1in}

\begin{itemize}
\item
类型标识符表明了被指向变量的类型,\lstinline| * |表示该变量为一个指针。\\[0.1in]
\item 
声明\lstinline| int * pi; |的含义是:\lstinline| pi |是一个指针,且\lstinline| *pi |是\lstinline| int |类型的。\\[0.1in]
\item 
\lstinline| * |与指针名之间的空格可选。通常在声明中使用空格,在指向变量时将其省略。
\end{itemize}
\end{frame}

\begin{frame}[fragile]\ft{\secname:指针声明}
\begin{itemize}
\item
\lstinline| pc |所指向的值\lstinline| (*pc) |是\lstinline| char |类型的,而\lstinline| pc |本身是“指向\lstinline| char |的指针”类型。\\[0.1in]
\item 
\lstinline| pc |的值是一个地址,在大多数系统中,它由一个无符号整数表示。但这并不表示可以把指针看做是整数类型。\\[0.1in]
\item 
一些处理整数的方法不能用来处理指针,反之亦然。如两个整数可以相乘,但指针不能。\\[0.1in]
\item 
指针是一种新的数据类型,而不是一种整数类型。
\end{itemize}
\end{frame}

\begin{frame}[fragile]\ft{\secname:使用指针在函数间通信}
这里将重点介绍如何通过指针解决函数间的通信问题。
\end{frame}

\begin{frame}[fragile,allowframebreaks]\ft{\secname:使用指针在函数间通信}
  \lstinputlisting
  [language=c,numbers=left,frame=single]  
  {ch09/code/swap3.c}
\end{frame}

\begin{frame}[fragile]\ft{\secname}
\begin{lstlisting}[backgroundcolor=\color{blue!10}]
Before: x =  5, y = 10.
After : x = 10, y =  5.
\end{lstlisting}
\pause \vspace{0.1in}

\begin{center}
{\Large Oh Ye!!!}
\end{center}
\end{frame}

\begin{frame}[fragile]\ft{\secname:使用指针在函数间通信}
\begin{itemize}
\item 
函数调用语句为
\begin{lstlisting}[backgroundcolor=\color{blue!10}]
swap(&x, &y);
\end{lstlisting}
故函数传递的是\lstinline| x |和\lstinline| y |的地址而不是它们的值。\\[0.15in]
\item 
函数声明为
\begin{lstlisting}[backgroundcolor=\color{blue!10}]
void swap(int * u, int * v);
\end{lstlisting}
也可简化为
\begin{lstlisting}[backgroundcolor=\color{blue!10}]
void swap(int *, int *);
\end{lstlisting}
\end{itemize}
\end{frame}

\begin{frame}[fragile]\ft{\secname:使用指针在函数间通信}
\begin{itemize}
\item
函数体中声明了一个临时变量!
\begin{lstlisting}[backgroundcolor=\color{blue!10}]
int temp;
\end{lstlisting}
\item
为了把\lstinline| x |的值存在\lstinline| temp |中,需使用以下语句
\begin{lstlisting}[backgroundcolor=\color{blue!10}]
temp = *u; 
\end{lstlisting}
因\lstinline| u |的值为\lstinline| &x |,即\lstinline| x |的地址,故\lstinline| *u |代表了\lstinline| x |的值。\\[0.1in]
\item 
同理,为了把\lstinline| y |的值赋给\lstinline| x |,需用以下语句
\begin{lstlisting}[backgroundcolor=\color{blue!10}]
*u = *v;
\end{lstlisting}
\end{itemize}
\end{frame}

\begin{frame}[fragile]\ft{\secname:使用指针在函数间通信}
该例中,用一个函数实现了\lstinline| x |和\lstinline| y |的数值交换。\vspace{0.1in}

\begin{itemize}
\item
首先函数使用\lstinline| x |和\lstinline| y |的地址作为参数,这使得它可以访问\lstinline| x |和\lstinline| y |变量。\\[0.1in]
\item 
通过使用指针和运算符\lstinline| * |,函数可以获得相应存储地址的数据,从而就可以改变这些数据。
\end{itemize}
\end{frame}

\begin{frame}[fragile]\ft{\secname:使用指针在函数间通信}
通常情况下,可以把变量的两类信息传递给一个函数,即传值与传址。
\end{frame}

\begin{frame}[fragile]\ft{\secname:传值}
\begin{itemize}
\item 调用方式为
\begin{lstlisting}[backgroundcolor=\color{blue!10}]
function1(x);
\end{lstlisting}
\item 定义方式为
\begin{lstlisting}[backgroundcolor=\color{blue!10}]
int function1(int num)
\end{lstlisting}
\item 适用范围:使用函数进行数据计算等操作。
\end{itemize}

\end{frame}

\begin{frame}[fragile]\ft{\secname:传址}
\begin{itemize}
\item 调用方式为
\begin{lstlisting}[backgroundcolor=\color{blue!10}]
function2(&x);
\end{lstlisting}
\item 定义方式为
\begin{lstlisting}[backgroundcolor=\color{blue!10}]
int function2(int * ptr)
\end{lstlisting}
\item 适用范围:改变调用函数中的多个变量的值。
\end{itemize}

\end{frame}
