\section{使程序可读的技巧}
\begin{frame}[fragile]\ft{\secname}
\begin{itemize}
\item 变量命名时做到“见其名知其意”;\\[0.1in]
\item 合理使用注释;\\[0.1in]
\item 使用空行分隔一个函数的各个部分,如声明、操作等;\\[0.1in]
\item 每条语句用一行。注意,C允许把多条语句放在同一行或一条语句放多行。
\end{itemize}
\end{frame}


\begin{frame}[fragile]\ft{\secname}
\lstinputlisting[language=c]{ch02/code/mile_km.c}
\end{frame}


\begin{frame}[fragile]\ft{\secname}
\begin{itemize}
	\item 建议在程序开始处,用一个注释说明文件名和程序的作用,这对以后浏览或打印程序很有帮助。\\[0.1in]
	\item 多个声明 \\[0.1in]
	
\begin{minipage}{.4\textwidth}
\begin{lstlisting}[language=c]
float mile, km;
\end{lstlisting}
\end{minipage}	$~~\Leftrightarrow~~$
\begin{minipage}{.4\textwidth}
\begin{lstlisting}[language=c]
float mile;
float km;
\end{lstlisting}
\end{minipage}

\item 输出多个值

\begin{itemize}
	\item 第一个printf语句用了两个占位符:第一个\%d为mile占位,第二个\%d为km占位;圆括号中有三个参数,之间用逗号隔开。\vspace{0.1in}
	\item 
	第二个printf语句说明输出的值可以是一个表达式。
\end{itemize}

\end{itemize}
 

\end{frame}
 