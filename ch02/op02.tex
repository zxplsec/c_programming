\section{上机操作}


\begin{frame}[fragile]\ft{\secname}

  直接运行以下程序,观察运行结果
  \lstinputlisting{ch02/code/debug1.c}
  \pause 
  然后在调试环境下在运行一次,观察运行结果。

\end{frame}

\begin{frame}[fragile]\ft{\secname}
  运行以下程序,观察运行结果
  \lstinputlisting{ch02/code/debug2.c}

  \pause 
  以该程序为例,简单介绍C语言如何进行调试。
\end{frame}

\begin{frame}[fragile]\ft{\secname}
  在编辑器中键入以下代码
  \lstinputlisting{ch02/code/debug3.c}
  在调试环境下查看各个变量的值。
\end{frame}


\begin{frame}[fragile]\ft{\secname}
  编写程序,求等差数列之和,请在键盘中输入该等差数列的首项、公差和项数。
  注意,请保持运行结果形如
  \begin{lstlisting}
Please input the start value: 3
Please input the space: 2
Please input the number of items: 6
3 + 5 + 7 + 9 + 11 + 13 = 48
  \end{lstlisting}

\end{frame}


\begin{frame}[fragile]\ft{\secname}
  重写以上程序,要求将等差数列求和的过程封装成函数。
  注意,请保持运行结果形如
  \begin{lstlisting}
Please input the start value: 3
Please input the space: 2
Please input the number of items: 6
3 + 5 + ... + 13 = 48.
\end{lstlisting}
\end{frame}
