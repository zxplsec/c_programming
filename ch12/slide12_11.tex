\section{枚举类型}

\begin{frame}[fragile]\ft{\secname}
  可使用枚举类型(enumerated type)声明代表整数常量的符号名称。用关键字\verb|enum|,可创建一个新“类型”并指定它可以具有的值。
实际上,\verb|enum|常量是\verb|int|类型,故在使用\verb|int|类型的任何地方都可使用它。\vspace{0.1in}

枚举类型的目的是为了提高程序可读性,其语法与结构相同。
\end{frame}

\begin{frame}[fragile]\ft{\secname}
  声明方式如下:
  \begin{lstlisting}[language=c,backgroundcolor=\color{red!20}]
enum spectrum {red, orange, yellow, green, blue};
enum spectrum color;
  \end{lstlisting}
  \begin{itemize}
  \item 第一个声明设置\verb|spectrum|为标记名,从而允许你把\verb|enum spectrum|作为一个类型名使用。\\[0.1in]
  \item 第二个声明使得\verb|color|成为该类型的一个变量。花括号中的标识符枚举了\verb|spectrum|变量可能有的值。
  \end{itemize}
\end{frame}

\begin{frame}[fragile]\ft{\secname}
可以使用以下语句:
  \begin{lstlisting}[language=c,backgroundcolor=\color{red!20}]
int c;
color = blue;
if (color == yellow)
  ...
for (color = red; color <= blue; color++)
  ...
  \end{lstlisting}
实际上,为\verb|spectrum|枚举的常量在\verb|0|到\verb|5|之间。
\end{frame}

\begin{frame}[fragile]\ft{\secname}
执行以下代码:
  \begin{lstlisting}[language=c,backgroundcolor=\color{red!20}]
printf("red = %d, orange = %d\n", red, orange);
  \end{lstlisting}
结果为
  \begin{lstlisting}[language=c,backgroundcolor=\color{red!20}]
red = 0, orange = 1
  \end{lstlisting}
\verb|red|为一个代表整数\verb|0|的命名常量,其他标识符分别是代表\verb|1|到\verb|5|的命名常量。
\end{frame}

\begin{frame}[fragile]\ft{\secname}
默认时,枚举列表中的常量被指定为整数值\verb|0、1、2|等。故,以下声明使得\verb|nina|具有值\verb|3|:
  \begin{lstlisting}[language=c,backgroundcolor=\color{red!20}]
enum kids {nippy, slats, skippy, nina, liz};
  \end{lstlisting}

\end{frame}

\begin{frame}[fragile]\ft{\secname}
  \begin{itemize}
  \item 
也可指定常量具有特定的整数值:
  \begin{lstlisting}[language=c,backgroundcolor=\color{red!20}]
enum levels {low = 100, medium = 500, high = 2000};
  \end{lstlisting}
\item 若只对一个常量赋值,而没对后面的常量赋值,则后面的常量会被赋予后续的值:
  \begin{lstlisting}[language=c,backgroundcolor=\color{red!20}]
enum feline {cat, lynx = 10, puma, tiger};
  \end{lstlisting}
则\verb|cat|的默认值为\verb|0|,\verb|lynx|、\verb|puma|、\verb|tiger|的默认值分别为\verb|10、11、12|。
  \end{itemize}
\end{frame}

\begin{frame}[fragile]\ft{\secname:enum的用法}
枚举类型的目的是为了提高程序可读性。如果是处理颜色,采用\verb|red|和\verb|blue|要比使用\verb|0|和\verb|1|更显而易见。
\end{frame}

\begin{frame}[fragile,allowframebreaks]\ft{\secname:enum的用法}
\lstinputlisting
[language=c,numbers=left,frame=single]
{ch12/code/enum.c}
\end{frame}


\begin{frame}[fragile,allowframebreaks]\ft{\secname:enum的用法}
  \begin{lstlisting}[backgroundcolor=\color{blue!20}]
Enter a color (empty line to quit):
orange
Poppies are orange.
Next color, please (empty line to quit): 
blue
Bluebells are blue.
Next color, please (empty line to quit): 
red
Roes are red.
Next color, please (empty line to quit): 
sdf
I don't know about the color sdf.
Next color, please (empty line to quit): 

Goodbye!
\end{lstlisting}
\end{frame}

