\section{嵌套结构体}
\begin{frame}[fragile]\ft{\secname}
有时候,一个结构体中嵌套另一个结构体是很方便的。\vspace{0.1in}

如\verb|Shalala|创建一个有关他的朋友的信息的结构体。
该结构体的一个成员是朋友的姓名,而姓名本身就可以标识为一个结构体,其中包含名和姓两个成员。
\end{frame}


\begin{frame}[fragile,allowframebreaks]\ft{\secname}
\lstinputlisting
[language=c,numbers=left,frame=single]
{ch12/code/friend.c}
\end{frame}


\begin{frame}[fragile]\ft{\secname}
    \begin{lstlisting}[backgroundcolor=\color{blue!20}]
Dear Ewen, 

  Thank your for the wonderful evening, Ewen.
You certainly prove that a personality coach
is a special kind of guy. We must get together
over a delicous grilled salmonand have a few laughs.

                           See you soon, 
                           Shalala
\end{lstlisting}
\end{frame}

\begin{frame}[fragile]\ft{\secname}
  \begin{itemize}
  \item 对嵌套结构的成员进行访问,只需使用两次点运算符:
    \begin{lstlisting}[basicstyle=\ttfamily]
fellow.handle.first      
    \end{lstlisting}
  \end{itemize}
\end{frame}
