\section{建立结构体声明}
\begin{frame}[fragile]\ft{\secname}
  结构体声明(structure declaration)是描述结构如何组合的主要方法,形如
  \begin{lstlisting}[language=c,backgroundcolor=\color{red!20}]
struct book
{
  char title[MAXTITL];
  char author[MAXAUTL];
  float value;
};    
  \end{lstlisting}
该声明描述了一个结构体,它由两个\verb|char|数组和一个\verb|float|变量组成。该声明并没有创建一个实际的数据对象,而是描述了组成这类对象的元素。  
\end{frame}

\begin{frame}[fragile]\ft{\secname}
  \begin{itemize}
  \item 关键字\verb|struct|后是一个可选的标记\verb|book|,是用于引用该结构体的快速标记。以下声明
  \begin{lstlisting}[basicstyle=\ttfamily]
struct book library;
  \end{lstlisting}
把\verb|library|声明为一个使用\verb|book|结构体的结构体变量。 \\[0.1in]
\item 接下来是用一对花括号扩起来的结构体成员列表。每个成员变量都用它自己的声明来描述,用一个分号来结束描述。每个成员可以是任意C类型,甚至可以是其他结构体。 \\[0.1in]
\item 结束花括号后的分号表示结构体声明的结束。
  \end{itemize}
\end{frame}

\begin{frame}[fragile]\ft{\secname}
  结构体声明可以放在任何函数的外面,也可以放在一个函数内部。\vspace{.1in}

  \begin{itemize}
  \item 
    如果是内部声明,则该结构体只能在该函数内部使用。\\[0.1in]
  \item
    如果是外部声明,则它可以被本文件该声明后的所有函数使用。
  \end{itemize}
\end{frame}

