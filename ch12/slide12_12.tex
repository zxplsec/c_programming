\section{typedef简介}

\begin{frame}[fragile]\ft{\secname}
\verb|typedef|工具是一种高级数据特性,它使你能够为某一种类型创建自己的名字。它与\verb|#define|相似,但有如下不同
\begin{itemize}
\item 与\verb|#define|不同,\verb|typedef|给出的符号名称仅限于对类型,而不是对值。\\[0.1in]
\item \verb|typedef|的解释由编译器,而不是预处理器执行。\\[0.1in]
\item 虽然它的范围有限,但在其受限范围内,\verb|typedef|比\verb|#define|更灵活。
\end{itemize}
\end{frame}

\begin{frame}[fragile]\ft{\secname}
观察代码
  \begin{lstlisting}[language=c,backgroundcolor=\color{red!20}]
typedef unsigned char BYTE;
BYTE x, y[10], * z;
  \end{lstlisting}
  \begin{itemize}
  \item 该代码为\verb|unsigned char|创建了一个名字\verb|BYTE|,接下来便可用\verb|BYTE|来定义变量。
  \item 该定义的作用域取决于\verb|typedef|语句所在的位置。如果定义在一个函数内部,则其作用域是局部的,限定在该函数内。若定义在函数外部,则具有全局作用域。
  \end{itemize}
\end{frame}

\begin{frame}[fragile]\ft{\secname}
  \begin{lstlisting}[language=c,backgroundcolor=\color{red!20}]
typedef char * STRING;
  \end{lstlisting}
使\verb|STRING|成为\verb|char|指针的标识符。因此
  \begin{lstlisting}[language=c,backgroundcolor=\color{red!20}]
STRING name, sign;
  \end{lstlisting}
的意思是
  \begin{lstlisting}[language=c,backgroundcolor=\color{red!20}]
char * name, * sign;
  \end{lstlisting} 
\end{frame}

\begin{frame}[fragile]\ft{\secname}
若这样做:
  \begin{lstlisting}[language=c,backgroundcolor=\color{red!20}]
#define STRING char *;
  \end{lstlisting}
则
  \begin{lstlisting}[language=c,backgroundcolor=\color{red!20}]
STRING name, sign;
  \end{lstlisting}
的意思是
  \begin{lstlisting}[language=c,backgroundcolor=\color{red!20}]
char * name, sign;
  \end{lstlisting}
\end{frame}

\begin{frame}[fragile]\ft{\secname}
也可对结构使用\verb|typedef|:
  \begin{lstlisting}[language=c,backgroundcolor=\color{red!20}]
typedef struct complex {
  float real;
  float imag;
} COMPLEX;
  \end{lstlisting}
  这样你就可以使用\verb|COMPLEX|来代替\verb|struct complex|来表示复数。\vspace{0.1in}


  {使用\verb|typedef|的原因之一是为经常出现的类型创建一个方便的、可识别的名称。}

\end{frame}

\begin{frame}[fragile]\ft{\secname}
使用\verb|typedef|来命名一个结构类型时,可省去结构的标记
  \begin{lstlisting}[language=c,backgroundcolor=\color{red!20}]
typedef struct  {
  double x;
  double y;
} vector;
  \end{lstlisting}
\end{frame}

\begin{frame}[fragile]\ft{\secname}
  然后,以下代码
  \begin{lstlisting}[language=c,backgroundcolor=\color{red!20}]
vector v1 = {3.0, 6.0};
vector v2;
v2 = v1;
  \end{lstlisting}
会被翻译成
  \begin{lstlisting}[language=c,backgroundcolor=\color{red!20}]
struct { double x; double y; } v1 = {3.0, 6.0};
struct { double x; double y; } v2;
v2 = v1;
  \end{lstlisting}
\end{frame}

\begin{frame}[fragile]\ft{\secname}
使用\verb|typedef|的另一个原因是\verb|typedef|的名称经常被用于复杂的类型。如
  \begin{lstlisting}[language=c,backgroundcolor=\color{red!20}]
typedef char(* FRPTC()) [5];
  \end{lstlisting}
这把\verb|FRPTC|声明为一个函数类型,该类型的函数返回一个指向含5个元素的\verb|char|数组的指针。
\end{frame}

\begin{frame}[fragile]\ft{\secname}
  切记:{使用\verb|typedef|并不创建新的类型,它只是创建了便于使用的标签。} 
\end{frame}
