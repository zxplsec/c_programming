\section{结构体变量}
\begin{frame}[fragile]\ft{\secname}
有了结构体声明,就可以创建一个结构体变量,如
\begin{lstlisting}
struct book library;  
\end{lstlisting}
这样,编译器会使用\verb|book|模板为该变量分配空间:
\begin{itemize}
  \item 一个长度为\verb|MAXTITL|的\verb|char|数组
  \item 一个长度为\verb|MAXAUTL|的\verb|char|数组
  \item 一个\verb|float|变量
\end{itemize}
这些变量是以一个名字\verb|library|结合在一起的。
\end{frame}

\begin{frame}[fragile]\ft{\secname}
  \begin{itemize}
  \item  在结构体变量的声明中,\verb|struct book|是一种新的数据类型,就如同\verb|int|或\verb|float|一样。\\[0.1in]
  \item 你可以创建一个\verb|struct book|类型的变量,也可以创建一个指向该结构体的指针。如
\begin{lstlisting}[language=c,backgroundcolor=\color{red!20}]
struct book lib1, lib2, * ptbook;  
\end{lstlisting}
  \end{itemize}
\end{frame}

\begin{frame}[fragile]\ft{\secname}

就计算机而言,声明
\begin{lstlisting}[language=c,numbers=left,frame=single]
struct book library;  
\end{lstlisting}
是以下声明的简化
\begin{lstlisting}[language=c,numbers=left,frame=single]
struct book
{
  char title[MAXTITL];
  char author[MAXAUTL];
  float value;
} library;  
\end{lstlisting}
也就是说,声明结构体的过程与定义结构体变量的过程可以被合并成一步。
\end{frame}

\begin{frame}[fragile]\ft{\secname}

将结构体声明与结构体变量定义合并在一起,是不需要使用标记的一种情况
\begin{lstlisting}[language=c,numbers=left,frame=single]
struct 
{
  char title[MAXTITL];
  char author[MAXAUTL];
  float value;
} library;  
\end{lstlisting}
然而,如果想多次使用一个结构体模板,就需要使用带标记的形式。
\end{frame}


\subsection{初始化结构体变量}

\begin{frame}[fragile]\ft{\secname:\subsecname}
要初始化一个结构体变量,可以这么做
\begin{lstlisting}[language=c,numbers=left,frame=single]
struct book library = {
  "C primer plus",
  "Stephan Prata",
  80
};  
\end{lstlisting}
即使用一个用花括号括起来的、用逗号隔开的初始化项目列表来进行初始化。
\begin{itemize}
\item 每个初始化项目必须与要初始化的结构体成员类型相匹配。
\item 建议把每个成员的初始化项目写在单独的一行。
\end{itemize}
\end{frame}

\subsection{访问结构体成员}

\begin{frame}[fragile]\ft{\secname:\subsecname}
  结构体像是一个“超级数组”,在这个超级数组内,一个元素可以是\verb|char|类型,下一个元素可以是\verb|float|类型,再下一个可以是\verb|int|数组。\vspace{0.1in}

  数组可以通过下标来访问每一个元素,那又如何访问结构体中的各个成员呢? \pause \vspace{0.1in}

 {\Large 用结构体成员运算符\verb|(.)|}。\vspace{0.1in}

  \verb|library.value|指的是\verb|library|的value成员,可以像使用任何其他\verb|float|变量那样使用\verb|library.value|。\vspace{0.1in}

从本质上讲,\verb|.title|、\verb|.author|和\verb|.value|在\verb|book|结构中扮演了下标的角色。
\end{frame}

\subsection{结构体的指定初始化项目}

\begin{frame}[fragile]\ft{\secname:\subsecname}
C99支持结构体的指定初始化项目,使用点运算符和成员名来标识具体的元素。
\begin{itemize}
\item 只初始化\verb|book|结构体的成员\verb|value|,可以这样做:
  \begin{lstlisting}[language=c,backgroundcolor=\color{red!20}]
struct book surprise = {.value = 20.50};    
  \end{lstlisting}
\item 可以按任意顺序使用指定初始化项目:
  \begin{lstlisting}[language=c,backgroundcolor=\color{red!20}]
struct book gift = {
  .value = 40.50,
  .author = "Dennis M. Ritchie",
  .title = "The C programming language"
};    
  \end{lstlisting}

\end{itemize}
\end{frame}
