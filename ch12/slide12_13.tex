\section{奇特的声明}

\begin{frame}[fragile]\ft{\secname}
  \begin{lstlisting}[language=c,backgroundcolor=\color{red!20}]
int board[8][8];     //`int数组的数组`
int ** ptr;          //`指向int的指针的指针`
int * risk[10];      
    //`具有10个元素的数组,每个元素是一个指向int的指针`
int (* rusk) [10];   
    //`一个指针,指向具有10个元素的int数组`
int * oof[3][4];
    //`一个3x4的数组,每个元素是一个指向int的指针`
int (* uuf) [3][4];
    //`一个指针,指向3X4的int数组`
int (* uof [3]) [4];
    //`一个具有3个元素的数组,每个元素是一个指向`
    //`具有4个元素的int数组的指针`
  \end{lstlisting}
\end{frame}

\begin{frame}[fragile]\ft{\secname}
  \begin{lstlisting}[language=c,backgroundcolor=\color{red!20}]
char * fump();    
   //`返回指向char的指针的函数`
char (* frump) ();     
   //`指向返回类型为char的函数的指针`
char (* flump[3]) (); 
   //`由3个指针组成的数组,每个指针指向返回值为char的函数`
  \end{lstlisting}
\end{frame}

\begin{frame}[fragile]\ft{\secname}

  \begin{lstlisting}[language=c,backgroundcolor=\color{red!20}]
typedef int arr5[5]
typedef arr5 * p_arr5;
typedef p_arr5 arrp10[10];
arr5 togs;
  //`togs为含5个元素的int数组`
p_arr5 p2;   
  //`p2为一个指针,指向具有5个元素的int数组`
arrp10 ap;
  //`ap是具有10个元素的指针数组,`
  //`每个指针指向具有5个元素的int数组`
  \end{lstlisting}

\end{frame}
