\section{常量与预处理器}

\begin{frame}[fragile]\ft{\secname}
\lstinputlisting[language=c,frame=single,numbers=left]
{ch04/code/circle1.c}
\end{frame}

\begin{frame}[fragile]\ft{\secname}
\begin{lstlisting}[backgroundcolor=\color{red!10}]
radius =  1.000000, circum = 6.283185, area = 3.141593
\end{lstlisting}
\end{frame}

\begin{frame}[fragile]\ft{\secname}
\lstinputlisting[language=c,frame=single,numbers=left]
{ch04/code/circle2.c}
\end{frame}

\begin{frame}[fragile]\ft{\secname}
\lstinputlisting[language=c,frame=single,numbers=left]
{ch04/code/circle3.c}
\end{frame}

\begin{frame}[fragile]\ft{\secname}
\lstinputlisting[language=c,frame=single,numbers=left]
{ch04/code/circle4.c}

\end{frame}

\begin{frame}[fragile]\ft{\secname:宏定义}
\begin{lstlisting}[title=宏定义的一般形式,backgroundcolor=\color{red!10}]
#define NAME value
\end{lstlisting}

\begin{itemize}
\item 没有使用分号是因为这是一种替代机制,而不是C的语句。\\[0.1in]
\item 符号常量请使用大写,其好处在于当看到它时便可立即知道是常量。\\[0.1in]
\item 符号常量的命名请遵循变量命名规则。
\end{itemize}
\end{frame}

\begin{frame}[fragile]\ft{\secname:宏定义}
\lstinline|#define|语句也可用于定义字符和字符串变量,前者用单引号,后者用双引号。
\vspace{0.1in}

\begin{lstlisting}
#define BEEP '\a'
#define TEE 'T'
#define ESC '\033'
#define OOPS "Now you have done it!"
\end{lstlisting}

\end{frame}

\begin{frame}[fragile]\ft{\secname:宏定义}
\lstinline|#define|语句也可用于定义字符和字符串变量,前者用单引号,后者用双引号。
\vspace{0.1in} \pause 

\begin{lstlisting}[title=常见错误]
#define B = 20
\end{lstlisting}
\vspace{0.1in}

如果这样做,\lstinline|B|将会被\lstinline|= 20|而不是\lstinline|20|代替。 这样以下语句
\begin{lstlisting}
c = a + B;
\end{lstlisting}
会被替换成如下错误的表达:
\begin{lstlisting}
c = a + = 20;
\end{lstlisting}
\end{frame}

\begin{frame}[fragile]\ft{\secname:\lstinline|const|修饰符}
C90允许使用关键字\lstinline|const|把一个变量声明转换为常量声明:
\begin{lstlisting}
const int MONTHS = 12; 
\end{lstlisting}
这使得\lstinline|MONTHS|成为一个只读值。你可以显示它,并把它用于计算中,但不能改变它的值。
\end{frame}
