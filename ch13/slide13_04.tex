\section{随机数函数与静态变量}

\begin{frame}[fragile]\ft{\secname} 
  C提供了{ rand()}来产生随机数,它使用了一个具有内部链接的静态变量。

  { rand()}是一个“伪随机数发生器”,这意味着可以预测数字的实际顺序,但这些数字在可能的取值范围内均匀地分布。
\end{frame}

\begin{frame}[fragile]\ft{\secname} 
  \lstinputlisting
  [language=c,frame=single]
  {ch13/code/rand0.c}
\end{frame}

\begin{frame}[fragile]\ft{\secname} 
  \lstinputlisting
  [language=c,frame=single]
  {ch13/code/r_drive0.c}
\end{frame}

\begin{frame}[fragile]\ft{\secname}
  先运行一次:
\begin{lstlisting}[backgroundcolor=\color{red!20}]
16838
5758
10113
17515
31051  
\end{lstlisting}
  再运行一次:
\begin{lstlisting}[backgroundcolor=\color{red!20}]
16838
5758
10113
17515
31051  
\end{lstlisting}
\end{frame}

\begin{frame}[fragile]\ft{\secname}
  \lstinputlisting
  [language=c,numbers=left,frame=single]
  {ch13/code/s_and_r.c}  
\end{frame}

\begin{frame}[fragile]\ft{\secname}
注意{ next}是一个具有内部链接的文件作用域变量,这意味着它可以同时被{ rand1()}和{ srand1()}使用,但不可以被其他文件中的函数使用。
\end{frame}

\begin{frame}[fragile,allowframebreaks]\ft{\secname}
  \lstinputlisting
  [language=c,numbers=left,frame=single]
  {ch13/code/r_drive1.c}  
\end{frame}

\begin{frame}[fragile,allowframebreaks]\ft{\secname}
  \begin{lstlisting}[backgroundcolor=\color{red!20}]
Please enter your choice for seed.
1
16838
5758
10113
17515
31051
Please enter next seed (q to quit)
2
908
22817
10239
12914
25837
Please enter next seed (q to quit)
q
Done
  \end{lstlisting}
\end{frame}