%This is a LaTeX template for homework assignments
\documentclass[8pt]{article}
\usepackage[top=1in, bottom=1in, left=1.1in, right=1.1in]{geometry}
\usepackage[utf8]{inputenc}
\usepackage{amsmath}
\usepackage{CJK}
\usepackage{enumerate}
\usepackage{ifthen}
\usepackage{listings}
\lstset{
  language=python,
  keywordstyle=\color{blue!70},
  frame=single,
  basicstyle=\ttfamily,
  commentstyle=\color{red},
  breakindent=0pt,
  rulesepcolor=\color{red!20!green!20!blue!20},
  rulecolor=\color{black},
  tabsize=4,
  numbersep=5pt,
  showstringspaces=false,
  breaklines=true,
  backgroundcolor=\color{red!10},
  showspaces=false,
  showtabs=false,
  extendedchars=false,
  escapeinside=``,
  frame=no,
}

\usepackage{fourier}
\usepackage{pgf}
\usepackage{tikz}
\usetikzlibrary{calc}
\usetikzlibrary{arrows,snakes,backgrounds,shapes,shadows}
\usetikzlibrary{matrix,fit,positioning,decorations.pathmorphing}



\newcommand{\tf}{\ttfamily}
\newcommand{\vv}{\boldsymbol v}

\newlength{\la}
\newlength{\lb}
\newlength{\lc}
\newlength{\ld}
\newlength{\lhalf}
\newlength{\lquarter}
\newlength{\lmax}
\newcommand{\xx}[4]{\\[.5pt]%
\settowidth{\la}{A.~#1~~}
\settowidth{\lb}{B.~#2~~}
\settowidth{\lc}{C.~#3~~}
\settowidth{\ld}{D.~#4~~}
%%
\ifthenelse{\lengthtest{\la>\lb}}
{\setlength{\lmax}{\la}}
{\setlength{\lmax}{\lb}}
\ifthenelse{\lengthtest{\lmax<\lc}}
{\setlength{\lmax}{\lc}}
{}
\ifthenelse{\lengthtest{\lmax<\ld}}
{\setlength{\lmax}{\ld}}
{}
%%
\setlength{\lhalf}{0.5\linewidth}
\setlength{\lquarter}{0.25\linewidth}
%%
\ifthenelse{\lengthtest{\lmax>\lhalf}}
{\noindent{}A.~#1 \\ B.~#2 \\ C.~#3 \\ D.~#4 }
{
\ifthenelse{\lengthtest{\lmax>\lquarter}}
{\noindent
\makebox[\lhalf][l]{A.~#1~~}%
\makebox[\lhalf][l]{B.~#2~~}\\
\makebox[\lhalf][l]{C.~#3~~}%
\makebox[\lhalf][l]{D.~#4~~}
}%
{\noindent
\makebox[\lquarter][l]{A.~#1~~}%
\makebox[\lquarter][l]{B.~#2~~}%
\makebox[\lquarter][l]{C.~#3~~}%
\makebox[\lquarter][l]{D.~#4~~}
}
}
}



\begin{document}
\begin{CJK}{UTF8}{gkai}
\begin{center}
{\Large \bf  武汉大学数学与统计学院2017-2018学年第二学期期末考试\\[0.1in]
  C语言上机(A卷)} \vspace{0.1in}

姓名: \line(1,0){100} ~~~~~~ 学号: \line(1,0){100}

\end{center}


\begin{enumerate} %\line(1,0){20}
\item (10分) 格式化输出:观察左边的运行结果,补充右边的程序:

\begin{minipage}{0.45\textwidth}
\begin{lstlisting}[showspaces=true]
*     456*
*456     *
*124.35*
*   124.350*
*1.243e+02 *
*   +124.35*
*0000124.35*
*    Hello World*
*          Hello*
*Hello          *
\end{lstlisting}
\end{minipage}\hfill
\begin{minipage}{0.45\linewidth}
\begin{lstlisting} 
#include<stdio.h>
#define N 456
#define X 124.35
#define S "Hello World"
int main(void)
{  
	...
   	return 0;
}
\end{lstlisting}
\end{minipage}

\item (15分)编写一个程序,把输入作为字符流读取,直到遇到EOF。利用switch结构,统计字符流中字符\lstinline|'a', 'b', 'c', 'd'|的个数(不区分大小写)。

\item (15分)编写递归函数,按顺序打印整数,其函数原型为
\begin{lstlisting}
void print(int n);
\end{lstlisting}
最后编写\lstinline|main()|函数测试之,程序运行的结果为
\begin{lstlisting}
Please enter an integer: 5
1  2  3  4  5  
\end{lstlisting}




\item (60分) 数组操作
\begin{enumerate}
\item (10分) 编写函数\lstinline|swap()|,实现两个数的交换;
\item (10分) 编写函数\lstinline|order_array()|,将数组从小到大进行排序。函数原型为
\begin{lstlisting}
void order_array(int * ar, int n);
\end{lstlisting}
\item(10分) 编写函数\lstinline|search_array()|,判断指定数有没有在数组中,若在返回\lstinline|true|,若不在返回\lstinline|false|。函数原型为
\begin{lstlisting}
bool search_array(int * ar, int n, int x);
\end{lstlisting}
\item (10分) 编写函数\lstinline|delete_array()|,删除下标为$i$的元素,并将其返回。其函数原型为
\begin{lstlisting}
int delete_array(int * ar, int n, int i);
\end{lstlisting}
\item (10分) 编写函数\lstinline|show_array()|,以行或列的方式输出数组。其函数原型为
\begin{lstlisting}
void show_array(int * ar, int n, char type);
\end{lstlisting}
这里,\lstinline|type == 'r'|表示按行输出,\lstinline|type == 'c'|表示按行输出,请在函数中使用switch结构。
\item (10分) 编写\lstinline|main()|函数,
\begin{itemize}
	\item 初始化一个长度为5的数组;
	\item 分别调用这些函数,输出相关信息;
	\item 请在必要时保护数组内容。
\end{itemize}
\end{enumerate}


\end{enumerate}



\end{CJK}
\end{document}
