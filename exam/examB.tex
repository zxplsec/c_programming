%This is a LaTeX template for homework assignments
\documentclass[8pt]{article}
\usepackage[top=1in, bottom=1in, left=1.1in, right=1.1in]{geometry}
\usepackage[utf8]{inputenc}
\usepackage{amsmath}
\usepackage{CJK}
\usepackage{enumerate}
\usepackage{ifthen}
\usepackage{listings}
\lstset{
  language=python,
  keywordstyle=\color{blue!70},
  frame=single,
  basicstyle=\ttfamily,
  commentstyle=\color{red},
  breakindent=0pt,
  rulesepcolor=\color{red!20!green!20!blue!20},
  rulecolor=\color{black},
  tabsize=4,
  numbersep=5pt,
  showstringspaces=false,
  breaklines=true,
  backgroundcolor=\color{red!10},
  showspaces=false,
  showtabs=false,
  extendedchars=false,
  escapeinside=``,
  frame=no,
}

\usepackage{fourier}
\usepackage{pgf}
\usepackage{tikz}
\usetikzlibrary{calc}
\usetikzlibrary{arrows,snakes,backgrounds,shapes,shadows}
\usetikzlibrary{matrix,fit,positioning,decorations.pathmorphing}



\newcommand{\tf}{\ttfamily}
\newcommand{\vv}{\boldsymbol v}

\newlength{\la}
\newlength{\lb}
\newlength{\lc}
\newlength{\ld}
\newlength{\lhalf}
\newlength{\lquarter}
\newlength{\lmax}
\newcommand{\xx}[4]{\\[.5pt]%
\settowidth{\la}{A.~#1~~}
\settowidth{\lb}{B.~#2~~}
\settowidth{\lc}{C.~#3~~}
\settowidth{\ld}{D.~#4~~}
%%
\ifthenelse{\lengthtest{\la>\lb}}
{\setlength{\lmax}{\la}}
{\setlength{\lmax}{\lb}}
\ifthenelse{\lengthtest{\lmax<\lc}}
{\setlength{\lmax}{\lc}}
{}
\ifthenelse{\lengthtest{\lmax<\ld}}
{\setlength{\lmax}{\ld}}
{}
%%
\setlength{\lhalf}{0.5\linewidth}
\setlength{\lquarter}{0.25\linewidth}
%%
\ifthenelse{\lengthtest{\lmax>\lhalf}}
{\noindent{}A.~#1 \\ B.~#2 \\ C.~#3 \\ D.~#4 }
{
\ifthenelse{\lengthtest{\lmax>\lquarter}}
{\noindent
\makebox[\lhalf][l]{A.~#1~~}%
\makebox[\lhalf][l]{B.~#2~~}\\
\makebox[\lhalf][l]{C.~#3~~}%
\makebox[\lhalf][l]{D.~#4~~}
}%
{\noindent
\makebox[\lquarter][l]{A.~#1~~}%
\makebox[\lquarter][l]{B.~#2~~}%
\makebox[\lquarter][l]{C.~#3~~}%
\makebox[\lquarter][l]{D.~#4~~}
}
}
}



\begin{document}
\begin{CJK}{UTF8}{gkai}
\begin{center}
{\Large \bf  武汉大学数学与统计学院2017-2018学年第二学期期末考试\\[0.1in]
  C语言上机(B卷)} \vspace{0.1in}

姓名: \line(1,0){100} ~~~~~~ 学号: \line(1,0){100}

\end{center}


\begin{enumerate} %\line(1,0){20}
\item (10分) 利用嵌套循环打印如下图案:
\begin{lstlisting}
   *
  ***
 *****
*******
 *****
  ***
   *
\end{lstlisting}

\item (15分)编写一个程序,把输入作为字符流读取,直到遇到EOF。统计大写字母、小写字母、数字字符和其他字符的个数。

\item (15分)编写递归函数,求一个整数各位数字之和,其函数原型为
\begin{lstlisting}
int sum_bits(int n);
\end{lstlisting}
最后编写\lstinline|main()|函数测试之,程序运行的结果为
\begin{lstlisting}[language=]
Please enter an integer: 3456
Sum of all digit bits for 3456 is 18.  
\end{lstlisting}



\item (60分) 向量操作:
\begin{enumerate}
\item  编写三个函数\lstinline|norm_1(), norm_2(), norm_inf()|,分别求向量的$1$范数、$2$范数和无穷范数:
$$
\begin{aligned}
\|\vv\|_1 &= |v_1|+|v_2|+\cdots+|v_n|, \quad
\|\vv\|_2 &= \sqrt{v_1^2+v_2^2+\cdots+v_n^2}, \quad
\|\vv\|_\infty &= \max\{|v_1|,|v_2|,\cdots,|v_n|\}\ 
\end{aligned}
$$
\item 编写函数\lstinline|scalar()|,求向量的数乘;
\item 编写函数\lstinline|dot_prod()|,求两向量的内积;
\item 编写函数\lstinline|angle()|,求两向量间的夹角;
\item 编写函数\lstinline|show()|,以行(type == 'r')或列(type == 'c')的方式输出数组。其函数原型为
\begin{lstlisting}
void show(int * ar, int n, char type);
\end{lstlisting}
\item 编写\lstinline|main()|函数,
\begin{itemize}
	\item 初始化一个长度为5的数组;
	\item 分别调用这些函数,输出相关信息;
	\item 请在必要时保护数组内容。
\end{itemize}
\end{enumerate}


\end{enumerate}



\end{CJK}
\end{document}
