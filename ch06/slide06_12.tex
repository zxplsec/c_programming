\section{使用函数返回值的循环}
\begin{frame}[fragile]\ft{\secname}
\begin{biancheng}
  写一个函数,计算一个数的整数次幂,如$n^p$。
\end{biancheng}
\rule{\textwidth}{1mm}\pause \vspace{0.1in}

请大家记住,头文件 \lstinline|math.h| 中提供了一个名为 \lstinline|pow| 的幂函数,允许计算浮点数次幂。


\end{frame}


\begin{frame}[fragile]\ft{\lstinline|pow()|的使用}

\lstinputlisting[numbers=left]{ch06/code/pow.c}

\end{frame}


\begin{frame}[fragile]\ft{\lstinline|pow()|的使用}


\begin{lstlisting}[backgroundcolor=\color{red!10}]
   2^3   = 8.000000
 sqrt 2  = 1.414214
 3^(1/4) = 1.316074
\end{lstlisting}

\end{frame}


\begin{frame}[fragile]\ft{\secname}
\begin{lstlisting}[numbers=left]
double power(double n, int p)
{
  double pow = 1;
  int i;
  
  for (i = 1; i <= p; i++)
    pow *= n;
    
  return pow;
}
\end{lstlisting}
\end{frame}


\begin{frame}[fragile]\ft{\secname}
写一个具有返回值的函数需要做以下事情:\vspace{0.1in}

\begin{itemize}
\item 定义函数时,说明它的返回值类型;\\[0.1in]
\item 使用关键字return指示要返回的值。
\end{itemize}
\end{frame}


\begin{frame}[fragile]\ft{\secname}
\begin{itemize}
\item 要声明函数类型,可以在函数名之前写出类型,就像声明一个变量时那样;\\[0.1in]
\item 关键字return使函数把它后面的值返回给调用函数。可以返回一个值,也可以返回一个表达式。
\begin{lstlisting}
return 2 * x + b;
\end{lstlisting}
\end{itemize}
\end{frame}


\begin{frame}[fragile]\ft{\secname}
在调用函数中,\vspace{0.1in}

\begin{itemize}
\item 可以把一个返回值赋给另一个变量。
\begin{lstlisting}
b = power(1.2, 3);
\end{lstlisting}
\item 可以把一个返回值作为一个表达式中的值。
\begin{lstlisting}
b = 2.0 + power(1.2, 3);
\end{lstlisting}
\item 也可以把一个返回值作为另一个函数的参数。
\begin{lstlisting}
printf("%f", power(1.2, 3));
\end{lstlisting}
\end{itemize}
\end{frame}

\begin{frame}[fragile,allowframebreaks]\ft{power函数的测试}
\lstinputlisting[numbers=left]{ch06/code/power.c}
\end{frame}


\begin{frame}[fragile]\ft{power函数的测试}
\begin{lstlisting}
Enter a number and the positive integer power
to which
the number will be raised. Enter qto quit.
1.2 12
1.2 to the power 12 is 8.9161
Enter next pair of numbers or q to quit.
2
16
2 to the power 16 is 65536
Enter next pair of numbers or q to quit.
q
\end{lstlisting}

\end{frame}
