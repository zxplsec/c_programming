\section{更多赋值运算符}
\begin{frame}[fragile]\ft{\secname}
\begin{lstlisting}
  +=   -=   *=   /=   %=
\end{lstlisting}
\end{frame}

\begin{frame}[fragile]\ft{\secname}
\begin{minipage}{.4\textwidth}
\begin{lstlisting}
num += 20
num -= 20
num /= 20
num *= 20
num %= 20
\end{lstlisting}
\end{minipage}$\blue~~~\Leftrightarrow~~~$
\begin{minipage}{.4\textwidth}
\begin{lstlisting}
num = num + 20
num = num - 20
num = num / 20
num = num * 20
num = num % 20
\end{lstlisting}
\end{minipage}
\pause 
\begin{minipage}{.4\textwidth}
\begin{lstlisting}
x *= 3 * y + 12
\end{lstlisting}
\end{minipage}$\blue~~~\Leftrightarrow~~~$
\begin{minipage}{.45\textwidth}
\begin{lstlisting}
x = x * (3 * y + 12)
\end{lstlisting}
\end{minipage}
\end{frame}

\begin{frame}[fragile]\ft{\secname}
\begin{itemize}
\item 
这些运算符具有与 \lstinline|=| 同样低的优先级。\\[0.1in]
\item
使用这些运算符可以让代码更为简洁,与长形式相比可能会产生效率更高的机器代码。
特别是变量名很长时,使用它们就显得非常有必要。如

\begin{lstlisting}
xxxxyyyyzzzz *= 3
\end{lstlisting}
\begin{lstlisting}
xxxxyyyyzzzz = xxxxyyyyzzzz * 3
\end{lstlisting}
\end{itemize}
\end{frame}
