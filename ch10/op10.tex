\section{数组与指针}

\begin{frame}[fragile]\ft{\secname}
\begin{biancheng} 
  编写一个函数,对数组按从小到大进行排序。简单排序算法原理:每次从左至右扫描序列,记下最小值的位置。
\end{biancheng}
\end{frame}


\begin{frame}[fragile,allowframebreaks]\ft{\secname}
\lstinputlisting
{ch10/code/ex10_01.c}
\end{frame}


\begin{frame}[fragile]\ft{\secname}
\begin{biancheng} 
编写一个程序,初始化一个\lstinline| double |数组,然后把数组内容复制到另外两个数组。
\begin{itemize}
\item 制作第一份拷贝的函数使用数组符号;
\item 制作第二份拷贝的函数使用指针符号,并使用指针的增量操作。
\item 把目标数组名和要复制的元素个数作为参数传递给函数。
\item 对一个长度为7的数组,请利用以上函数将其第3到第5个元素复制到一个长度为3的数组中。
\end{itemize}
\end{biancheng}
% \begin{lstlisting}
% double source[5] = {1.1, 2.2, 3.3, 4.4, 5.5};
% double target1[5], target2[5];
% copy_arr(source, target1, 5);
% copy_ptr(source, target2, 5);
% \end{lstlisting}
\end{frame}

\begin{frame}[fragile,allowframebreaks]\ft{\secname}
\lstinputlisting
{ch10/code/ex10_02.c}
\end{frame}

 



\begin{frame}[fragile]\ft{\secname}
\begin{biancheng} 
编写一个函数,求一个\lstinline| double |数组的最大值及其下标。
\end{biancheng}
\end{frame}

\begin{frame}[fragile,allowframebreaks]
\lstinputlisting
[language=c,numbers=left,frame=single]
{ch10/code/ex10_03.c}
\end{frame}

 

\begin{frame}[fragile]\ft{\secname}
\begin{biancheng} 
编写一个函数,将两个长度相同的数组相加,结果存储到第三个数组中。
\end{biancheng}
\end{frame}

\begin{frame}[fragile,allowframebreaks]\ft{\secname}
\lstinputlisting
[language=c,numbers=left,frame=single]
{ch10/code/ex10_04.c}
\end{frame}


 


\begin{frame}[fragile]\ft{\secname}
\begin{biancheng} 
编写一个函数,求两个三维向量的内积和外积。
\end{biancheng}
设
$$
\vec u = (a_1,a_2,a_3)^T, ~~~ 
\vec v = (b_1,b_2,b_3)^T
$$
则内积为
$$
\vec u \cdot \vec v = a_1b_1+a_2b_2+a_3b_3
$$
外积为
$$
\vec u \times \vec v = 
\left|
\begin{array}{ccc}
\mathbf i & \mathbf j & \mathbf k \\
a_1 & a_2 & a_3 \\
b_1 & b_2 & b_3
\end{array}
\right| = (a_2b_3-a_3b_2, a_3b_1-a_1b_3, a_1b_2-a_2b_1)^T.
$$
\end{frame}

\begin{frame}[fragile,allowframebreaks]\ft{\secname}
\lstinputlisting
[language=c,numbers=left,frame=single]
{ch10/code/ex10_05.c}
\end{frame}



\begin{frame}[fragile]
\begin{biancheng} 
编写一个函数,提示用户输入三个数集,每个数集包括5个\lstinline| double |值。程序应当实现以下功能:
\begin{enumerate}
\item 把输入信息存储到一个$3\times5$的数组中
\item 计算出每个数集的平均值
\item 计算所有数的平均值
\item 找出这$15$个数中的最大值
\item 打印出结果
\end{enumerate}
\end{biancheng}
\end{frame}

% \begin{frame}[fragile,allowframebreaks]
% \lstinputlisting
% [language=c,numbers=left,frame=single]
% {ch10/code/ex10_12.h}
% \end{frame}

% \begin{frame}[fragile,allowframebreaks]
% \lstinputlisting
% [language=c,numbers=left,frame=single]
% {ch10/code/ex10_12.c}
% \end{frame}

  

\begin{frame}[fragile]
\begin{biancheng} 
用变长数组重写以上程序。
\end{biancheng}
\end{frame}

\begin{frame}[fragile]
\begin{biancheng} 
用一维数组重写以上程序。
\end{biancheng}
\end{frame}
