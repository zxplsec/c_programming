\section{数组}

\begin{frame}[fragile]\ft{\secname}
\begin{itemize}
\item 数组由一系列类型相同的元素构成。\\[0.2in]
\item 数组声明必须包括元素的个数与类型。
\end{itemize}

\begin{lstlisting}[language=c,backgroundcolor=\color{red!20}]
float candy[365];
char code[12];
int states[50]; 
\end{lstlisting}

\end{frame}

\begin{frame}[fragile]\ft{\secname:数组初始化}
\begin{lstlisting}[language=c,backgroundcolor=\color{red!20}]
int array[6] = {1, 2, 4, 6, 8, 10};
\end{lstlisting}
可用初始化列表(花括号括起来的一系列数值,用逗号隔开)对数组进行初始化。
\end{frame}

\begin{frame}[fragile]\ft{\secname:数组初始化}
  \lstinputlisting
  [language=c,numbers=left,frame=single]
  {ch10/code/day_mon1.c}
\end{frame}

\begin{frame}[fragile]\ft{\secname:数组初始化}
\begin{lstlisting}[backgroundcolor=\color{red!20}]
Month  1 has 31 days.
Month  2 has 28 days.
Month  3 has 31 days.
Month  4 has 30 days.
Month  5 has 31 days.
Month  6 has 30 days.
Month  7 has 31 days.
Month  8 has 31 days.
Month  9 has 30 days.
Month 10 has 31 days.
Month 11 has 30 days.
Month 12 has 31 days.
\end{lstlisting}
\end{frame}

\begin{frame}[fragile]\ft{\secname:用const初始化数组}
\lstinputlisting
[language=c,numbers=left,frame=single]
{ch10/code/day_mon1_const.c}
\end{frame}

\begin{frame}[fragile]\ft{\secname:用const初始化数组}
若用关键字\lstinline| const |初始化数组,则该数组为只读数组,其元素均为常量,不允许修改。
\end{frame}

\begin{frame}[fragile]\ft{\secname:如果数组没有初始化,...}
\lstinputlisting
[language=c,numbers=left,frame=single]
{ch10/code/no_data.c}
\end{frame}

\begin{frame}[fragile]\ft{\secname:如果数组没有初始化,...}
\begin{lstlisting}[backgroundcolor=\color{red!20}]
 i    no_data[i]
 0             0
 1             0
 2             0
 3             0
 4    1606416376
 5         32767
\end{lstlisting}
\end{frame}

\begin{frame}[fragile]\ft{\secname:如果数组没有初始化,...}
同普通变量一样,数组元素的值在初始化之前是不确定的,编译器将使用存储单元中原有的数值。
\end{frame}

\begin{frame}[fragile]\ft{\secname:如果初始化列表的元素个数与数组大小不一致,...}
\lstinputlisting
[language=c,numbers=left,frame=single]
{ch10/code/some_data.c}
\end{frame}

\begin{frame}[fragile]\ft{\secname:如果初始化列表的元素个数与数组大小不一致,...}
\begin{lstlisting}[backgroundcolor=\color{red!20}]
 i    no_data[i]
 0            11
 1            12
 2             0
 3             0
 4             0
 5             0
\end{lstlisting}
\end{frame}

\begin{frame}[fragile]\ft{\secname:如果初始化列表的元素个数与数组大小不一致,...}
\begin{itemize}
\item
若不初始化数组,则数组元素存储的是无用的数值;\\[0.1in]
\item
若部分初始化数组,则未初始化的元素会被设置为0;\\[0.1in]
\item
若初始化列表中元素的个数大于数组大小,则编译器会警告或报错。
\end{itemize}
\end{frame}

\begin{frame}[fragile]\ft{\secname:如果初始化列表的元素个数与数组大小不一致,...}
可以省略括号中的数字,让编译器自动匹配数组大小与初始化列表中的项目个数。
\end{frame}

\begin{frame}[fragile]\ft{\secname:如果初始化列表的元素个数与数组大小不一致,...}
\lstinputlisting
[language=c,numbers=left,frame=single]
{ch10/code/day_mon2.c}
\end{frame}

\begin{frame}[fragile]\ft{\secname:如果初始化列表的元素个数与数组大小不一致,...}
  数组中元素个数等于整个数组的大小除以单个元素的大小,其中\lstinline| sizeof days |用于计算整个数组的大小,
  \lstinline| sizeof days[0] |用于计算一个元素的大小,均以字节为单位。
\end{frame}

\begin{frame}[fragile]\ft{\secname:指定初始化项目(C99)}
若想对数组的最后一个元素初始化,\vspace{0.1in}

\begin{itemize}
\item 
对于传统C,需要对每一个元素都初始化之后,才能对某个元素进行初始化。
\begin{lstlisting}[language=c,backgroundcolor=\color{red!20}]
  int arr[6] = {0, 0, 0, 0, 0, 212};
\end{lstlisting}
\pause\vspace{0.1in}

\item 
而对C99,在初始化列表中使用带方括号的元素下标可指定某个特定的元素:
\begin{lstlisting}[language=c,backgroundcolor=\color{red!20}]
int arr[6] = {[5] = 212}; //set arr[5] to 212
\end{lstlisting}
\end{itemize}
\end{frame}

\begin{frame}[fragile]\ft{\secname:指定初始化项目(C99)}
\lstinputlisting
[language=c,numbers=left,frame=single]
{ch10/code/designate.c}
\end{frame}

\begin{frame}[fragile]\ft{\secname:指定初始化项目(C99)}
\begin{lstlisting}[backgroundcolor=\color{red!20}]
Month  1 has 31 days.
Month  2 has 29 days.
Month  3 has  0 days.
Month  4 has  0 days.
Month  5 has 31 days.
Month  6 has 30 days.
Month  7 has 31 days.
Month  8 has  0 days.
Month  9 has  0 days.
Month 10 has  0 days.
Month 11 has  0 days.
Month 12 has  0 days.
\end{lstlisting}
\end{frame}

\begin{frame}[fragile]\ft{\secname:指定初始化项目(C99)}
\begin{itemize}
\item 
对于普通初始化,在初始化一个或多个元素后,未初始化元素都将被设置为0。\\[0.1in]
\item 
若在一个指定初始化项目后有不止一个值,则这些值将对后续的数组元素初始化。\\[0.1in]
\item
若多次对一个元素进行初始化,则最后一次有效。
\end{itemize}

\end{frame}

\begin{frame}[fragile]\ft{\secname:为数组赋值}
\begin{itemize}
\item 
声明完数组后,可用下标对数组成员赋值。\\[0.1in]
\item
C不支持把数组当做一个整体来进行赋值,也不支持用花括号括起来的列表形式进行赋值(初始化时除外)。
\end{itemize}
\end{frame}

\begin{frame}[fragile]\ft{\secname:为数组赋值}
\begin{lstlisting}[language=c,backgroundcolor=\color{red!20}]
int main(void)
{
  int arr1[5] = {1, 2, 3, 4};
  int arr2[5];
  
  arr2 = arr1;            // invalid
  arr2[5] = arr1[5];      // OK
  arr2[5] = {1, 2, 3, 4}; // invalid
}
\end{lstlisting}
\end{frame}

\begin{frame}[fragile]\ft{\secname:数组边界}
使用数组时,下标不能超过数组的边界。例如,假设你有这样的声明:
\begin{lstlisting}[language=c,backgroundcolor=\color{red!20}]
int arr[20];
\end{lstlisting}
那么你在使用下标时,要确保其范围在0到19之间,编译器不会检查这种错误。
\end{frame}

\begin{frame}[fragile,allowframebreaks]\ft{\secname:数组边界}
\lstinputlisting
[language=c,numbers=left,frame=single]
{ch10/code/bounds.c}
\end{frame}

\begin{frame}[fragile]\ft{\secname:数组边界}
\begin{lstlisting}[backgroundcolor=\color{red!20}]
value1 = 14, value2 = 88
-1 -1
 0 1
 1 3
 2 5
 3 7
 4 9
 5 32767
 6 424094912
value1 = -1, value2 = 9
\end{lstlisting}
使用超过数组边界的下标可能会改变其他变量的值。
\end{frame}

\begin{frame}[fragile]\ft{\secname:指定数组大小}
\begin{lstlisting}[language=c,backgroundcolor=\color{red!20}]
int n = 5;                     
int m = 8;                     
float a1[5];              //OK
float a2[5*2 + 1];        //OK
float a3[sizeof(int) + 1];//OK
float a4[-1];             //Invalid
float a5[0];              //Invalid
float a6[2.5];            //Invalid
float a7[(int) 2.5];      //OK, float to int
float a8[n];              //C99 OK
float a9[m];              //C99 OK
\end{lstlisting}
\end{frame}


