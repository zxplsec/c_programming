\section{关键概念}
\begin{frame}[fragile]\ft{\secname}
\begin{itemize}
\item
数组是一种派生类型,因它建立在其他类型之上。\\[0.1in]
\item
也就是说,你不是仅仅声明一个数组,而是声明了一个int数组、float数组或其他类型的数组。\\[0.1in]
\item
所谓的其他类型本身就可以是一种数组类型,此时便可得到数组的数组,即二维数组。
\end{itemize}
\end{frame}

\begin{frame}[fragile]\ft{\secname}
\begin{itemize}
\item
编写处理数组的函数是有好处的,因为使用特定的函数执行特定的功能有助于程序的模块化。\\[0.1in]
\item
\red{使用数组名作为实参时,要知道并不是把整个数组传递给函数,而是传递它的地址;因此对应的形参是一个指针。}\\[0.1in]
\item
处理数组时,函数必须知道数组的地址和元素的个数。数组地址直接传递给函数,数组元素的个数可在函数内部设置,也可当做参数传递给函数。但后者更为通用,这样可以处理不同大小的数组。
\end{itemize}
\end{frame}

\begin{frame}[fragile]\ft{\secname}
\begin{itemize}
\item 数组与指针之间联系紧密,指针符号和数组符号的运算往往可以互换使用。\\[0.1in]
\item 正是这个原因,才允许处理数组的函数使用指针(而不是数组)作为形参,同时在函数中使用数组符号处理数组。
\end{itemize}
\end{frame}
