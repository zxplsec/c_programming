\section{多维数组}

\begin{frame}[fragile]\ft{\secname} 
编制程序,计算出年降水总量、年降水平均量,以及月降水平均量。
\end{frame}

\begin{frame}[fragile,allowframebreaks]\ft{\secname} 
\lstinputlisting
[language=c,numbers=left,frame=single,breaklines=true]
{ch10/code/rain.c}
\end{frame}


\begin{frame}[fragile]\ft{\secname}
\begin{lstlisting}[basicstyle=\ttfamily\scriptsize,backgroundcolor=\color{red!20}]
YEAR RAINFALL (inches)
 2000            32.4
 2001            37.9
 2002            49.8
 2003            44.0 
 2004            32.9

The yearly average is 39.4 inches.

MONTHLY AVERAGES:
  Jan  Feb  Mar  Apr  May  Jun  Jul  Aug  Sep  Oct  Nov  Dec
  7.3  7.3  4.9  3.0  2.3  0.6  1.2  0.3  0.5  1.7  3.6  6.7
\end{lstlisting}
\end{frame}

\begin{frame}[fragile]\ft{\secname:初始化二维数组}
回忆一下一维数组的初始化:
\begin{lstlisting}[language=c,backgroundcolor=\color{red!20}]
sometype ar1[5] = {val1, val2, val3, val4, val5};
\end{lstlisting}
\end{frame}

\begin{frame}[fragile]\ft{\secname:初始化二维数组}
对于二维数组{\tf rain[5][12]}, \vspace{0.1in}

\begin{itemize}
\item
\red{它是包含5个元素的数组,而每个元素又是包含12个{\tf float}数的数组。}
\\[0.1in]
\item
对某个元素初始化,即用一个初始化列表对一个一维float数组进行初始化。\\[0.1in]
\item 
要对整个二维数组初始化,可以采用逗号隔开的5个初始化列表进行初始化。
\end{itemize}
\end{frame}

\begin{frame}[fragile]\ft{\secname:初始化二维数组}
\begin{lstlisting}[language=c,backgroundcolor=\color{red!20},basicstyle=\footnotesize\ttfamily]
const float rain[YEARS][MONTHS] =
{
  {4.3,4.3,4.3,3.0,2.0,1.2,0.2,0.2,0.4,2.4,3.5,6.6},
  {8.5,8.2,1.2,1.6,2.4,0.0,5.2,0.9,0.3,0.9,1.4,7.3},
  {9.1,8.5,6.7,4.3,2.1,0.8,0.2,0.2,1.1,2.3,6.1,8.4},
  {7.2,9.9,8.4,3.3,1.2,0.8,0.4,0.0,0.6,1.7,4.3,6.2},
  {7.6,5.6,3.8,2.8,3.8,0.2,0.0,0.0,0.0,1.3,2.6,5.2}
};
\end{lstlisting}
\end{frame}

\begin{frame}[fragile]\ft{\secname:初始化二维数组}
\begin{itemize}
\item
用了5个数值列表,都用花括号括起来。\\[0.05in]
\item
第一个列表赋给第一行,第二个列表赋给第二行,依此类推。\\[0.05in]
\item 
若第一个列表只有10个数值,则第一行前10个元素得以赋值,最后两个元素被设置为0。\\[0.05in]
\item 
若列表中数值个数多于12个,则会被警告或报错;这些数值不会影响到下一行的赋值。
\end{itemize}
\end{frame}

\begin{frame}[fragile]\ft{\secname:初始化二维数组}
初始化时,也可省略内部的花括号。 \vspace{0.05in}

\begin{itemize}
\item
只要保证数值的个数正确,初始化效果就是一样。\\[0.1in]
\item 
如果数值个数不够,则会按照先后顺序来逐行赋值,未被赋值的元素会被初始化为0。
\end{itemize}
\end{frame}

\begin{frame}[fragile]\ft{\secname:更多维数的数组}
三维数组的声明方式:
\begin{lstlisting}[language=c,backgroundcolor=\color{red!20}]
int box[10][20][30];
\end{lstlisting}
\begin{itemize}
\item
直观理解:一维数组是排成一行的数据,二维数组是放在一个平面上的数据,三维数组是把平面数据一层一层地叠起来。\\[0.1in]
\item 
另一种理解:三维数组是数组的数组的数组。即:{\tf box}是包含10个元素的数组,其中每个元素又是包含20个元素的数组,这20个元素的每一个又是包含30个元素的数组。
\end{itemize}
\end{frame}
