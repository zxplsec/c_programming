\section{函数、数组与指针}

\begin{frame}[fragile]\ft{\secname}
编写函数,求一个数组的各元素之和。
\end{frame}

\begin{frame}[fragile]\ft{\secname}
方式一:在函数中给定固定的数组大小。
\begin{lstlisting}[language=c,backgroundcolor=\color{red!20}]
int sum(int * ar)
{
  int i;
  int total = 0;
  
  for (i = 0; i < 10; i++)
    total += ar[i];
    
  return total;  
}
\end{lstlisting} \pause 
但该函数仅在数组长度为10时可工作。
\end{frame}

\begin{frame}[fragile]\ft{\secname}
方式二:将数组大小作为参数传递给函数。
\begin{lstlisting}[language=c,backgroundcolor=\color{red!20}]
int sum(int * ar, int n)
{
  int i;
  int total = 0;
  
  for (i = 0; i < n; i++)
    total += ar[i];
    
  return total;  
}
\end{lstlisting} \pause 
该方式更为灵活,第一个参数把数组地址和数组类型的信息传递给函数,第二个参数把数组的元素个数传递给函数。
\end{frame}

\begin{frame}[fragile]\ft{\secname}
在做函数声明时,以下四种函数原型是等价的:
\begin{lstlisting}[language=c,backgroundcolor=\color{red!20}]
int sum(int * ar, int n);
int sum(int *, int);

int sum(int ar[], int n);
int sum(int [], int);
\end{lstlisting}
\end{frame}

\begin{frame}[fragile]\ft{\secname}
在定义函数时,名称不可以省略。故在定义时以下两种形式是等价的:
\begin{lstlisting}[language=c,backgroundcolor=\color{red!20}]
int sum(int * ar, int n)
{
  ...
}

int sum(int ar[], int n)
{
  ...
}
\end{lstlisting}
\end{frame}

\begin{frame}[fragile]\ft{\secname}
观察以下程序,其功能是同时打印原数组的大小和代表数组的函数参量的大小。
\end{frame}

\begin{frame}[fragile,allowframebreaks]\ft{\secname}
\lstinputlisting
[language=c,numbers=left,frame=single]
{ch10/code/sum_arr1.c}
\end{frame}


\begin{frame}[fragile]\ft{\secname}
\begin{lstlisting}[backgroundcolor=\color{red!20}]
The size of ar is 8 bytes.
The total number of marbles is 190.
The size of marbles is 40 bytes.
\end{lstlisting}
\rule{\textwidth}{0.2mm} \pause 

\begin{itemize}
\item \lstinline| marbles |的大小为40字节,因\lstinline| marbles |包含10个\lstinline| int |数,每个数占4个字节。
\item \lstinline| ar |的大小为8个字节,因\lstinline| ar |本身并不是一个数组,而是一个指向\lstinline| marbles |首元素的指针。
\end{itemize}
\end{frame}

\begin{frame}[fragile]\ft{\secname:使用指针参数}
以上程序中,\lstinline| sum() |使用一个指针参量来确定数组的开始,使用一个整数参量来指明数组的元素个数。但这并不是向函数传递数组信息的唯一办法。
\end{frame}

\begin{frame}[fragile]\ft{\secname:使用指针参数}
另一种办法是传递两个指针,第一个指针指明数组的起始地址,第二个指针指明数组的结束地址。
\end{frame}

\begin{frame}[fragile,allowframebreaks]\ft{\secname:使用指针参数}
\lstinputlisting
[language=c,numbers=left,frame=single]
{ch10/code/sum_arr2.c}
\end{frame}


\begin{frame}[fragile]\ft{\secname:使用指针参数}
\begin{itemize}
\item 指针\lstinline| start |最初指向\lstinline| marbles |的首元素,执行赋值表达式\lstinline|total += *start|时,把首元素的值加到\lstinline| total |上。\\[0.1in]
\item 然后表达式\lstinline| start++ |使指针变量\lstinline| start |加1,\red{指向数组的下一个元素}。
\end{itemize}
\end{frame}

\begin{frame}[fragile]\ft{\secname:使用指针参数}
函数\lstinline| sump |使用第二个指针来控制循环次数:
\begin{lstlisting}[language=c,backgroundcolor=\color{red!20}]
while (start < end)
\end{lstlisting}
因这是对一个不相等关系的判断,故处理的最后一个元素将是\lstinline| end |所指向的位置之前的元素。
这就意味着\lstinline| end |实际指向的元素是在数组最后一个元素之后。\vspace{0.1in}\pause

\red{C保证在为数组分配存储空间时,指向数组之后的第一个位置的指针也是合法的。}
\vspace{0.1in}\pause

使用这种“越界”指针可使函数调用的形式更为整洁:
\begin{lstlisting}[language=c,backgroundcolor=\color{red!20}]
answer = sump(marbles, marbles + SIZE);
\end{lstlisting}
\end{frame}

\begin{frame}[fragile]\ft{\secname:使用指针参数}
如果让\lstinline| end |指向最后一个元素,函数调用的形式则为
\begin{lstlisting}[language=c,backgroundcolor=\color{red!20}]
answer = sump(marbles, marbles + SIZE - 1);
\end{lstlisting}
这种写法不仅看起来不整洁,也不容易被记住,从而容易导致编程错误。
\vspace{0.1in}\pause

\red{请记住:虽然C保证指针\lstinline| marbles+SIZE |是合法的,但对该地址存储的内容\lstinline| marbles[SIZE] |不作任何保证。}
\end{frame}

\begin{frame}[fragile]\ft{\secname:使用指针参数}
\begin{minipage}{0.4\textwidth}
\begin{lstlisting}[language=c,backgroundcolor=\color{red!20}]
total += *start;
start++;
\end{lstlisting}
\end{minipage}
~~$\Longleftrightarrow$~~
\begin{minipage}{0.4\textwidth}
\begin{lstlisting}[language=c,backgroundcolor=\color{red!20}]
total += *start++;
\end{lstlisting}
\end{minipage}
\begin{itemize}
\item
\lstinline| * |和\lstinline| ++ |优先级相同,从右往左结合。故\lstinline| ++ |应用于\lstinline| start |,而不是应用于\lstinline| *start |。也就是说,是指针自增1,而不是指针所指向的数据自增1。 \\[0.1in]\pause 
\item
后缀形式\lstinline| total += *start++ |表示先把指针指向的数据加到\lstinline| total |上,然后指针再自增1。\\[0.1in]\pause
\item
前缀形式\lstinline| total += *++start |表示指针先自增1,然后再把其指向的数据加到\lstinline| total |上。\\[0.1in]\pause
\item
若使用\lstinline| (*start)++ |,则会使用\lstinline| start |指向的数据,然后再使该数据自增1,而不是使指针自增1。此时,指针所指向的地址不变,但其中的元素会更新。
\end{itemize}
\end{frame}

\begin{frame}[fragile,allowframebreaks]\ft{\secname:关于优先级}
\lstinputlisting
[language=c,numbers=left,frame=single]
{ch10/code/order.c}
\end{frame}

\begin{frame}[fragile]\ft{\secname:关于优先级}
\begin{lstlisting}[backgroundcolor=\color{red!20}]
*p1 = 100, *p2 = 100, *p3 = 300
*p1++ = 100, *++p2 = 200, (*p3)++ = 300
*p1 = 200, *p2 = 200, *p3 = 301
\end{lstlisting}
\end{frame}

\begin{frame}[fragile]\ft{\secname:小结}
\begin{itemize}
\item 处理数组的函数实际上使用指针作为参数,不过在编写此类函数时,数组符号与指针符号都可以使用。若使用数组符号,则函数处理数组这一事实更加明显。\\[0.1in]
  
\item 在C中,以下两个表达式等价

\begin{minipage}{0.35\textwidth}
\begin{lstlisting}
       ar[i]
\end{lstlisting}
\end{minipage}
~~$\Longleftrightarrow$~~
\begin{minipage}{0.35\textwidth}
\begin{lstlisting}
*(ar+i)
\end{lstlisting}
\end{minipage}

不管\lstinline| ar |是一个数组名还是一个指针变量,这两个表达式都可以工作。然而只有当\lstinline| ar |是一个指针变量时,才可以使用\lstinline| ar++ |这样的表达式。
\\[0.1in]
\item
指针符号更接近机器语言,并且某些编译器在编译时能生成效率更高的代码。
\end{itemize}
\end{frame}
